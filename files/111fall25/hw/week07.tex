\documentclass[11pt,twoside]{amsart}
\usepackage{amssymb, amsmath, enumerate, libertine, microtype, hyperref,tikz-cd,hyperref}
\usepackage[normalem]{ulem}
\usepackage{fullpage}
\usepackage[T1]{fontenc}
\renewcommand{\labelitemi}{$\cdot$}
\usepackage{mathrsfs}
\usepackage{phaistos}
\definecolor{dark-red}{rgb}{0.4,0.15,0.15}
%   \definecolor{dark-blue}{rgb}{0.15,0.15,0.4}
%   \definecolor{medium-blue}{rgb}{0,0,0.5}
\setcounter{secnumdepth}{2}
\setcounter{tocdepth}{1}
\hypersetup{
    colorlinks, linkcolor=dark-red,
    citecolor=dark-red, urlcolor=dark-red
}


\theoremstyle{plain}
\newtheorem{prop}{Proposition}%[section]
\newtheorem{lemma}[prop]{Lemma}
\newtheorem{thm}[prop]{Theorem}
\newtheorem{obs}[prop]{Observation}
\newtheorem{app}[prop]{Application}
\newtheorem*{MainThm}{Main Theorem}
\newtheorem{cor}[prop]{Corollary}
\newtheorem{conj}[prop]{Conjecture}
\theoremstyle{remark}
\newtheorem{rmk}[prop]{Remark}
\newtheorem{prob}{Problem}
\newtheorem{bonus}[prop]{Bonus Problem}
\newtheorem{exc}{Exercise}
\theoremstyle{definition}
\newtheorem{ex}[prop]{Example}
\theoremstyle{definition}
\newtheorem{defn}[prop]{Definition}

\newcommand{\RR}{\mathbb{R}}
\newcommand{\ZZ}{\mathbb{Z}}
\newcommand{\CC}{\mathbb{C}}
\newcommand{\NN}{\mathbb{N}}
\newcommand{\QQ}{\mathbb{Q}}
\newcommand{\PP}{\mathbb{P}}
\newcommand{\kk}{\mathsf{k}}
\newcommand{\FF}{\mathbb{F}}
\newcommand{\cS}{\mathcal{S}}
\newcommand{\cT}{\mathcal{T}}
\newcommand{\ssC}{\mathsf{C}}

\newcommand{\id}{\operatorname{id}}
\newcommand{\Mat}{\mathsf{Mat}}

\title{Math 111: Calculus\\ Homework due Friday Week 7}
% uncomment the following line and add your name if you are using this as a template for solutions
% \author{Your Name}

\begin{document}
\maketitle

\noindent Make sure to review the homework instructions in the syllabus before writing your solutions. In particular, show your work and write in complete sentences (but also aim for concise explanations).

\begin{prob}
If a spherical tank of radius $4$ feet has $h$ feet of water to a height of $h$ feet present in the tank, then the volume of the water in the tank is given by
\[
  V = \frac{\pi}{3}h^2(12-h)
\]
in cubic feet. (You do not need to justify this, and may use it in your subsequent work.)
\begin{enumerate}[(a)]
\item What is the instantaneous rate of change of water in the tank with respect to height of the water at the instant $h=1$ foot? Make sure you include units in your answer.
\item Now suppose the height of the water is being regulated so that the height of the water at time $t$ is given by
\[
  h(t) = \sin(\pi t)+1
\]
where $t$ is measured in hours and $h(t)$ is still measured in feet. At what rate is the height of the water changing with respect to time at the instant $t=2$? Include units.
\item Continuing with the assumptions in (b), at what instantaneous rate is the volume of the water changing with respect to \emph{time} at the instant $t=2$?
\item What are the main differences between the rates found in (a) and (c)? Include a discussion of the relevant units.
\end{enumerate}
\end{prob}

\begin{prob}
Let $f(x)=x+\sin x$.
\begin{enumerate}[(a)]
\item Use desmos or another graphing utility to sketch a graph of $y=f(x)$ and explain why $f$ has an inverse function.
\item Spend at least ten minutes attempting to produce an explicit formula for $f^{-1}(x)$ and then write ``It appears that the inverse of $f$ cannot be expressed in terms of the functions we usually encounter in this class.''
\item Let $g(x)=f^{-1}(x)$ and compute
\[
  g'\left(\frac{\pi}{4}+\frac{\sqrt{2}}{2}\right).
\]
\item What is the equation of the tangent line to $y=f^{-1}(x)$ at $x=\pi$? (Make sure you explain why your answer works despite some trouble with the formula for $g'(\pi)$.)
\end{enumerate}
\end{prob}

\begin{prob}
For a constant $a$, consider the curve $D_a$ given by solutions to
\[
  y^2(y^2-1)=x^2(x^2-a).
\]
\begin{enumerate}[(a)]
\item Explain why $(0,1)$ is a point on $D_a$ for all $a$.
\item Use implicit differentiation to determine the equation for the tangent line to $D_a$ at $(0,1)$.
\item Use desmos or another graphing utility to sketch $D_a$ and its tangent line at $(0,1)$ for $a=0,1/2,1,2$.
\end{enumerate}
\end{prob}

\begin{prob}
Supoose $f\colon \mathbb{R}\to \mathbb{R}$ is differentiable and also equals its own inverse, that is,
\[
  f(f(x))=x\quad\text{for all $x$ in }\RR.
\]
Note that such functions exist; for instance, $f(x)=x$ and $f(x)=-x$ are both differentiable with $f(f(x))=x$.
\begin{enumerate}[(a)]
\item Find a third example of such a function. (Bonus: Find infinitely many such functions.)
\item Explain why $f'(x)\ne 0$ for all $x$. (Hint: Apply the chain rule to $f(f(x))=x$. Note: Your argument should work for \emph{all} such $f$, not a specific example.)
\item Call a real number $a$ a \emph{fixed point} of $f$ when $f(a)=a$. Explain why
\[
  f'(a) = \pm 1
\]
whenever $a$ is a fixed point of $f$.
\end{enumerate}
\end{prob}


\end{document}