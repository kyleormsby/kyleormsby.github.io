\documentclass[11pt,twoside]{amsart}
\usepackage{amssymb, amsmath, enumerate, libertine, hyperref,xcolor,booktabs,longtable,microtype,multirow}
\usepackage[normalem]{ulem}
\usepackage{fullpage}
\usepackage[T1]{fontenc}
\renewcommand{\labelitemi}{$\cdot$}
\usepackage[normalem]{ulem}
\definecolor{dark-red}{rgb}{0.4,0.15,0.15}
%   \definecolor{dark-blue}{rgb}{0.15,0.15,0.4}
%   \definecolor{medium-blue}{rgb}{0,0,0.5}
\setcounter{secnumdepth}{2}
\setcounter{tocdepth}{1}
\hypersetup{
    colorlinks, linkcolor=dark-red,
    citecolor=dark-red, urlcolor=dark-red
}

\title{Math 111 F01 \& F02: Calculus\\ Course Information \& Syllabus}
\author[Math 111 F01 \& F02:  Calculus]{Fall 2025}

\begin{document}
\maketitle

%\vspace{-5mm}
\thispagestyle{empty}

\vspace{-.5cm}

\begin{center}
\fbox{
\begin{minipage}{4.5in}
\begin{tabular}{rl}
Place: &Library 204\\
Time: &F01 --- MWF, 13:40-14:30\\
&F02 --- MWF, 14:40-15:30\\
Instructor: &Kyle Ormsby (\href{mailto:ormsbyk@reed.edu}{ormsbyk@reed.edu})\\
Office Hours: &W 15:30-16:30, Th 13:00-14:00 in Lib 306\\
Course Assistant: &Acadia Macdonald\\
Problem Sessions: &TBD\\
Textbook: &\href{https://activecalculus.org/acs2e/}{Active Calculus} (2nd ed.)\\
Website: &\href{https://kyleormsby.github.io/111/}{kyleormsby.github.io/111/}\\
Zulip: &\href{https://math111f01f02-2025.zulipchat.com/}{math111f01f02-2025.zulipchat.com/}
\end{tabular}
\end{minipage}
}
\end{center}

\smallskip

\subsection*{Course description}
Calculus --- the mathematical study of change --- is one of humanity's most fruitful inventions and is fundamental to the study of physical sciences, computer algorithms, and technology. This course is an introduction to differential and integral single variable calculus for liberal arts students with a focus on concepts and computations (but not proofs). We will explore fundamental mathematics and wide-ranging applications.

\subsection*{Learning outcomes}
Here are the overarching goals of our course. These ``learning outcomes'' are different from --- but aligned with --- our ``learning targets'' which appear later in the syllabus.
\begin{itemize}
\item Gain fluency with functions of a single real variable and be able to determine information about functions via limits, differentiation, and integration.
\item Interpret and apply the concepts of function, limit, derivative, and integral.
\item Conceptually understand and state the formal definitions of derivatives (as limits of difference quotients) and integrals (as Riemann sums).
\item State, understand, and apply both versions of the Fundamental Theorem of Calculus.
\item Compute derivatives via the power, product, quotient, and chain rules.
\item Compute indefinite and definite integrals via substitution and integration by parts.
\item Understand, manipulate, and compute derivatives and integrals of polynomial, rational, exponential, logarithmic, and trigonometric functions.
\item Solve problems involving optimization, implicit differentiation, and related rates.
\item Understand, manipulate, and compute basic Taylor series.
\item \textbf{Communicate mathematical ideas verbally and in writing}.
\end{itemize}

\subsection*{Distribution requirements}
This course can be used towards your Group III, ``Natural, Mathematical, and Psychological Science,'' requirement.  It accomplishes the following goals for the group:
\begin{itemize}
\item Use and evaluate quantitative data or modeling, or use logical/mathematical reasoning to evaluate, test, or prove statements.
\item Given a problem or question, formulate a hypothesis or conjecture, and design an experiment, collect data or use mathematical reasoning to test or validate it.
\end{itemize}
This course \textbf{does not} satisfy the ``primary data collection and analysis'' requirement.

\subsection*{Defining success}
The primary goal of this course is for you to \textbf{learn calculus}. To do so, you will need to do three things:
\begin{itemize}
\item \textbf{Engage:} Do reading assignments, submit reflection questions, attend class, and participate in collaborative problem-solving.
\item \textbf{Meet learning targets:} Demonstrate proficiency on the course's learning targets via your work on quizzes and the final exam.
\item \textbf{Synthesize and communicate:} Put it all together on advanced homework problems and write well-communicated solutions.
\end{itemize}
You can skip ahead in the syllabus to see how these three components translate into your final grade.

\subsection*{Text}
We will use \emph{Active Calculus} (2nd ed.) as a reference for the course.  The book is freely accessible  online at \href{https://activecalculus.org/acs2e/}{activecalculus.org/acs2e/}.  Each lecture will be paired with a \textbf{required reading} that you need to do before class.  Lectures and readings will be complementary, and both will be necessary to fully appreciate and learn the course content.

You will also need copies of the \emph{Active Calculus Activities Workbooks}. You can either write in print versions of the workbooks (available for purchase at the bookstore) or use a tablet computer to write on the PDF versions of the workbooks:
\begin{itemize}
\item \href{https://activecalculus.org/wp-content/uploads/2024/08/acs-activity-workbook-14-2024.pdf}{Ch.~1--4 Workbook} [pdf]
\item \href{https://activecalculus.org/wp-content/uploads/2024/08/acs-activity-workbook-58-2024.pdf}{Ch.~5--8 Workbook} [pdf]
\end{itemize}

I also \textbf{strongly recommend} doing the auto-assessed WeBWorK exercises at the end of each section of our textbook. (You'll need to use the HTML version of the book to access these.) The WeBWorK exercises are strictly for practice --- which is important! --- but will not count towards your final grade.

Finally, you can find video lectures on each section of the book on GVSUmath's YouTube channel:
\begin{itemize}
\item \href{https://www.youtube.com/playlist?list=PL9bIjQJDwfGuXQHuS5Jkmum_CFILoCZX-}{Chapters 1--4},
\item \href{https://www.youtube.com/playlist?list=PL9bIjQJDwfGtewW75Nw7PnGNSkfqwAm3v}{Chapters 5--6}.
\end{itemize}


\subsection*{Participation}
I will deliver interactive in-person lectures with frequent breaks for collaborative problem-solving. Your required reading will lay the groundwork for productively participating in these meetings. To demonstrate that you have completed the reading, and also allow me to adapt our course meetings to student needs, you are required to submit daily reading reflections via \href{https://docs.google.com/forms/d/e/1FAIpQLSd96ciBCGuT8PwRP9BCgPyGX9RjLKlX3jA72CQhwt1rdRaRDg/viewform?usp=sharing&ouid=109824948403640299404}{this Google Form}. 

Submitting reading reflections, attending class, and participating in collaborative problem-solving will form the \textbf{engagement} part of your course grade. If you miss a class, it is your responsibility to catch up with the material via the course notes, textbook, office hours, and discussions with peers.

All students are encouraged --- but not required --- to attend office hours and problem sessions (see below).

\subsection*{Quizzes}\label{quiz}
We will have an in-class quiz every other week that will give you the opportunity to demonstrate proficiency with our learning targets. The quizzes will focus on computational techniques, and should not require deep synthesis of ideas. Each quiz question is aligned with a learning target, and succeeding on these problems is how you \textbf{meet learning targets}.

\subsection*{Advanced homework}\label{hw}
In weeks when we do not have a quiz, you will complete an advanced homework assignment. These assignments will consist of two or three questions requiring synthesis and deeper thought. In your solutions, you are expected to communicate mathematics clearly in sentences, paragraphs, and notation. Excellent solutions take many forms, but they all have the following characteristics:
\begin{itemize}
\item they are written as explanations for other students in the course; in particular, they fully explain all of their reasoning and do not assume that the reader will fill in details;
\item when graphical reasoning is called for, they include large, carefully drawn and labeled diagrams;
\item they are neatly written or typeset;\footnote{Interested students are 
encouraged to prepare solutions in the \LaTeX~document preparation 
system.  A guide to \LaTeX~resources is available on the course 
website.  Nearly all of the \texttt{.pdf} files on the course website are produced by \LaTeX; you can find their associated source files by changing the \texttt{.pdf} suffix to \texttt{.tex} in the file's URL.} and
\item they use complete sentences, even when formulas or symbols are involved.
\end{itemize}

Correctly answering homework questions is how you \textbf{synthesize and communicate} your learning in calculus. Homework will be due via Gradescope every other Friday by 22:00. You are permitted and encouraged to work with your peers on homework problems.  You must cite those with whom you worked, and you must write up solutions independently.  \textbf{Duplicated solutions will not be accepted and constitute a violation of the Honor Principle.} I recommend starting the homework early so that you can take advantage of office hours and problem sessions.

\subsection*{Feedback}
You will receive timely feedback on your quizzes and homework via Gradescope, either from me or the course grader (a mathematics undergraduate).

Each problem will be assessed holistically and receive one of the following marks:
\begin{itemize}
\item \textbf{S}: Successful. Great job! Your solution is complete, correct, and well-communicated. There might be minor errors, but they are not central to the relevant learning target.
\item \textbf{R}: Revisable. Your work is complete and makes good progress on the problem, but contains a minor issue that needs clarification or correction. You should work to understand the issue and how to correct it, then come to an office hour to explain what happened and show me how to fix it. If you are convincing, the R will become an S; if not, it will revert to an N.
\item \textbf{N}: New attempt needed. You may have demonstrated partial understanding, but there is a major issue with your work that indicates a need for further study. You can try again on a future quiz or homework!
\end{itemize}

At the end of the semester, \textbf{only your S marks will count towards your final grade}. In particular, there is no partial credit. \textbf{You need two S marks on a given learning target in order to meet that target.}

\subsection*{Deadlines, revisions, and tokens}
Learning takes time and consistency. The deadlines and flow of work in this course are designed to facilitate your learning. I acknowledge, though, that sometimes your time won't cooperate with the course schedule. That's why each students will start the term with five \textbf{tokens}. The tokens have no value unless you use them, which you can do in the following ways:
\begin{itemize}
\item \textbf{Extend a homework or quiz deadline to the next class meeting.} To use a token in this fashion, simply submit the assignment by the revised deadline and include the word ``TOKEN'' on the assignment. (In the case of quizzes, you will need to arrange a makeup time with me.) You cannot extend reading reflection deadlines or make up absences with tokens.
\item \textbf{Revise N-marked homework or quiz problems.} You usually need to receive an R to revise a homework or quiz problem, but you can use a token to revise an R and ultimately earn an S.
\item \textbf{Other things.} If you think of something that a token \emph{should} be able to do that's not on this list, talk to me, and we might be able to arrange it.
\end{itemize}

\subsection*{Final exam}
We will have a two-hour final exam as scheduled by the registrar. The exam will be a lot like an extended quiz, and will provide a final opportunity for you to earn S marks on remaining learning targets. It will also include \emph{new} questions on our \textbf{core} learning topics.

\subsection*{Grades}\label{grading}
Your final grade in this course is determined by your engagement (reading reflections, attendance, and participation), meeting learning targets on quizzes and the final exam (two S marks to meet a target), and advanced homework assignments (synthesis and communication). For each primary letter grade (A through D) you need to meet minimum standards in \textbf{all} three categories:
\begin{itemize}
\item \textbf{A:} $\ge 16$ learning targets achieved, $\ge 90$\% of engagement points, $\ge 90$\% homework problems at S.
\item \textbf{B:} $\ge 13$ learning targets achieved, $\ge 75$\% of engagement points, $\ge 75$\% homework problems at S.
\item \textbf{C:} $\ge 10$ learning targets achieved, $\ge 65$\% of engagement points, $\ge 65$\% homework problems at S.
\item \textbf{D:} $\ge 8$ learning targets achieved, $\ge 50$\% of engagement points, $\ge 50$\% homework problems at S.
\end{itemize}
If a student does not meet a minimum requirement for a D, then they will receive an F for the course. While unlikely, I reserve the right to shift the total number of points or targets necessary for each grade, but such shifts will only make it easier to receive a given grade.

The final exam will include questions on each of the core learning targets. Receiving an S on these questions still counts towards the two S marks needed to satisfy that standard, but will also be used to adjust your final grade in the following fashion:
\begin{itemize}
\item 7 core standards earn an S --- raise grade by a + (\emph{e.g.}, B to B+);
\item 4, 5, or 6 core standards earn an S --- no change to grade;
\item 1, 2, or 3 core standards earn an S --- lower grade by a -- (\emph{e.g.}, B to B--);
\item 0 core standards earn an S --- lower grade by a full letter (\emph{e.g.}, B to C).
\end{itemize}
The grades of D and F do not admit + or -- adjustments at Reed, and thus these modifications do not apply to those grades.

\subsection*{Learning targets}
As you can see from the structure of this course, our learning targets are essential features of success. The table below lists all of our learning targets, organized by chapter in the textbook. Be aware that the standards numbering does \emph{not} correspond to section numbering in the textbook; furthermore, the chapter summaries are there to give a big picture view of each chapter's content; they are not learning targets in themselves.

\renewcommand{\arraystretch}{1.2}
\begin{center}
\begin{longtable}{@{}ll@{}}
\toprule
\textbf{Chapter 1} & Conceptual meaning and definitions of limits and derivatives.\\
D.1 & I can evaluate limits graphically and algebraically.\\
D.2 [core] & I can interpret the definition of the derivative.\\
D.3 & I can use derivatives to produce tangent lines to graphs and best linear approximations.\\
\midrule

\textbf{Chapter 2} & Apply differentiation rules and differentiate important classes of functions.\\
DR.1 & I can apply the linearity of the differentiation operator.\\
DR.2 & I can differentiate polynomial, exponential, and trigonometric functions.\\
DR.3 & I can evaluate derivatives of products and quotients.\\
DR.4 [core] & I can evaluate derivatives of composite functions via the chain rule.\\
\midrule

\textbf{Chapter 3} & Use derivatives to solve related rates and optimization problems.\\
AD.1 & I can set up and solve related rates problems using derivatives.\\
AD.2 [core] & I can use derivatives to find the extrema of a function.\\
AD.3 & I can set up and solve applied optimization problems.\\
\midrule

\textbf{Chapter 4} & Conceptual meaning and definition of the definite integral.\\
I.1 [core] & I can interpret the definition of the integral.\\
I.2 & I can approximate the value of an integral via Riemann sums.\\
I.3 [core] & I can use the first Fundamental Theorem of Calculus to evaluate a definite integral.\\
\midrule

\textbf{Chapter 5} & Use substitution and integration by parts to evaluate integrals.\\ 
IR.1 & I can use the second FTC to produce antiderivatives.\\
IR.2 [core] & I can use substitution to evaluate definite and indefinite integrals.\\
IR.3 & I can use integration by parts to evaluate definite and indefinite integrals.\\
\midrule

\textbf{Chapter 6} & Set up and solve problems using integrals.\\
AI.1 [core] & I can use integrals to compute area and length.\\
AI.2 & I can use integrals to compute volume, mass, and center of mass.\\ % \midrule

% \textbf{Chapter 8} & Use Taylor series to approximate functions.\\
% T.1 & I can interpret the definitions of Taylor polynomials and series.\\
% T.2 [core] & I can compute basic classes of Taylor series.\\ 
\bottomrule
\end{longtable}
\end{center}



\subsection*{Joint expectations}
As members of a communal learning environment, we should all expect consideration, fairness, patience, and curiosity from each other.  Our aim is to all learn more through cooperation and genuine listening and sharing, not to compete or show off.  I expect diligence and academic and intellectual honesty from each of you.  You should expect that I will do my best to focus the course on interesting, pertinent topics, and that I will provide feedback and guidance that will help you excel as a student.

\subsection*{Help}
There are a number of resources you can access for help with this course's content.  Everyone is welcome and encouraged to attend my office hours.  I am also happy to arrange office hours by appointment; please use \href{https://calendar.notion.so/meet/kyleormsby/office}{this link} to schedule a time.  Office hours are an opportunity to clarify difficult material and also delve deeper into topics that interest you.  Please reach out to me if there are barriers preventing you from effectively utilizing this opportunity.

Our course assistant will run a two-hour problem session each week at a time and location TBD.  We will share problem sessions with the other sections of Math 111.  The problem sessions will provide a structured, facilitated environment in which you can collaborate on homework.  All students are encouraged to attend.

The Mathematics \& Statistics Department also hosts drop-in tutoring on Sunday, Monday, Tuesday, Wednesday, and Thursday 19:00--21:00 in Lib 204. Upperclass tutors will be available to clarify concepts and help you with homework problems.

Finally, every Reed student is entitled to one hour of free individual tutoring per week.  Use the tutoring app in IRIS to arrange to work with a student tutor.

\subsection*{Zulip}
Our sections of Math 111 have a Zulip workspace.  Use the Zulip workspace to ask questions (of me or the class), collaborate on problems, and share resources. The Zulip workspace is an extension of our classroom and the above joint expectations extend to this setting.

You will receive an email invitation to join our Zulip workspace during the first week of classes. Please use channels and threads to keep conversations organized.

\subsection*{Technology}
The use of electronic devices (cell phones, computers, tablets, calculators, \emph{etc}.) is prohibited in the classroom without prior authorization from the instructor.  That said, legitimate uses of technology (\emph{e.g.}, note-taking or filling in the PDF versions of the workbooks) will be accommodated --- 
just talk to me first.

Students are not permitted to consult or utilize chatbot / large language model / ``AI'' utilities in their work.

\subsection*{The Internet}
You are welcome to use Internet resources to supplement content we cover in this course, with the exception of solutions to homework problems.  \textbf{Copying solutions from the Internet is an Honor Principle violation and will result in an academic misconduct report. The same goes for ``AI''-generated solutions.}

\subsection*{Academic accommodations}
If you have a documented disability requiring academic accommodation, please have  Disability \& Accessibility Resources (DAR)  provide a letter during the first week of classes.  I will then contact you to schedule a meeting during which we can discuss your accommodations.  If you believe you have an undocumented disability and that accommodations would ensure equal access to your Reed education, I would be happy to help you contact DAR.

\subsection*{Acknowledgments}
The structure and emphasis of this course benefits greatly from the conversations I've had with colleagues at Reed, especially Marcus Robinson, Jamie Pommersheim, and participants in Center for Teaching and Learning workshops. I am also immensely grateful to David Clark (Grand Valley State University) for sharing his insights on alternative grading and course structure. Many of the features of this syllabus and course design were adapted from his Fall 2024 Math 202 (Calculus 2) syllabus.

\bigskip \bigskip

\begin{center}
Remember: \emph{Math is hard, but we'll get through this together!}
\end{center}


\end{document}