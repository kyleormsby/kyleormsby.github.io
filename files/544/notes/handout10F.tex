\documentclass[11pt,twoside]{amsart}
\usepackage{amssymb, amsmath, enumerate, palatino, hyperref}
\usepackage[normalem]{ulem}
\usepackage{fullpage}
\usepackage[T1]{fontenc}
\renewcommand{\labelitemi}{\guillemotright}
\usepackage{mathrsfs}
\usepackage{phaistos}
\usepackage{tikz}
\definecolor{dark-red}{rgb}{0.4,0.15,0.15}
\hypersetup{
    colorlinks, linkcolor=dark-red,
    citecolor=dark-red, urlcolor=dark-red
}


\theoremstyle{plain}
\newtheorem{prop}{Proposition}%[section]
\newtheorem{lemma}[prop]{Lemma}
\newtheorem{thm}[prop]{Theorem}
\newtheorem{obs}[prop]{Observation}
\newtheorem{app}[prop]{Application}
\newtheorem*{MainThm}{Main Theorem}
\newtheorem{cor}[prop]{Corollary}
\newtheorem{conj}[prop]{Conjecture}
\theoremstyle{remark}
\newtheorem{rmk}[prop]{Remark}
\newtheorem{prob}{Problem}
\newtheorem{bonus}[prop]{Bonus Problem}
\newtheorem{exc}{Exercise}
\theoremstyle{definition}
\newtheorem{ex}[prop]{Example}
\theoremstyle{definition}
\newtheorem{defn}[prop]{Definition}

\newcommand{\RR}{\mathbb{R}}
\newcommand{\ZZ}{\mathbb{Z}}
\newcommand{\CC}{\mathbb{C}}
\newcommand{\NN}{\mathbb{N}}
\newcommand{\QQ}{\mathbb{Q}}
\newcommand{\PP}{\mathbb{P}}
\newcommand{\kk}{\mathsf{k}}
\newcommand{\FF}{\mathbb{F}}
\newcommand{\cS}{\mathcal{S}}
\newcommand{\cT}{\mathcal{T}}
\newcommand{\ssC}{\mathsf{C}}
\newcommand{\sU}{\mathscr{U}}
\newcommand{\ol}{\overline}

\newcommand{\id}{\operatorname{id}}
\newcommand{\Int}{\operatorname{Int}}
\title{Math 544: Topology\\ Friday Week 10}
%\author{Your Name}

\begin{document}
\maketitle

Recall that $S^1$ is the unit circle centered at $0$ in $\RR^2$. Now consider some circles in space:
\[
\begin{aligned}
  C_0 &:= S^1\times 0,\\
  C_1 &:= (3,0,0)+S^1\times 0,\\
  C_2 &:= (0,1/2,0)+0\times S^1.
\end{aligned}
\]
Then we may form the following link\footnote{A \emph{link} is an embedding of a disjoint union of circles into space.} complement spaces
\[
\begin{aligned}
  X &:= \RR^3\smallsetminus (C_0 \cup C_1),\\
  Y &:= \RR^3\smallsetminus (C_0 \cup C_2).
\end{aligned}
\]
More colloquially, $X$ is the complement of two unlinked circles, and $Y$ is the complement of two linked circles.

\begin{prob}
Apply the Seifert--van Kampen theorem to determine the fundamental groups of $X$ and $Y$.
\end{prob}

We are now going to physically model the spaces $X$ and $Y$ using the ambient universe and some carabiners. When a pair of carabiners is unlinked, the universe minus those carabiners is $X$; after linking the carabiners, we get $Y$.

\begin{prob}
Choose (path homotopy classes of) loops $a,b$ such $\pi_1X$ is generated by $a$ and $b$.
\begin{enumerate}[(a)]
\item Use the provided cord and carabiners to model the loop $[a,b]=aba^{-1}b^{-1}$.
\item What does your computation from Problem 1 tell you about $[a,b]$?
\end{enumerate}
\end{prob}

\begin{prob}
Retaining the cord configuration you created in Problem 2, carefully link the two carabiners so that you now have a loop in $Y$.
\begin{enumerate}[(a)]
\item Argue that the resulting loop in $Y$ is $[c,d]$ for $c,d$ generators of $\pi_1Y$.
\item What does your computation from Problem 1 tell you about $[c,d]$?
\item Verify your assertion pulling on the cord.
\end{enumerate}
\end{prob}

\hrulefill
\bigskip

Knot theory is the study of tame embeddings $S^1\subseteq \RR^3$ (or $S^1\subseteq S^3$) up to ambient isotopy. Here \emph{tame} means that the embedding can be extended to a solid torus (\emph{i.e.} ``thickened'') and an \emph{ambient isotopy} between knots $K,L\colon S^1\hookrightarrow \RR^3$ is a homotopy
\[
  H\colon \RR^3\times I\to \RR^3
\]
such that each $H(-,t)$ is a homeomorphism, $H(-,0) = \id_{\RR^3}$, and $H(-,1)\circ K = L$. These definition can be extended to links (where $S^1$ is replaced by a disjoint union of circles).

The \emph{knot group} of a link $K$ is $\pi_1(\RR^3\smallsetminus K)$. In the above problems, we studied the knot groups of a trivial link with two components (\emph{i.e.} $\pi_1X$) and the knot group of the \emph{Hopf link} (\emph{i.e.} $\pi_1Y$).

The \emph{trefoil knot} $T$ has parametrization
\[
  t\longmapsto ((2+\cos(3t))\cos(2t),(2+\cos(3t))\sin(2t),\sin(3t))
\]
viewed as a path $[0,2\pi]\to \RR^3$ (which naturally descends to an embedding $T\colon S^1\cong \RR/2\pi\ZZ\hookrightarrow \RR^3$).

\begin{prob}
In this problem, you will determine the knot group of $T$.
\begin{enumerate}[(a)]
\item Show that $T$ lives in a torus inside $\RR^3$.
\item Let $U$ be an open thickening of the torus minus $T$ and let $V$ be the complement of an appropriate closed thickening of the torus (so that $U\cup V = \RR^3\smallsetminus T$). Use the Seifert--van Kampen theorem applied to $U$ and $V$ to prove that
\[
  \pi_1(\RR^3\smallsetminus T) \cong \langle a,b\mid a^2=b^3\rangle.
\]

\end{enumerate}
\end{prob}

\hrulefill
\bigskip

The group $\langle a,b\mid a^2=b^3\rangle$ is a presentation of the \emph{braid group} $B_3$ on three strands. We will see it in a different guise next week.

Given a link diagram (a nice projection of your link onto the plane with under- and over-crossings recroded), the \emph{Wirtinger presentation} allows you to produce a presentation for the associated knot group. The proof is a direct application of the Seifert--van Kampen theorem. See D.~Rolfsen, \emph{Knots and links}, \S 3D for details.
\end{document}