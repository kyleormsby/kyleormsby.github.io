\documentclass[11pt,twoside]{amsart}
\usepackage{amssymb, amsmath, enumerate, palatino, hyperref,tikz,tikz-cd}
\usepackage[normalem]{ulem}
\usepackage{fullpage}
\usepackage[T1]{fontenc}
\renewcommand{\labelitemi}{\guillemotright}
\usepackage{mathrsfs}
\usepackage{phaistos}


\theoremstyle{plain}
\newtheorem{prop}{Proposition}%[section]
\newtheorem{lemma}[prop]{Lemma}
\newtheorem{thm}[prop]{Theorem}
\newtheorem{obs}[prop]{Observation}
\newtheorem{app}[prop]{Application}
\newtheorem*{MainThm}{Main Theorem}
\newtheorem{cor}[prop]{Corollary}
\newtheorem{conj}[prop]{Conjecture}
\theoremstyle{remark}
\newtheorem{rmk}[prop]{Remark}
\newtheorem{prob}{Problem}
\newtheorem{bonus}[prop]{Bonus Problem}
\newtheorem{exc}{Exercise}
\theoremstyle{definition}
\newtheorem{ex}[prop]{Example}
\theoremstyle{definition}
\newtheorem{defn}[prop]{Definition}

\newcommand{\RR}{\mathbb{R}}
\newcommand{\ZZ}{\mathbb{Z}}
\newcommand{\CC}{\mathbb{C}}
\newcommand{\NN}{\mathbb{N}}
\newcommand{\QQ}{\mathbb{Q}}
\newcommand{\PP}{\mathbb{P}}
\newcommand{\kk}{\mathsf{k}}
\newcommand{\FF}{\mathbb{F}}
\newcommand{\cS}{\mathcal{S}}
\newcommand{\cT}{\mathcal{T}}
\newcommand{\ssC}{\mathsf{C}}
\newcommand{\sU}{\mathscr{U}}
\newcommand{\ol}{\overline}

\newcommand{\id}{\operatorname{id}}
\newcommand{\Int}{\operatorname{Int}}
\newcommand{\cs}{\mathbin{\#}}
\newcommand{\Ab}{\mathsf{Ab}}
\newcommand{\Top}{\mathsf{Top}}
\newcommand{\Grp}{\mathsf{Grp}}


\title{Math 544: Topology\\ Final Exam Practice Problems}
%\author{Your Name}

\begin{document}
\maketitle

\noindent Use these problems to prepare for your final oral exam. You are welcome to collaborate on them. I will ask you about at least one of these problems during your oral exam.

\begin{prob}
A topological space is called \emph{$\sigma$-compact} if it can be expressed as a union of countably many compact subspaces. Show that a locally Euclidean Hausdorff space is a topological manifold if and only if it is $\sigma$-compact.
\end{prob}

\begin{prob}
Prove that the surfaces with polygonal presentations
\[
  \langle a,b,c,d\mid ad^{-1}cdc^{-1}ba^{-1}b^{-1}\rangle
\]
and
\[
  \langle e,f,g,h\mid ehgfe^{-1}h^{-1}g^{-1}f^{-1}\rangle
\]
are homeomorphic.
\end{prob}

\begin{prob}
Recall that $\pi_1(X,p)$ may be regarded as the collection of based homotopy classes of pointed maps $(S^1,1)\to (X,p)$. Let $[S^1,X]$ denote the set of homotopy classes of maps $S^1\to X$ (with no conditions on basepoints). Let $\Phi\colon \pi_1(X,p)\to [S^1,X]$ denote the function that forgets basepoints. Show that $\Phi([f]) = \Phi([g])$ if and only if $[f]$ and $[g]$ are conjugate in $\pi_1(X,p)$. Conclude that when $X$ is path connected, $\Phi$ induces a bijection between $[S^1,X]$ and conjugacy classes in $\pi_1(X,p)$.
\end{prob}

\begin{prob}
Let $X$ denote the union of $n$ lines through the origin in $\RR^3$. Compute $\pi_1(\RR^3\smallsetminus X)$.
\end{prob}

\begin{prob}
Let $n,s\in S^2$ denote the north and south poles of $S^2$ and let $X := S^2/\{n,s\}$. Give $X$ the structure of a CW-complex and use this to compute $\pi_1X$.
\end{prob}


\end{document}