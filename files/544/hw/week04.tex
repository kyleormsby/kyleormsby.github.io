\documentclass[11pt,twoside]{amsart}
\usepackage{amssymb, amsmath, enumerate, palatino, hyperref}
\usepackage[normalem]{ulem}
\usepackage{fullpage}
\usepackage[T1]{fontenc}
\renewcommand{\labelitemi}{\guillemotright}
\usepackage{mathrsfs}
\usepackage{phaistos}


\theoremstyle{plain}
\newtheorem{prop}{Proposition}%[section]
\newtheorem{lemma}[prop]{Lemma}
\newtheorem{thm}[prop]{Theorem}
\newtheorem{obs}[prop]{Observation}
\newtheorem{app}[prop]{Application}
\newtheorem*{MainThm}{Main Theorem}
\newtheorem{cor}[prop]{Corollary}
\newtheorem{conj}[prop]{Conjecture}
\theoremstyle{remark}
\newtheorem{rmk}[prop]{Remark}
\newtheorem{prob}{Problem}
\newtheorem{bonus}[prop]{Bonus Problem}
\newtheorem{exc}{Exercise}
\theoremstyle{definition}
\newtheorem{ex}[prop]{Example}
\theoremstyle{definition}
\newtheorem{defn}[prop]{Definition}

\newcommand{\RR}{\mathbb{R}}
\newcommand{\ZZ}{\mathbb{Z}}
\newcommand{\CC}{\mathbb{C}}
\newcommand{\NN}{\mathbb{N}}
\newcommand{\QQ}{\mathbb{Q}}
\newcommand{\PP}{\mathbb{P}}
\newcommand{\kk}{\mathsf{k}}
\newcommand{\FF}{\mathbb{F}}
\newcommand{\cS}{\mathcal{S}}
\newcommand{\cT}{\mathcal{T}}
\newcommand{\ssC}{\mathsf{C}}
\newcommand{\sU}{\mathscr{U}}
\newcommand{\ol}{\overline}

\newcommand{\id}{\operatorname{id}}
\newcommand{\Int}{\operatorname{Int}}
\newcommand{\cs}{\mathbin{\#}}
\title{Math 544: Topology\\ Homework due Friday Week 4}
%\author{Your Name}

\begin{document}
\maketitle

\noindent Problems taken from \emph{Introduction to Topological Manifolds} are marked ITM $x$--$y$. Please review the syllabus for expectations and policies regarding homework.

\begin{prob}[ITM 4--1,2]
Show that for $n>1$, $\RR^n$ is not homeomorphic to any open subset of $\RR$. [\emph{Hint}: if $U\subseteq \RR$ is open and $x\in U$, then $U\smallsetminus \{x\}$ is not connected.] Use this to prove that a nonempty topological space cannot be both a $1$-manifold and an $n$-manifold for some $n>1$.
\end{prob}

\begin{prob}[ITM 4--15]
Suppose that $G$ is a topological group.
\begin{enumerate}[(a)]
\item Show that every open subgroup of $G$ is also closed.
\item For any neighborhood $U$ of $1$, show that the subgroup $\langle U\rangle$ generated by $U$ is open and closed in $G$.
\item For any connected subset $U\subseteq G$ containing $1$, show that $\langle U\rangle$ is connected.
\item Show that if $G$ is connected, then every connected neighborhood of $1$ generates $G$.
\end{enumerate}
\end{prob}

\begin{prob}[ITM 4--18]
Let $M_1$ and $M_2$ be $n$-manifolds. For $i=1,2$, let $B_i\subseteq M_i$ be regular coordinate balls, and let $M_i' = M_i\smallsetminus B_i$. Choose a homeomorphism $f\colon \partial M_2'\to \partial M_1'$. (You may assume that $\partial M_i'\cong S^{n-1}$ for $i=1,2$ and thus such an $f$ exists.) Let $M_1 \cs M_2$ (called the \emph{connected sum} of $M_1$ and $M_2$) be the adjunction space $M_1'\cup_f M_2'$.
\begin{enumerate}[(a)]
\item Show that $M_1\cs M_2$ is an $n$-manifold (without boundary).
\item Show that if $M_1$ and $M_2$ are connected and $n>1$, then $M_1\cs M_2$ is connected. (Is this also true for $n=1$?)
\item Show that if $M_1$ and $M_2$ are compact, then $M_1\cs M_2$ is compact.
\end{enumerate}
\end{prob}

\begin{prob}[ITM 4--13]
Let $X$ be a locally compact Hausdorff space. The \emph{one-point compactification} of $X$ is the topological space $X^*$ defined as follows: For $\infty$ some object not in $X$, let $X^* = X\amalg \{\infty\}$ with the following topology $\mathscr T$ consisting of open subsets of $X$ along with $U\subseteq X^*$ such that $X^*\smallsetminus U$ is a compact subset of $X$.
\begin{enumerate}[(a)]
\item Show that $\mathscr T$ is a topology.
\item Show that $X^*$ is a compact Hausdorff space.
\item Use stereographic projection to show that $S^n \cong (\RR^n)^*$.
\item Show that $X$ is open in $X^*$ and has the subspace topology.
\item Show that $X$ is dense in $X^*$ if and only if $X$ is noncompact.
\end{enumerate}
\end{prob}

\begin{prob}
Let $X$ be a metric space with metric $d$. Given $x\in X$ and nonempty $A\subseteq X$, define
\[
  d(x,A) := \inf\{d(x,a)\mid a\in A\}.
\]
Give a direct proof of Urysohn's lemma for metric spaces by considering the function
\[
  f(x) = \frac{d(x,A)}{d(x,A)+d(x,B)}.
\]
\end{prob}



\end{document}