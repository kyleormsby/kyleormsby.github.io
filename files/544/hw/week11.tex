\documentclass[11pt,twoside]{amsart}
\usepackage{amssymb, amsmath, enumerate, palatino, hyperref,tikz,tikz-cd}
\usepackage[normalem]{ulem}
\usepackage{fullpage}
\usepackage[T1]{fontenc}
\renewcommand{\labelitemi}{\guillemotright}
\usepackage{mathrsfs}
\usepackage{phaistos}


\theoremstyle{plain}
\newtheorem{prop}{Proposition}%[section]
\newtheorem{lemma}[prop]{Lemma}
\newtheorem{thm}[prop]{Theorem}
\newtheorem{obs}[prop]{Observation}
\newtheorem{app}[prop]{Application}
\newtheorem*{MainThm}{Main Theorem}
\newtheorem{cor}[prop]{Corollary}
\newtheorem{conj}[prop]{Conjecture}
\theoremstyle{remark}
\newtheorem{rmk}[prop]{Remark}
\newtheorem{prob}{Problem}
\newtheorem{bonus}[prop]{Bonus Problem}
\newtheorem{exc}{Exercise}
\theoremstyle{definition}
\newtheorem{ex}[prop]{Example}
\theoremstyle{definition}
\newtheorem{defn}[prop]{Definition}

\newcommand{\RR}{\mathbb{R}}
\newcommand{\ZZ}{\mathbb{Z}}
\newcommand{\CC}{\mathbb{C}}
\newcommand{\NN}{\mathbb{N}}
\newcommand{\QQ}{\mathbb{Q}}
\newcommand{\PP}{\mathbb{P}}
\newcommand{\kk}{\mathsf{k}}
\newcommand{\FF}{\mathbb{F}}
\newcommand{\cS}{\mathcal{S}}
\newcommand{\cT}{\mathcal{T}}
\newcommand{\ssC}{\mathsf{C}}
\newcommand{\sU}{\mathscr{U}}
\newcommand{\ol}{\overline}

\newcommand{\id}{\operatorname{id}}
\newcommand{\Int}{\operatorname{Int}}
\newcommand{\cs}{\mathbin{\#}}
\newcommand{\Ab}{\mathsf{Ab}}
\newcommand{\Top}{\mathsf{Top}}
\newcommand{\Grp}{\mathsf{Grp}}


\title{Math 544: Topology\\ Homework due Friday Week 11}
%\author{Your Name}

\begin{document}
\maketitle

\noindent Problems taken from \emph{Introduction to Topological Manifolds} are marked ITM $x$--$y$. Please review the syllabus for expectations and policies regarding homework.

\begin{prob}[ITM 10--1]
Use the Seifert--van Kampen theorem to give another proof that $S^n$ is simply connected when $n\ge 2$.
\end{prob}

\begin{prob}[ITM 10--2]
Let
\[
  X := S^2 \cup \{(0,0,z)\mid -1\le z\le 1\}\subseteq \RR^3
\]
and let $N = (0,0,1)\in X$. Compute $\pi_1(X,N)$, giving explicit generators.
\end{prob}

\begin{prob}[ITM 10--7]
Suppose $M$ and $N$ are connected $n$-manifolds with $n\ge 3$. Prove that the fundamental group of $M\cs N$ is isomorphic to $\pi_1(M)*\pi_1(N)$. [\emph{Hint}: Prove ITM 4--19 and 10--6 and then use these results.]
\end{prob}

\begin{prob}[ITM 10--11]
For each of the following spaces, give a presentation of the fundamental group together with a specific loop representing each generator.
\begin{enumerate}[(a)]
\item A closed disk with two interior points removed.
\item The projective plane with two points removed.
\item A connected sum of $n$ tori with one point removed.
\item A connected sum of $n$ tori with two points removed.
\end{enumerate}
\end{prob}

\begin{prob}[ITM 10--13]
Let $n$ be an integer greater than $2$. Construct a polygonal presentation whose geometric realization has a fundamental group that is cyclic of order $n$.
\end{prob}



\end{document}