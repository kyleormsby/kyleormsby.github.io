\documentclass[11pt,twoside]{amsart}
\usepackage{amssymb, amsmath, enumerate, palatino, hyperref,xcolor,booktabs,longtable}
\usepackage[normalem]{ulem}
\usepackage{fullpage}
\usepackage[T1]{fontenc}
\renewcommand{\labelitemi}{\guillemotright}
\usepackage[normalem]{ulem}
\definecolor{dark-red}{rgb}{0.4,0.15,0.15}
%   \definecolor{dark-blue}{rgb}{0.15,0.15,0.4}
%   \definecolor{medium-blue}{rgb}{0,0,0.5}
\setcounter{secnumdepth}{2}
\setcounter{tocdepth}{1}
\hypersetup{
    colorlinks, linkcolor=dark-red,
    citecolor=dark-red, urlcolor=dark-red
}

\title{Math 224 Sections E \& G: Advanced Multivariable Calculus\\ Course Information \& Syllabus}
\author[Math 224:  Adv.~Multivariable Calculus]{Winter 2024}

\begin{document}
\maketitle

%\vspace{-5mm}
\thispagestyle{empty}

\vspace{-.5cm}

\begin{center}
\fbox{
\begin{minipage}{4.75in}
\begin{tabular}{rl}
Place: &Benson Hall 117\\
Section E: &MWF 1:30--2:20\\
Section G: &MWF 2:30--3:20\\
Instructor: &Kyle Ormsby (\href{mailto:ormsbyk@uw.edu}{\nolinkurl{ormsbyk@uw.edu}}), PDL C--524\\
Drop-in Hours: & M 11-12 and Th 1--2 in C-524\\
Textbook: & Stewart, \emph{Calculus: Early Transcendentals}, 8th ed.\\
& (or bookstore version containing Chapters 10--17)\\
Website: &\url{kyleormsby.github.io/224/}\\
Canvas: &\url{canvas.uw.edu/courses/1697416/}
\end{tabular}
\end{minipage}
}
\end{center}

\smallskip

\subsection*{Course description}
In caclulus, we study how things change: vary a parameter and analyze how an output changes. Multivariable calculus studies change when there are multiple inputs and outputs: vary multiple parameters and analyze how multiple outputs change. This grafts geometry onto the edifice of derivatives and integrals, resulting in a fantastic and beautiful theory with deep computational power: everything from electrodynamics to contemporary machine learning depends on it.

Our course will focus on the following topics: double and triple integrtion; integration in polar, cylindrical, and spherical coordinates; vector fields; line integrals; curl and divergence; Green’s Theorem; Stokes’ Theorem; and the Divergence Theorem. The goal is for you to develop computational facility and conceptual understanding of these ideas, and be able to apply them in future mathematical, scientific, and technological work.

\subsection*{Course meetings}
I will deliver interactive lectures in Benson Hall (BNS) 117.  Interactive components and student-teacher feedback are valuable components of the course, and students are encouraged to participate in-person if able and healthy. I will not take or score attendance, but students are responsible for catching up on any material they miss.

\subsection*{Texts}
This course will use Chapters 14--16 of Stewart's \emph{Calculus: Early Transcendentals} as its primary reference. A custom version containing only Chapters 10--17 is available at the bookstore. Each class meeting will be paired with suggested reading. Lectures and readings are intended to complement each other, and you are strongly encouraged to engage with each reading.

\subsection*{Homework}
You are responsible for completing homework assignments via WebAssign (linked on the course Canvas page). Homework will be made available on Monday of the week in which the material is covered in lecture, and will be due the Thursday of the following week. Late work will not be accepted, but you may miss up to 10\% of the points from WebAssign over the course of the quarter without penalty (see the Assessment section below).

I will also provide supplementary problems each week that are deeper and more conceptual. They will not be graded, but you are encouraged to engage with and ask questions about them.

\subsection*{Collaboration}
You are permitted and encouraged to work with your peers on homework problems. But beware: insufficient independent mastery of homework content will make it challenging to succeed on the exams.

\subsection*{Exams}
We will have two 50-minute midterm exams and a 110-minute final exam. The exams are scheduled for the following dates:
\begin{itemize}
\item \textbf{Midterm 1:} 29 January (in class), Sections 15.1--15.9
\item \textbf{Midterm 2:} 21 February (in class), Sections 14.5, 14.6, 16.1--16.5
\item \textbf{Final Exam Section E:} 11 March (BNS 117) 2:30--4:20
\item \textbf{Final Exam Section G:} 12 March (BNS 117), 2:30--4:20
\end{itemize}
\noindent The final exam is cumulative and will cover material through Section 16.9.

You may bring one handwritten two-sided 8.5"$\,{\times}\,$11" note sheet to reference during each exam, as well as a TI-30XIIS calculate. You may \emph{not} use a graphing calculator, phone, smartwatch, \emph{etc.}~on the exams.

If you anticipate an unavoidable conflict with one or more of the exams, please contact me right away.

\subsection*{Assessment}
Students will be assessed based on their demonstrated competence with the learning objectives of the class. Your raw score for the course will be computed by the following formula:
\[
  0.15\times (\text{Homework}) + 0.25\times (\text{Midterm 1}) + 0.25\times (\text{Midterm 2}) + 0.35\times (\text{Final Exam}).
\]
In order to permit students to miss up to 10\% of the homework assignments, your homework score will be computed by
\[
  \max\left\{\frac{\text{Points earned on WebAssign}}{0.9\times (\text{Points possible on WebAssign})},1\right\}.
\]
The course will be graded on a curve in accordance with past instances of Math 224. Your curved score will be greater than or equal to your raw score. If you complete the course with a raw score of at least $0.65$, you will earn a grade of at least 2.0.

\subsection*{Drop-in hours}
I will hold weekly drop-in hours in my office, Padelford (PDL) C-524, at the following times:
\begin{itemize}
\item Mondays 11:00--12:00,
\item Thursdays 1:00--2:00.
\end{itemize}
During these times, I am available to discuss course material and help with homework problems. I am also available via email (\href{mailto:ormsbyk@uw.edu}{\nolinkurl{ormsbyk@uw.edu}}) and by appointment --- if in doubt, please reach out!

\subsection*{The Internet}
You are welcome to use Internet resources to supplement content we cover in this course, with the exception of solutions to homework problems.

\subsection*{Academic Honesty}
Please do not cheat. This includes receiving unauthorized assistance on assignments and exams, as well as representing work which does not belong to you as your own. Any instances of cheating will receive a score of zero on the assignment, and will be reported to the Dean’s Representative for Academic Misconduct. If you have any questions about what constitutes academic misconduct, please reach out to me.

\subsection*{Religious accommodations}
In line with Washington state law, this course accommodates student absences to allow students to take holidays for reasons of faith or conscience or for organized activities conducted under the auspices of a religious denomination, church, or religious organization, so that students’ grades are not adversely impacted by the absences. See \url{https://registrar.washington.edu/staffandfaculty/religious-accommodations-policy/} for further information.

\subsection*{Access and accommodations}
Your experience in this class is important to me. It is the policy and practice of the University of Washington to create inclusive and accessible learning environments consistent with federal and state law. If you have already established accommodations with Disability Resources for Students (DRS), please activate your accommodations via myDRS so we can discuss how they will be implemented in this course.

If you have not yet established services through DRS, but have a temporary health condition or permanent disability that requires accommodations (conditions include but are not limited to: mental health, attention-related, learning, vision, hearing, physical or health impacts), contact DRS directly to set up an Access Plan. DRS facilitates the interactive process that establishes reasonable accommodations. Contact DRS at \url{disability.uw.edu}.

\newpage

\subsection*{Schedule}
Below you will find our anticipated schedule for the course. I reserve the right to modify the schedule as needed, but hope to hew to it quite closely.

\begin{center}
\begin{longtable}{lllll} \toprule
Week &Day &Topic &Reading\\ \midrule
1 &W I.3 &welcome \& warmup \\
&F I.5 &iterated integrals &15.1\\ \midrule
2 &M I.8 &double integrals &15.2\\
&W II.10 &double integrals in polar coordinates &15.3\\
&F II.12 &applications of double integrals &15.4\\ \midrule
3 &M I.15 &NO CLASS (MLK Day) &\\
&W I.17 &surface area &15.5\\
&F I.19 &triple integrals &15.6\\ \midrule
4 &M I.22 &triple integrals in cylindrical coordinates &15.7\\
&W I.24 &triple integrals in spherical coordinates &15.8\\
&F I.26 &change of variables in multiple integrals &15.9\\ \midrule
5 &M I.29 &Midterm Exam 1 (Sections 15.1--15.9) &\\
&W I.31 &chain rule &14.5\\
&F II.2 &directional derivatives and the gradient &14.6\\ \midrule
6 &M II.5 &vector fields &16.1\\
&W II.7 &line integrals &16.2\\
&F II.9 &fundamental theorem for line integrals &16.3\\ \midrule
7 &M II.12 &Green's theorem &16.4\\
&W II.14 &more Green's theorem &16.4\\
&F II.16 &curl and divergence &16.5\\ \midrule
8 &M II.19 &NO CLASS (Pres.~Day) &\\
&W II.21 &Midterm Exam 2 (Sections 14.5, 14.6, 16.1--16.5) &\\
&F II.23 &parametric surfaces and their areas &16.6\\ \midrule
9 &M II.26 &more parametric surfaces &16.6\\
&W II.28 &surface integrals &16.7\\
&F III.1 &Stokes' theorem &16.8\\ \midrule
10 &M III.4 &more Stokes' theorem &16.8\\
&W III.6 &divergence theorem &16.9\\
&F III.8 &divergence theorem / review &16.9\\ \midrule
11 &M III.11 & Section E Final 2:30--4:20 in BNS 117\\
&Tu III.12 & Section G Final 2:30--4:20 in BNS 117\\ \bottomrule
\end{longtable}
\end{center}

\bigskip \bigskip
\bigskip \bigskip

\begin{center}
Remember: \emph{Math is hard, but we're going to get through this together!}
\end{center}



\end{document}