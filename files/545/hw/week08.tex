\documentclass[11pt,twoside]{amsart}
\usepackage{amssymb, amsmath, enumerate, palatino, hyperref,tikz,tikz-cd}
\usepackage[normalem]{ulem}
\usepackage{fullpage}
\usepackage[T1]{fontenc}
\renewcommand{\labelitemi}{\guillemotright}
\usepackage{mathrsfs}
\usepackage{phaistos}


\theoremstyle{plain}
\newtheorem{prop}{Proposition}%[section]
\newtheorem{lemma}[prop]{Lemma}
\newtheorem{thm}[prop]{Theorem}
\newtheorem{obs}[prop]{Observation}
\newtheorem{app}[prop]{Application}
\newtheorem*{MainThm}{Main Theorem}
\newtheorem{cor}[prop]{Corollary}
\newtheorem{conj}[prop]{Conjecture}
\theoremstyle{remark}
\newtheorem{rmk}[prop]{Remark}
\newtheorem{prob}{Problem}
\newtheorem{bonus}[prop]{Bonus Problem}
\newtheorem{exc}{Exercise}
\theoremstyle{definition}
\newtheorem{ex}[prop]{Example}
\theoremstyle{definition}
\newtheorem{defn}[prop]{Definition}

\newcommand{\RR}{\mathbb{R}}
\newcommand{\ZZ}{\mathbb{Z}}
\newcommand{\CC}{\mathbb{C}}
\newcommand{\NN}{\mathbb{N}}
\newcommand{\QQ}{\mathbb{Q}}
\newcommand{\PP}{\mathbb{P}}
\newcommand{\TT}{\mathbb{T}}
\newcommand{\kk}{\mathsf{k}}
\newcommand{\FF}{\mathbb{F}}
\newcommand{\cS}{\mathcal{S}}
\newcommand{\cT}{\mathcal{T}}
\newcommand{\ssC}{\mathsf{C}}
\newcommand{\sU}{\mathscr{U}}
\newcommand{\ol}{\overline}

\newcommand{\id}{\operatorname{id}}
\newcommand{\Int}{\operatorname{Int}}
\newcommand{\cs}{\mathbin{\#}}
\newcommand{\Ab}{\mathsf{Ab}}
\newcommand{\Top}{\mathsf{Top}}
\newcommand{\Grp}{\mathsf{Grp}}
\newcommand{\Aut}{\operatorname{Aut}}


\title{Math 545: Manifolds\\ Homework due Friday Week 8}
%\author{Your Name}

\begin{document}
\maketitle

\noindent Problems taken from \emph{Introduction to Smooth Manifolds} are marked ISM $x$--$y$. Please review the syllabus for expectations and policies regarding homework.

\begin{prob}[ISM 4--8]
This problem shows that the converse of ISM Theorem 4.29 is false. Let $\pi\colon \RR^2\to \RR$ be defined by $\pi(x,y)=xy$. Show that $\pi$ is surjective and smooth, and for each smooth manifold $P$, a map $F\colon \RR\to P$ is smooth if and only if $F\circ \pi$ is smooth; but $\pi$ is not a smooth submersion.
\end{prob}

\begin{prob}[ISM 4--12]
Using the covering map $\varepsilon^2\colon \RR^2\to \TT^2$, show that the immersion $X\colon \RR^2\to \RR^3$ defined in ISM Example 4.2(d) descends to a smooth embedding of $\TT^2$ into $\RR^3$. Specifically, show that $X$ passes to the quotient to define a smooth map $\tilde X\colon \TT^2\to \RR^3$, and then show that $\tilde X$ is a smooth embedding whose image is the given surface of revolution.
\end{prob}

\begin{prob}[ISM 4--13]
Define a map $F\colon S^2\to \RR^4$ by $F(x,y,z) = (x^2-y^2,xy,xz,yz)$. Using the standard smooth covering map $q\colon S^2\to \RR\PP^2$, show that $F$ descends to a smooth embedding of $\RR\PP^2$ into $\RR^4$.
\end{prob}









\end{document}