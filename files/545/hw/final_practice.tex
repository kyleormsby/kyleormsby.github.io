\documentclass[11pt,twoside]{amsart}
\usepackage{amssymb, amsmath, enumerate, palatino, hyperref,tikz,tikz-cd}
\usepackage[normalem]{ulem}
\usepackage{fullpage}
\usepackage[T1]{fontenc}
\renewcommand{\labelitemi}{\guillemotright}
\usepackage{mathrsfs}
\usepackage{phaistos}


\theoremstyle{plain}
\newtheorem{prop}{Proposition}%[section]
\newtheorem{lemma}[prop]{Lemma}
\newtheorem{thm}[prop]{Theorem}
\newtheorem{obs}[prop]{Observation}
\newtheorem{app}[prop]{Application}
\newtheorem*{MainThm}{Main Theorem}
\newtheorem{cor}[prop]{Corollary}
\newtheorem{conj}[prop]{Conjecture}
\theoremstyle{remark}
\newtheorem{rmk}[prop]{Remark}
\newtheorem{prob}{Problem}
\newtheorem{bonus}[prop]{Bonus Problem}
\newtheorem{exc}{Exercise}
\theoremstyle{definition}
\newtheorem{ex}[prop]{Example}
\theoremstyle{definition}
\newtheorem{defn}[prop]{Definition}

\newcommand{\RR}{\mathbb{R}}
\newcommand{\ZZ}{\mathbb{Z}}
\newcommand{\CC}{\mathbb{C}}
\newcommand{\NN}{\mathbb{N}}
\newcommand{\QQ}{\mathbb{Q}}
\newcommand{\PP}{\mathbb{P}}
\newcommand{\kk}{\mathsf{k}}
\newcommand{\FF}{\mathbb{F}}
\newcommand{\cS}{\mathcal{S}}
\newcommand{\cT}{\mathcal{T}}
\newcommand{\ssC}{\mathsf{C}}
\newcommand{\sU}{\mathscr{U}}
\newcommand{\ol}{\overline}

\newcommand{\id}{\operatorname{id}}
\newcommand{\Int}{\operatorname{Int}}
\newcommand{\cs}{\mathbin{\#}}
\newcommand{\Ab}{\mathsf{Ab}}
\newcommand{\Top}{\mathsf{Top}}
\newcommand{\Grp}{\mathsf{Grp}}


\title{Math 545: Manifolds\\ Final Exam Practice Problems}
%\author{Your Name}

\begin{document}
\maketitle

\noindent Use these problems to prepare for your final oral exam. You are welcome to collaborate on them. I will ask you about at least one of these problems during your oral exam.

\begin{prob}
Let $X = S^1\vee S^1$ and let $n>2$. Give an explicit description of a covering map $p\colon Y\to X$ such that the image of $p_*\colon \pi_1(Y,y)\to \pi_1(X,x)$ is a free group on $n$ generators and is a normal subgroup of index $n-1$. (You may describe $Y$ by drawing a picture, but you must justify why $p$ has the indicated properties.)
\end{prob}

\begin{prob}
Recall (from the Math 544 final exam) that the \emph{mapping torus} for a continuous map $f\colon X\to X$ is the quotient space
\[
  M_f := (X\times I)/{\sim}
\]
where $\sim$ is the equivalence relation generated by $(x,1)\sim (f(x),0)$.
\begin{enumerate}[(a)]
\item Use the Mayer--Vietoris theorem to show that there is a long exact sequence of singular homology groups
\[
  \cdots \to H_n(X)\xrightarrow{1-f_*} H_n(X)\to H_n(M_f)\to H_{n-1}(X)\to \cdots.
\]
(As part of your argument, you should identify the unlabeled arrows in the long exact sequence.)
\item Compute the homology $M_f$ where $f\colon S^2\to S^2$ is the antipodal map.
\end{enumerate}
\end{prob}

\begin{prob}
Let $M$ be a smooth manifold with chart $(\varphi,U)$ around a point $p\in M$. Let $\mathcal C_p$ be the set of all smooth maps $\gamma\colon (-\varepsilon,\varepsilon)\to U$ such that $\gamma(0)=p$. Define an equivalence relation $\sim$ on $\mathcal C_p$ by declaring $\delta\sim \gamma$ if and only if
\[
  (\varphi\circ \delta)'(0) = (\varphi\circ \gamma)'(0).
\]
Give $\mathcal C_p/{\sim}$ the structure of a real vector space and prove that $T_pM$ and $\mathcal C_p/{\sim}$ are linearly isomorphic.
\end{prob}

\begin{prob}
Let $X$ denote the set of pairs of orthogonal lines through the origin in $\RR^{n+1}$, viewed as a subspace of the product $\RR\PP^n\times \RR\PP^n$. Prove that $X$ is an embedded submanifold of $\RR\PP^n\times \RR\PP^n$ and determine the dimension of $X$. 
\end{prob}

\begin{prob}
Let $f_1\colon M_1\to N$ and $f_2\colon M_2\to N$ be smooth maps. We say $f_1$ and $f_2$ \emph{intersect transversely} when the product map $f_1\times f_2\colon M_1\times M_2\to N\times N$ intersects the diagonal $\Delta_N = \{(q,q)\mid q\in N\}\subseteq N\times N$ transversely. Prove that if $f_1$ and $f_2$ intersect transversely, then the fiber product $F = (f_1\times f_2)^{-1}\Delta_N$ is a submanifold of $M_1\times M_2$, and for any $(p_1,p_2)\in F$,
\[
  T_{(p_1,p_2)}F = \{(v_1,v_2)\mid v_i\in T_{p_i}M_i,~(df_1)_{p_1}(v_1) = (df_2)_{p_2}(v_2)\}.
\]
\end{prob}




\end{document}