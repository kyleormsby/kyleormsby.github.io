\documentclass[11pt,twoside]{amsart}
\usepackage{amssymb, amsmath, enumerate, palatino, hyperref,tikz,tikz-cd}
\usepackage[normalem]{ulem}
\usepackage{fullpage}
\usepackage[T1]{fontenc}
\renewcommand{\labelitemi}{\guillemotright}
\usepackage{mathrsfs}
\usepackage{phaistos}


\theoremstyle{plain}
\newtheorem{prop}{Proposition}%[section]
\newtheorem{lemma}[prop]{Lemma}
\newtheorem{thm}[prop]{Theorem}
\newtheorem{obs}[prop]{Observation}
\newtheorem{app}[prop]{Application}
\newtheorem*{MainThm}{Main Theorem}
\newtheorem{cor}[prop]{Corollary}
\newtheorem{conj}[prop]{Conjecture}
\theoremstyle{remark}
\newtheorem{rmk}[prop]{Remark}
\newtheorem{prob}{Problem}
\newtheorem{bonus}[prop]{Bonus Problem}
\newtheorem{exc}{Exercise}
\theoremstyle{definition}
\newtheorem{ex}[prop]{Example}
\theoremstyle{definition}
\newtheorem{defn}[prop]{Definition}

\newcommand{\RR}{\mathbb{R}}
\newcommand{\ZZ}{\mathbb{Z}}
\newcommand{\CC}{\mathbb{C}}
\newcommand{\NN}{\mathbb{N}}
\newcommand{\QQ}{\mathbb{Q}}
\newcommand{\PP}{\mathbb{P}}
\newcommand{\kk}{\mathsf{k}}
\newcommand{\FF}{\mathbb{F}}
\newcommand{\cS}{\mathcal{S}}
\newcommand{\cT}{\mathcal{T}}
\newcommand{\ssC}{\mathsf{C}}
\newcommand{\sU}{\mathscr{U}}
\newcommand{\ol}{\overline}

\newcommand{\id}{\operatorname{id}}
\newcommand{\Int}{\operatorname{Int}}
\newcommand{\cs}{\mathbin{\#}}
\newcommand{\Ab}{\mathsf{Ab}}
\newcommand{\Top}{\mathsf{Top}}
\newcommand{\Grp}{\mathsf{Grp}}
\newcommand{\Aut}{\operatorname{Aut}}


\title{Math 545: Manifolds\\ Homework due Friday Week 5}
%\author{Your Name}

\begin{document}
\maketitle

\noindent Problems taken from \emph{Introduction to Topological Manifolds} are marked ITM $x$--$y$; problems taken from \emph{Introduction to Smooth Manifolds} are marked ISM $x$--$y$. Please review the syllabus for expectations and policies regarding homework.

\begin{prob}[ITM 13--7, \textsc{Brouwer fixed point theorem}]
For each integer $n\ge 0$, prove that every continuous map $f\colon \overline{\mathbb B}^n\to \overline{\mathbb{B}}^n$ has a fixed point. (Use homology to emulate the fundamental group-based proof for $n=2$.)
\end{prob}

\begin{prob}[ITM 13--8]
Show that if $n$ is even, then $C_2 = \{\pm 1\}$ is the only nontrivial group that can act freely and continuously on $S^n$. (\emph{Hint}: Show that if $G$ acts continuously on $S^n$, then degree induces a homomorphism $G\to C_2$. \emph{Bonus}: What happens when $n$ is odd?)
\end{prob}

\begin{prob}[ITM 13--9]
Use the cell structure of $\RR\PP^3$ to compute its homology. (\emph{Bonus}: Compute the homology of $\RR\PP^n$ for all $n$.)
\end{prob}

\begin{prob}[ISM 1--8]
An \emph{angle function} on a subset $U\subseteq S^1\subseteq \CC$ is a continuous function $\theta\colon U\to \RR$ such that $\exp(i\theta(z))=z$ for all $z\in U$. Show that there exists an angle function $\theta$ on an open subset $U\subseteq S^1$ if and only if $U\ne S^1$. For any such angle function, show that $(U,\theta)$ is a smooth coordinate chart for $S^1$ with a standard smooth structure.
\end{prob}

\begin{prob}[ISM 1--9]
For $n\ge 0$, \emph{complex projective space}, $\CC\PP^n$, is the set of all $1$-dimensional $\CC$-linear subspaces of $\CC^{n+1}$ with the quotient topology induced by the natural projection $\pi\colon \CC^{n+1}\smallsetminus \{0\}\to \CC\PP^n$. Show that $\CC\PP^n$ is a compact $2n$-dimensional topological manifold, and show how to give it a smooth structure analogous to the smooth structure on $\RR\PP^n$.
\end{prob}





\end{document}