\documentclass[11pt,twoside]{amsart}
\usepackage{amssymb, amsmath, enumerate, palatino, hyperref,tikz,tikz-cd}
\usepackage[normalem]{ulem}
\usepackage{fullpage}
\usepackage[T1]{fontenc}
\renewcommand{\labelitemi}{\guillemotright}
\usepackage{mathrsfs}
\usepackage{phaistos}


\theoremstyle{plain}
\newtheorem{prop}{Proposition}%[section]
\newtheorem{lemma}[prop]{Lemma}
\newtheorem{thm}[prop]{Theorem}
\newtheorem{obs}[prop]{Observation}
\newtheorem{app}[prop]{Application}
\newtheorem*{MainThm}{Main Theorem}
\newtheorem{cor}[prop]{Corollary}
\newtheorem{conj}[prop]{Conjecture}
\theoremstyle{remark}
\newtheorem{rmk}[prop]{Remark}
\newtheorem{prob}{Problem}
\newtheorem{bonus}[prop]{Bonus Problem}
\newtheorem{exc}{Exercise}
\theoremstyle{definition}
\newtheorem{ex}[prop]{Example}
\theoremstyle{definition}
\newtheorem{defn}[prop]{Definition}

\newcommand{\RR}{\mathbb{R}}
\newcommand{\ZZ}{\mathbb{Z}}
\newcommand{\CC}{\mathbb{C}}
\newcommand{\NN}{\mathbb{N}}
\newcommand{\QQ}{\mathbb{Q}}
\newcommand{\PP}{\mathbb{P}}
\newcommand{\kk}{\mathsf{k}}
\newcommand{\FF}{\mathbb{F}}
\newcommand{\cS}{\mathcal{S}}
\newcommand{\cT}{\mathcal{T}}
\newcommand{\ssC}{\mathsf{C}}
\newcommand{\sU}{\mathscr{U}}
\newcommand{\ol}{\overline}

\newcommand{\id}{\operatorname{id}}
\newcommand{\Int}{\operatorname{Int}}
\newcommand{\cs}{\mathbin{\#}}
\newcommand{\Ab}{\mathsf{Ab}}
\newcommand{\Top}{\mathsf{Top}}
\newcommand{\Grp}{\mathsf{Grp}}
\newcommand{\Aut}{\operatorname{Aut}}


\title{Math 545: Manifolds\\ Homework due Friday Week 4}
%\author{Your Name}

\begin{document}
\maketitle

\noindent Problems taken from \emph{Introduction to Topological Manifolds} are marked ITM $x$--$y$. Please review the syllabus for expectations and policies regarding homework.

\begin{prob}[12--4]
Let $\mathscr E \cong S^1\vee S^1$ be the figure-eight space consisting of unit circles centered at $\pm i$ in $\CC$, and let $X$ be the union of the real axis of $\CC$ with infinitely many unit circles centered at $2\pi k+i$, $k\in \ZZ$. Let $q\colon X\to \mathscr E$ be the map sending each circle in $X$ onto the upper circle in $\mathscr E$ by translation by a real number, and sending the real axis onto the lower circle by $x\mapsto ie^{ix}-i$. You may assume that $q$ is a covering map.
\begin{enumerate}[(a)]
\item Identify the subgroup $q_*\pi_1(X,0)$ of $\pi_1(\mathscr E,0)$ in terms of the generators of $\pi_1(\mathscr E,0)$.
\item Determine the automorphism group $\Aut_q(X)$.
\item Determine whether $q$ is a normal covering.
\end{enumerate}
\end{prob}

\begin{prob}[12--6]
Let
\[
  E = \{(x,y)\in \RR^3\times \RR^3\mid x\ne y\}
\]
considered as a subspace of $\RR^3\times \RR^3=\RR^6$. Define an equivalence relation $\sim$ on $E$ by $(x,y)\sim (y,x)$ for all $(x,y)\in E$. Compute the fundamental group of $E/{\sim}$.  (\emph{Note}: The space $E/{\sim}$ is the \emph{unordered configuration space of two points in $\RR^3$}.)
\end{prob}

\begin{prob}[12--9]
Find a covering space action of a group $\Gamma$ on the plane such that $\RR^2/\Gamma$ is homeomorphic to the Klein bottle.
\end{prob}

\begin{prob}
Let $X$ be a finite connected simple graph, \emph{i.e.}, a connected $1$-dimensional CW complex with $p$ $0$-cells and $q$ $1$-cells with ``no loops'' and ``no parallel edges''. Compute $H_1(X)$ in terms of $p$ and $q$. (You may use the fact that such a graph has a \emph{spanning tree}: a subgraph which is a tree containing all the vertices of the original graph.)
\end{prob}

\begin{prob}[13--3, \textsc{Invariance of Dimension}] Prove that if $m\ne n$, then a nonempty topological space cannot be both an $m$-manifold and an $n$-manifold. (You may use Corollary 13.24 identifying the homology of $\RR^n\smallsetminus \{0\}$; you may want to utilize [but still need to prove] the results from Problem 13--2.)
\end{prob}



\end{document}