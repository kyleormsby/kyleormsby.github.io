\documentclass[11pt,twoside]{amsart}
\usepackage{amssymb, amsmath, enumerate, palatino, hyperref}
\usepackage[normalem]{ulem}
\usepackage{fullpage}
\usepackage[T1]{fontenc}
\renewcommand{\labelitemi}{$\cdot$}
\usepackage{mathrsfs}
\usepackage{phaistos}
\usepackage{tikz}

\definecolor{dark-red}{rgb}{0.4,0.15,0.15}
\hypersetup{
    colorlinks, linkcolor=dark-red,
    citecolor=dark-red, urlcolor=dark-red
}


\theoremstyle{plain}
\newtheorem{prop}{Proposition}%[section]
\newtheorem{lemma}[prop]{Lemma}
\newtheorem{thm}[prop]{Theorem}
\newtheorem{obs}[prop]{Observation}
\newtheorem{app}[prop]{Application}
\newtheorem*{MainThm}{Main Theorem}
\newtheorem{cor}[prop]{Corollary}
\newtheorem{conj}[prop]{Conjecture}
\theoremstyle{remark}
\newtheorem{rmk}[prop]{Remark}
\newtheorem{prob}{Problem}
\newtheorem{bonus}[prop]{Bonus Problem}
\newtheorem{exc}{Exercise}
\theoremstyle{definition}
\newtheorem{ex}[prop]{Example}
\theoremstyle{definition}
\newtheorem{defn}[prop]{Definition}

\newcommand{\RR}{\mathbb{R}}
\newcommand{\ZZ}{\mathbb{Z}}
\newcommand{\CC}{\mathbb{C}}
\newcommand{\NN}{\mathbb{N}}
\newcommand{\QQ}{\mathbb{Q}}
\newcommand{\PP}{\mathbb{P}}
\newcommand{\kk}{\mathsf{k}}
\newcommand{\FF}{\mathbb{F}}
\newcommand{\cS}{\mathcal{S}}
\newcommand{\cT}{\mathcal{T}}
\newcommand{\ssC}{\mathsf{C}}
\newcommand{\sU}{\mathscr{U}}
\newcommand{\ol}{\overline}

\newcommand{\id}{\operatorname{id}}
\newcommand{\Int}{\operatorname{Int}}
\title{Math 545: Topology\\ Wednesday Week 1}
%\author{Your Name}

\begin{document}
\maketitle

Given a group presentation
\[
  G = \langle x_1,x_2,\ldots,x_m\mid r_1,r_2,\ldots,r_n\rangle
\]
we know how to build a 2-dimensional cell complex $X_G$ (called the \emph{presentation complex}) with
\begin{itemize}
\item one 0-cell,
\item one 1-cell (necessarily a loop) for each generator $x_i$,
\item one 2-cell for each relation $r_j$, glued onto the 1-skeleton according to its word.
\end{itemize}
For instance, if $G = \langle x,y\mid xyx^{-1}y^{-1}\rangle$ is the free Abelian group on two generators, then $(X_G)_1 = S^1\vee S^1$ and $X_G$ has a unique 2-cell with boundary traversing the figure-eight once in one orientation then once in the opposite orientation.

Recall that the presentation complex has a nice property:
\[
  \pi_1 X_G \cong G.
\]
Indeed, we've set $X_G$ up so that $\pi_1X_G$ is generated by the loops in the 1-skeleton, $x_1,\ldots,x_m$, and the 2-cells witness the relations $r_1,\ldots,r_n$.

We will now construct a new 2-dimensional cell complex $\tilde X_G$ called the \emph{Cayley complex} of $G$. It comes equipped with a map $\tilde X_G\to X_G$ whose properties will be of significant interest.

We begin by connecting some dots to form the \emph{Cayley graph} $\Gamma_G$ of $G$; this will ultimately be the 1-skeleton of the Cayley complex, $(\tilde X_G)_1 = \Gamma_G$.\footnote{\emph{Warning}: This construction depends on the chosen generators of $G$, so should perhaps be written $\Gamma_{G,S}$ for $S$ a set of generators of $G$. If $G$ is specified by a presentation, then we will always take $S$ to be the given generators.} The vertices of $\Gamma_G$ are the elements of $G$, and there is a labeled directed edge $g\xrightarrow{x_i}h$ if and only if $h = gx_i$ where $x_i\in S$, the set of generators. In other words, the edges are of the form $g\xrightarrow{x_i}gx_i$.

\begin{prob}
Draw the Cayley graphs for the cyclic groups of order $2$, $3$, and $\infty$,\footnote{Of course, these are isomorphic to $\ZZ/2\ZZ$, $\ZZ/3\ZZ$, and $\ZZ$, respectively, but this notation will help us remember the multiplicative notation.}
\[
  C_2 = \langle x\mid x^2\rangle,\qquad C_3 = \langle y\mid y^3\rangle,\qquad C_\infty = \langle z\rangle.
\]
\end{prob}

To construct $\tilde X_G$, we need to glue some 2-cells to $\Gamma_G$. Here is the rule:
\begin{itemize}
\item For each $g\in G$ and relation $r_j$, attach a 2-cell based at $g$ via the loop given by the word $r_j$.
\end{itemize}

\begin{prob}
Construct the Cayley complexes $\tilde X_{C_2}$, $\tilde X_{C_3}$, and $\tilde X_{C_{\infty}}$.
\end{prob}

\begin{prob}
Observe that $\tilde X_G$ is simply connected (\emph{i.e.} path connected with trivial fundamental group).
\end{prob}

The Cayley complex $\tilde X_G$ admits a natural left action by $G$. For $h\in G$, we make the following definitions:
\begin{itemize}
\item on 0-cells $g\in G$, $h\cdot g = hg$,
\item on 1-cells, $h\cdot(g\xrightarrow{x_i}gx_i) = hg\xrightarrow{x_i}hgx_i$,
\item on 2-cells, if $g\in G$ and $r_j = xyz\cdots$, then the 2-cell attached along the loop \[g\xrightarrow{x}gx\xrightarrow{y}gxy\xrightarrow{z}gxyz\to\cdots\] is taken homeomorphically to the 2-cell attached along \[hg\xrightarrow{x}hgx\xrightarrow{y}hgxy\xrightarrow{z}hgxyz\to\cdots.\]
\end{itemize}

\begin{prob}
Convince yourself that the above rules define a continuous left action of $G$ on $\tilde X_G$.
\end{prob}

\begin{prob}
Argue that the orbit space $\tilde X_G/G$ is homeomorphic to $X_G$, \emph{i.e.}, the quotient of the Cayley complex by its $G$-action is the presentation complex.
\end{prob}

We now have a natural quotient map
\[
  \tilde X_G\longrightarrow X_G.
\]

\begin{prob}
Illustrate the quotient map $\tilde X_G\to X_G$ for $G = C_2,C_3,C_\infty$.
\end{prob}

\begin{prob}
What general properties does $\tilde X_G\to X_G$ have? How is it similar to and different from $\RR\to S^1$, $t\mapsto \exp(2\pi it)$?
\end{prob}

\begin{prob}
Illustrate the Cayley and presentation complexes for $C_2*C_2 = \langle a,b\mid a^2,b^2\rangle$ and the free group on two generators.
\end{prob}

\end{document}