\documentclass[11pt,twoside]{amsart}
\usepackage{amssymb, amsmath, enumerate, libertine, hyperref,xcolor,booktabs,longtable,microtype}
\usepackage[normalem]{ulem}
\usepackage{fullpage}
\usepackage[T1]{fontenc}
\renewcommand{\labelitemi}{$\cdot$}
\usepackage[normalem]{ulem}
\definecolor{dark-red}{rgb}{0.4,0.15,0.15}
%   \definecolor{dark-blue}{rgb}{0.15,0.15,0.4}
%   \definecolor{medium-blue}{rgb}{0,0,0.5}
\setcounter{secnumdepth}{2}
\setcounter{tocdepth}{1}
\hypersetup{
    colorlinks, linkcolor=dark-red,
    citecolor=dark-red, urlcolor=dark-red
}

\title{Math 111 F06: Calculus\\ Course Information \& Syllabus}
\author[Math 111 F06:  Calculus]{Fall 2024}

\begin{document}
\maketitle

%\vspace{-5mm}
\thispagestyle{empty}

\vspace{-.5cm}

\begin{center}
\fbox{
\begin{minipage}{4.5in}
\begin{tabular}{rl}
Place: &Physics 240A\\
Time: &MWF, 13:10-14:00\\
Instructor: &Kyle Ormsby (\href{mailto:ormsbyk@reed.edu}{ormsbyk@reed.edu})\\
Drop-in Hours: &TBD in Lib 306\\
Course Assistant: &Amelie el Mahmoud (\href{mailto:aelmahmoud}{aelmahmoud@reed.edu})\\
Problem Sessions: &TBD\\
Textbook: &openstax \href{https://openstax.org/details/books/calculus-volume-1}{\emph{Calculus}} (Vol.~1) \\
Website: &\href{https://kyleormsby.github.io/111/}{kyleormsby.github.io/111/}
\end{tabular}
\end{minipage}
}
\end{center}

\smallskip

\subsection*{Course description}
Calculus --- the mathematical study of change --- is one of humanity's most fruitful inventions and is fundamental to the study of physical sciences, computer algorithms, and technology. This course is an introduction to differential and integral single variable calculus for liberal arts students with a focus on concepts and computations (but not proofs). We will explore fundamental mathematics and wide-ranging applications.

\subsection*{Learning outcomes}
\begin{itemize}
\item Gain fluency with functions of a single real variable and be able to determine information about functions via limits, differentiation, and integration.
\item Interpret and apply the concepts of function, limit, derivative, and integral.
\item Conceptually understand and state the formal definitions of derivatives (as limits of difference quotients) and integrals (as Riemann sums).
\item State, understand, and apply both versions of the Fundamental Theorem of Calculus.
\item Compute derivatives via the power, product, quotient, and chain rules.
\item Compute indefinite and definite integrals via substitution and integration by parts.
\item Understand, manipulate, and compute derivatives and integrals of polynomial, rational, exponential, logarithmic, and trigonometric functions.
\item Solve problems involving optimization, implicit differentiation, and related rates.
\item \textbf{Communicate mathematical ideas verbally and in writing}.
\end{itemize}

\subsection*{Distribution requirements}
This course can be used towards your Group III, ``Natural, Mathematical, and Psychological Science,'' requirement.  It accomplishes the following goals for the group:
\begin{itemize}
\item Use and evaluate quantitative data or modeling, or use logical/mathematical reasoning to evaluate, test, or prove statements.
\item Given a problem or question, formulate a hypothesis or conjecture, and design an experiment, collect data or use mathematical reasoning to test or validate it.
\end{itemize}
This course \textbf{does not} satisfy the ``primary data collection and analysis'' requirement.

\subsection*{Text}
We will use openstax \emph{Calculus} (Vol.~1) as a reference for the course.  The book is freely accessible in HTML and PDF formats online at \href{https://openstax.org/details/books/calculus-volume-1}{openstax.org/details/books/calculus-volume-1}; you can also order an inexpensive physical copy from that site.  Each lecture will be paired with a \textbf{required reading} that you need to do before class.  Lectures and readings will be complementary, and both will be necessary to fully appreciate and learn the course content.

\subsection*{Participation}
I will deliver interactive in-person lectures with frequent breaks for problem-solving. Your required reading will lay the groundwork for productively participating in these meetings. This course does not have a formal attendance policy, but I will use your engagement to assess the participation portion of your grade. If you miss a class, it is your responsibility to catch up with the material via the course notes, textbook, and discussions with peers.

All students are encouraged --- but not required --- to engage in drop-in hours and problem sessions (see below).

\subsection*{ALEKS Diagnostic Tool}
This year, several intro courses in Mathematical and Natural Sciences are working together to try a new mathematical diagnostic and learning tool, ALEKS ``Assessment and LEarning in Knowledge Spaces''). It is an adaptive test, and expected to take no more than 80 minutes. For my section of Math 111, \textbf{you are required to take ALEKS by Monday 9 August at noon}; this is your first homework assignment. Although this is a required assignment, it is not a placement exam, nor will your overall score impact your grade (in fact, I will not even receive your score). There is also no ``target'' score; instead, your results will direct you to specific potential areas of mathematical growth, and online learning modules to support that growth. You are further encouraged to use this information in seeking more specific support from tutors and faculty. Some further instructions:
\begin{itemize}
\item Use your Reed email address when you set up your ALEKS account (otherwise your assessment will not be counted).
\item You are encouraged to use pen and paper to work through problems.
\item In order for this tool to be maximally helpful to you, it is very important that you do not use any additional outside resources in completing the test.
\item Complete ALEKS in one sitting in a timely manner, free of other distractions.
\item If you are being asked to complete this assignment in more than one class this year, you only need to take ALEKS once.
\item You can use ALEKS's Prep and Learning Modules for a full year to help improve your preparation.
\end{itemize}

\subsection*{Quizzes}
Each Monday course meeting will feature a brief quiz focused on computational proficiency. You will only receive credit for completion of these quizzes, and you'll self-assess your work immediately after each quiz. I hope the quizzes will provide useful practice and a strong signal for whether you are keeping up with computational aspects of the material that will appear on exams. Make-up quizzes will not be offered, but each student can miss up to two quizzes without penalty.


\subsection*{Homework}
Homework is due via Gradescope\footnote{Gradescope is an online homework submission and evaluation platform. You will receive a link to register via email and on Zulip.} every Wednesday by 22:00. Homework due Wednesday of week $N$ covers topics through Friday of week $N-1$, and you are strongly encouraged to start homework early so that you can take advantage of office hours and problem sessions.  Excellent solutions take many forms, but they all have the following characteristics:

\begin{itemize}
\item they are written as explanations for other students in the course; in particular, they fully explain all of their reasoning and do not assume that the reader will fill in details;
\item when graphical reasoning is called for, they include large, carefully drawn and labeled diagrams;
\item they are neatly written or typeset;\footnote{Interested students are 
encouraged to prepare solutions in the \LaTeX~document preparation 
system.  A guide to \LaTeX~resources is available on the course 
website.  Nearly all of the \texttt{.pdf} files on the course website are produced by \LaTeX; you can find their associated source files by changing the \texttt{.pdf} suffix to \texttt{.tex} in the file's URL.} and
\item they use complete sentences, even when formulas or symbols are involved.
\end{itemize}

\textbf{Given the exigencies of contemporary existence, I will be flexible with deadlines as long as you communicate with me about extensions.} If health, family, or emergent national crises might impede the timely completion of your homework, please contact me as early as possible.

\subsection*{Collaboration}
You are permitted and encouraged to work with your peers on homework problems.  You must cite those with whom you worked, and you must write up solutions independently.  \textbf{Duplicated solutions will not be accepted and constitute a violation of the Honor Principle.}

Expect an announcement in the first week of class regarding a system by which you can coordinate study sessions with peers.


\subsection*{Feedback}
You will receive timely feedback on your homework via Gradescope, either from me or the course grader (a mathematics undergraduate).  Each homework problem can earn up to five points for mathematical content, and two points for the quality of writing.  If your answer is incorrect, this will be reflected in the score, and there will also be a comment indicating where things went wrong with your solution.  You are strongly encouraged to engage with this comment, understand your error, and try to come up with a correct solution.  You are very welcome to post questions about old homework problems to the Zulip workspace (see below) and talk about them with me in office hours (see the Help section).



\subsection*{Exams}
We will have three 50-minute midterm exams and a two-hour final exam. All exams will be in-class, closed book, and no calculator, but you will be allowed a two-sided cue sheet.

\begin{itemize}
\item Exam 1: Monday 23 September
\item Exam 2: Friday 18 October
\item Exam 3: Friday 22 November
\item Final Exam: as scheduled by the registrar
\end{itemize}



\subsection*{Joint expectations}
As members of a communal learning environment, we should all expect consideration, fairness, patience, and curiosity from each other.  Our aim is to all learn more through cooperation and genuine listening and sharing, not to compete or show off.  I expect diligence and academic and intellectual honesty from each of you.  You should expect that I will do my best to focus the course on interesting, pertinent topics, and that I will provide feedback and guidance that will help you excel as a student.

\subsection*{Help}
There are a number of resources you can access for help with this course's content.  Everyone is welcome and encouraged to attend my drop-in hours (times TBD).  I am also happy to arrange drop-in hours by appointment.  Drop-in hours are an opportunity to clarify difficult material and also delve deeper into topics that interest you.  Please reach out to me if there are barriers preventing you from effectively utilizing this opportunity.

Our course assistant, Amelie el Mahmoud, will run a two-hour problem session each week at a time and location TBD.  We will share problem sessions with Marcus Robinson's sections of Math 111, F02 and F03.  The problem sessions will provide a structured, facilitated environment in which you can collaborate on homework.  All students are encouraged to attend.

The math department also hosts drop-in tutoring on Sunday, Monday, Tuesday, Wednesday, and Thursday 19:00--21:00 in Lib 204. Upperclass tutors will be available to clarify concepts and help you with homework problems.

Finally, every Reed student is entitled to one hour of free individual tutoring per week.  Use the tutoring app in IRIS to arrange to work with a student tutor.

\subsection*{Zulip}
Our section of Math 111 has a Zulip workspace.  Use the Zulip wprkspace to ask questions (of me or the class), collaborate on problems, and share resources. The Zulip workspace is an extension of our classroom and the above joint expectations extend to this setting.

You will receive an email invitation to join our Zulip workspace during the first week of classes. Please use channels and threads to keep conversations organized.

\subsection*{Technology}
The use of electronic devices (cell phones, computers, tablets, 
calculators, \emph{etc}.) is prohibited in the classroom without prior 
authorization from the instructor.  That said, legitimate uses 
of technology (\emph{e.g.}, note-taking) will be accommodated --- 
just talk to me first.

Students are not permitted to consult or utilize chatbot / large language model / ``AI'' utilities in their work.

\subsection*{The Internet}
You are welcome to use Internet resources to supplement content we cover in this course, with the exception of solutions to homework problems.  \textbf{Copying solutions from the Internet is an Honor Principle violation and will result in an academic misconduct report. The same goes for ``AI''-generated ``solutions''.}

\subsection*{Academic accommodations}
If you have a documented disability requiring academic accommodation, please have  Disability \& Accessibility Resources (DAR)  provide a letter during the first week of classes.  I will then contact you to schedule a meeting during which we can discuss your accommodations.  If you believe you have an undocumented disability and that accommodations would ensure equal access to your Reed education, I would be happy to help you contact DAR.

\subsection*{Grades}
Your grade will reflect a composite assessment of the work you produce for the class, weighted in the following fashion:  30\% homework, 20\% final exam, 15\% exam 3, 15\% exam 2, 10\% exam 1, 5\% quizzes, 5\% class participation.

\bigskip \bigskip

\begin{center}
Remember: \emph{Math is hard, but we'll get through this together!}
\end{center}


\end{document}