\documentclass[11pt,twoside]{amsart}
\usepackage{amssymb, amsmath, enumerate, libertine, microtype, hyperref,tikz-cd,hyperref}
\usepackage[normalem]{ulem}
\usepackage{fullpage}
\usepackage[T1]{fontenc}
\renewcommand{\labelitemi}{$\cdot$}
\usepackage{mathrsfs}
\usepackage{phaistos}
\definecolor{dark-red}{rgb}{0.4,0.15,0.15}
%   \definecolor{dark-blue}{rgb}{0.15,0.15,0.4}
%   \definecolor{medium-blue}{rgb}{0,0,0.5}
\setcounter{secnumdepth}{2}
\setcounter{tocdepth}{1}
\hypersetup{
    colorlinks, linkcolor=dark-red,
    citecolor=dark-red, urlcolor=dark-red
}


\theoremstyle{plain}
\newtheorem{prop}{Proposition}%[section]
\newtheorem{lemma}[prop]{Lemma}
\newtheorem{thm}[prop]{Theorem}
\newtheorem{obs}[prop]{Observation}
\newtheorem{app}[prop]{Application}
\newtheorem*{MainThm}{Main Theorem}
\newtheorem{cor}[prop]{Corollary}
\newtheorem{conj}[prop]{Conjecture}
\theoremstyle{remark}
\newtheorem{rmk}[prop]{Remark}
\newtheorem{prob}{Problem}
\newtheorem{bonus}[prop]{Bonus Problem}
\newtheorem{exc}{Exercise}
\theoremstyle{definition}
\newtheorem{ex}[prop]{Example}
\theoremstyle{definition}
\newtheorem{defn}[prop]{Definition}

\newcommand{\RR}{\mathbb{R}}
\newcommand{\ZZ}{\mathbb{Z}}
\newcommand{\CC}{\mathbb{C}}
\newcommand{\NN}{\mathbb{N}}
\newcommand{\QQ}{\mathbb{Q}}
\newcommand{\PP}{\mathbb{P}}
\newcommand{\kk}{\mathsf{k}}
\newcommand{\FF}{\mathbb{F}}
\newcommand{\cS}{\mathcal{S}}
\newcommand{\cT}{\mathcal{T}}
\newcommand{\ssC}{\mathsf{C}}

\newcommand{\id}{\operatorname{id}}
\newcommand{\Mat}{\mathsf{Mat}}

\title{Math 111: Calculus\\ Homework due Wednesday Week 9}
% uncomment the following line and add your name if you are using this as a template for solutions
% \author{Your Name}
% 5.1 13, 22, 44 6.1 15, 16, 17

\begin{document}
\maketitle

\begin{prob}
Let $L_n$ denote the left-endpoint Riemann sum using $n$ subintervals and let $R_n$ denote the corresponding right-endpoint Riemann sum. Compute the indicated left and right Riemann sums for the given functions on the indicated interval.
\begin{enumerate}[(a)]
\item $R_4$ for $g(x) = \cos(\pi x)$ on $[0,1]$
\item $L_8$ for $x^2-2x+1$ on $[0,2]$
\end{enumerate}
\end{prob}


\begin{prob}
Estimate the area under the depicted curve by using the left Riemann sum $L_8$.
\begin{center}
\includegraphics[width=3in]{left_riemann.jpeg}
\end{center}
\end{prob}

\begin{prob}
Suppose $f\colon [a,b]\to \RR$ is an integrable function.
\begin{enumerate}[(a)]
\item Explain why $R_n-L_n = (f(b)-f(a))\cdot \frac{b-a}{n}$.
\item Suppose further that $f$ is increasing on $[a,b]$. Explain why, in this case, the error betweeen either $L_n$ or $R_n$ and $\int_a^b f(x)\,dx$ is at most $(f(b)-f(a))\cdot \frac{b-a}{n}$.
\end{enumerate}
\end{prob}

\begin{prob}
Use the information presented in the below graph of $y=h(x)$ to compute $\int_0^{12}h(x)\,dx$ geometrically.
\begin{center}
\includegraphics[width=3in]{geom_int.jpeg}
\end{center}
\end{prob}

\begin{prob}
Suppose that
\[
  A = \int_0^{2\pi} \sin^2t\,dt\qquad\text{and}\qquad B = \int_0^{2\pi} \cos^2t\,dt.
\]
Show that $A+B=2\pi$ and $A=B$.
\end{prob}

\begin{prob}
Suppose $f\colon [0,2]\to \RR$ is given by $f(x) = \sqrt{4-x^2}$.
\begin{enumerate}[(a)]
\item Use a geometric argument to compute the average value $f_{ave}$ of $f$ over $[0,2]$.
\item Find a point $c$ in $[0,2]$ such that $f(c) = f_{ave}$. 
\end{enumerate}
\end{prob}

\begin{prob}
Use FTC1 to compute the following integrals:
\begin{enumerate}[(a)]
\item $\frac{d}{dx} \int_1^x e^{-t^2}\,dt$
\item $\frac{d}{dx} \int_1^{\sqrt{x}} \frac{t^2}{1+t^4}\,dt$
\end{enumerate}
\end{prob}

\begin{prob}
The graph of $y = \int_0^x f(t)\,dt$, where $f$ is a piecewise constant function, is shown here.

\begin{center}
\includegraphics[width=3in]{accum_graph.jpeg}
\end{center}

\begin{enumerate}[(a)]
\item Over what intervals is $f$ positive? negative? Over which intervals, if any, is $f$ equal to $0$?
\item What are the maximum and minimum values of $f$?
\item What is the average value of $f$?
\end{enumerate}
\end{prob}


\end{document}