\documentclass[11pt,twoside]{amsart}
\usepackage{amssymb, amsmath, enumerate, palatino, hyperref,tikz,tikz-cd}
\usepackage[normalem]{ulem}
\usepackage{fullpage}
\usepackage[T1]{fontenc}
\renewcommand{\labelitemi}{\guillemotright}
\usepackage{mathrsfs}
\usepackage{phaistos}


\theoremstyle{plain}
\newtheorem{prop}{Proposition}%[section]
\newtheorem{lemma}[prop]{Lemma}
\newtheorem{thm}[prop]{Theorem}
\newtheorem{obs}[prop]{Observation}
\newtheorem{app}[prop]{Application}
\newtheorem*{MainThm}{Main Theorem}
\newtheorem{cor}[prop]{Corollary}
\newtheorem{conj}[prop]{Conjecture}
\theoremstyle{remark}
\newtheorem{rmk}[prop]{Remark}
\newtheorem{prob}{Problem}
\newtheorem{bonus}[prop]{Bonus Problem}
\newtheorem{exc}{Exercise}
\theoremstyle{definition}
\newtheorem{ex}[prop]{Example}
\theoremstyle{definition}
\newtheorem{defn}[prop]{Definition}

\newcommand{\RR}{\mathbb{R}}
\newcommand{\ZZ}{\mathbb{Z}}
\newcommand{\CC}{\mathbb{C}}
\newcommand{\NN}{\mathbb{N}}
\newcommand{\QQ}{\mathbb{Q}}
\newcommand{\PP}{\mathbb{P}}
\newcommand{\TT}{\mathbb{T}}
\newcommand{\HH}{\mathbb{H}}
\newcommand{\kk}{\mathsf{k}}
\newcommand{\FF}{\mathbb{F}}
\newcommand{\cS}{\mathcal{S}}
\newcommand{\cT}{\mathcal{T}}
\newcommand{\ssC}{\mathsf{C}}
\newcommand{\sU}{\mathscr{U}}
\newcommand{\ol}{\overline}

\newcommand{\id}{\operatorname{id}}
\newcommand{\Int}{\operatorname{Int}}
\newcommand{\cs}{\mathbin{\#}}
\newcommand{\Ab}{\mathsf{Ab}}
\newcommand{\Top}{\mathsf{Top}}
\newcommand{\Grp}{\mathsf{Grp}}
\newcommand{\Aut}{\operatorname{Aut}}
\newcommand{\SU}{\mathrm{SU}}
\newcommand{\SO}{\mathrm{SO}}
\renewcommand{\O}{\mathrm{O}}
\newcommand{\U}{\mathrm{U}}
\newcommand{\GL}{\mathrm{GL}}
\newcommand{\gl}{\mathfrak{gl}}
\newcommand{\Lie}{\operatorname{Lie}}



\title{Math 546: Manifolds\\ Homework due Friday Week 5}
%\author{Your Name}

\begin{document}
\maketitle

\noindent Problems taken from \emph{Introduction to Smooth Manifolds} are marked ISM $x$--$y$. Please review the syllabus for expectations and policies regarding homework.

\begin{prob}[ISM 11--5]
For any smooth manifold $M$, show that $T^*M$ is a trivial vector bundle if and only if $TM$ is trivial.
\end{prob}

\begin{prob}[ISM 11--13]
The \emph{length} of a smooth curve segment $\gamma\colon [a,b]\to \RR^n$ is defined to be the value of the (ordinary) integral
\[
  L(\gamma) = \int_a^b |\gamma'(t)|]\,dt.
\]
Show that there is no smooth covector field $\omega\in \mathfrak X^*(\RR^n)$ with the property that $\int_\gamma \omega = L(\gamma)$ for every smooth curve $\gamma$.
\end{prob}

\begin{prob}[ISM 11--14]
Consider the following two covector fields on $\RR^3$:
\[
\begin{aligned}
  \omega &= -\frac{4z\,dx}{(x^2+1)^2}+\frac{2y\,dy}{y^2+1}+\frac{2x\,dz}{x^2+1},\\
  \eta &= -\frac{4xz\,dx}{(x^2+1)^2}+\frac{2y\,dy}{y^2+1}+\frac{2\,dz}{x^2+1}.
\end{aligned}
\]
\begin{enumerate}[(a)]
\item Set up and evaluate the line integral of each covector field along the straight line segment from $(0,0,0)$ to $(1,1,1)$.
\item Determine whether either of these covector fields is exact.
\item For each one that is exact, find a potential function and use it to recompute the line integral.
\end{enumerate}
\end{prob}

\begin{prob}[ISM 11--16]
Let $M$ be a compact smooth manifold of positive dimension. Show that every exact covector field on $M$ vanishes at least at two points in each component of $M$.
\end{prob}

\begin{prob}[ISM 12--7]
Let $(e^1,e^2,e^3)$ be the standard dual basis for $(\RR^3)^*$. Show that $e^1\otimes e^2\otimes e^3$ is not equal to a sum of an alternating tensor and a symmetric tensor.
\end{prob}


\end{document}