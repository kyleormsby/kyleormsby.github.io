\documentclass[11pt,twoside]{amsart}
\usepackage{amssymb, amsmath, enumerate, palatino, hyperref,tikz,tikz-cd}
\usepackage[normalem]{ulem}
\usepackage{fullpage}
\usepackage[T1]{fontenc}
\renewcommand{\labelitemi}{\guillemotright}
\usepackage{mathrsfs}
\usepackage{phaistos}


\theoremstyle{plain}
\newtheorem{prop}{Proposition}%[section]
\newtheorem{lemma}[prop]{Lemma}
\newtheorem{thm}[prop]{Theorem}
\newtheorem{obs}[prop]{Observation}
\newtheorem{app}[prop]{Application}
\newtheorem*{MainThm}{Main Theorem}
\newtheorem{cor}[prop]{Corollary}
\newtheorem{conj}[prop]{Conjecture}
\theoremstyle{remark}
\newtheorem{rmk}[prop]{Remark}
\newtheorem{prob}{Problem}
\newtheorem{bonus}[prop]{Bonus Problem}
\newtheorem{exc}{Exercise}
\theoremstyle{definition}
\newtheorem{ex}[prop]{Example}
\theoremstyle{definition}
\newtheorem{defn}[prop]{Definition}

\newcommand{\RR}{\mathbb{R}}
\newcommand{\ZZ}{\mathbb{Z}}
\newcommand{\CC}{\mathbb{C}}
\newcommand{\NN}{\mathbb{N}}
\newcommand{\QQ}{\mathbb{Q}}
\newcommand{\PP}{\mathbb{P}}
\newcommand{\TT}{\mathbb{T}}
\newcommand{\HH}{\mathbb{H}}
\newcommand{\kk}{\mathsf{k}}
\newcommand{\FF}{\mathbb{F}}
\newcommand{\cS}{\mathcal{S}}
\newcommand{\cT}{\mathcal{T}}
\newcommand{\ssC}{\mathsf{C}}
\newcommand{\sU}{\mathscr{U}}
\newcommand{\ol}{\overline}
\newcommand{\dd}{\mathsf{d}}

\newcommand{\id}{\operatorname{id}}
\newcommand{\Int}{\operatorname{Int}}
\newcommand{\cs}{\mathbin{\#}}
\newcommand{\Ab}{\mathsf{Ab}}
\newcommand{\Top}{\mathsf{Top}}
\newcommand{\Grp}{\mathsf{Grp}}
\newcommand{\Aut}{\operatorname{Aut}}
\newcommand{\SU}{\mathrm{SU}}
\newcommand{\SO}{\mathrm{SO}}
\renewcommand{\O}{\mathrm{O}}
\newcommand{\U}{\mathrm{U}}
\newcommand{\GL}{\mathrm{GL}}
\newcommand{\gl}{\mathfrak{gl}}
\newcommand{\Lie}{\operatorname{Lie}}



\title{Math 546: Manifolds\\ Homework due Friday Week 7}
%\author{Your Name}

\begin{document}
\maketitle

\noindent Problems taken from \emph{Introduction to Smooth Manifolds} are marked ISM $x$--$y$. Please review the syllabus for expectations and policies regarding homework.

\begin{prob}[14--4]
Let $V$ be a finite-dimensional vector space.
\begin{enumerate}[(a)]
\item Show that an ordered $k$-tuple $(v_1,\ldots,v_k)$ of elements of $V$ is linearly dependent if and only if $v_1\wedge\cdots v_k=0$.
\item Show that two linearly independent ordered $k$-tuples $(v_1,\ldots,v_k)$ and $(w_1,\ldots,w_k)$ of elements of $V$ have the same span if and only if
\[
  v_1\wedge\cdots\wedge v_k = cw_1\wedge\cdots\wedge w_k
\]
for some nonzero scalar $c$.
\end{enumerate}
\end{prob}

\begin{prob}[14--5, \textsc{Cartan's Lemma}]
Let $M$ be a smooth $n$-manifold with or without boundary, and let $(\omega^1,\ldots,\omega^k)$ be an ordered $k$-tuple of smooth $1$-forms on an open subset $U\subseteq M$ such that $(\omega^1|_p,\ldots,\omega^k|_p)$ is linearly independent for each $p\in U$. Given smooth $1$-forms $\alpha^1,\ldots,\alpha^k$ on $U$ such that
\[
  \sum_{i=1}^k \alpha^i\wedge \omega^i = 0
\]
show that each $\alpha^i$ can be written as a linear combination of $\omega^1,\ldots,\omega^k$ with smooth coefficients.
\end{prob}

\begin{prob}[ISM 15--1]
Suppose $M$ is a smooth manifold that is the union of two orientable open submanifolds with connected intersection. Show that $M$ is orientable. Use this to give another proof that $S^n$ is orientable.
\end{prob}

\begin{prob}[ISM 16--3(a)]
Suppose $E$ and $M$ are oriented smooth $n$-manifolds with or without boundary, and $\pi\colon E\to M$ is a smooth orientation-preserving $k$-sheeted covering map or generalized covering map. Show that
\[
  \int_E \pi^*\omega = k\int_M \omega
\]
for any compactly supported $n$-form $\omega$ on $M$.
\end{prob}

\begin{prob}
Let $M$ be an oriented smooth $n$ manifold, $f\in C^\infty(M)$, and $\omega\in \Omega^{n-1}_c(M)$. Use Stokes' theorem to prove the following generalization of integration by parts:
\[
  \int_M f\, \dd\omega = \int_{\partial M}f\omega - \int_M \dd f\wedge \omega.
\]
\end{prob}


\end{document}