\documentclass[11pt,twoside]{amsart}
\usepackage{amssymb, amsmath, enumerate, palatino, hyperref,tikz,tikz-cd}
\usepackage[normalem]{ulem}
\usepackage{fullpage}
\usepackage[T1]{fontenc}
\renewcommand{\labelitemi}{\guillemotright}
\usepackage{mathrsfs}
\usepackage{phaistos}


\theoremstyle{plain}
\newtheorem{prop}{Proposition}%[section]
\newtheorem{lemma}[prop]{Lemma}
\newtheorem{thm}[prop]{Theorem}
\newtheorem{obs}[prop]{Observation}
\newtheorem{app}[prop]{Application}
\newtheorem*{MainThm}{Main Theorem}
\newtheorem{cor}[prop]{Corollary}
\newtheorem{conj}[prop]{Conjecture}
\theoremstyle{remark}
\newtheorem{rmk}[prop]{Remark}
\newtheorem{prob}{Problem}
\newtheorem{bonus}[prop]{Bonus Problem}
\newtheorem{exc}{Exercise}
\theoremstyle{definition}
\newtheorem{ex}[prop]{Example}
\theoremstyle{definition}
\newtheorem{defn}[prop]{Definition}

\newcommand{\RR}{\mathbb{R}}
\newcommand{\ZZ}{\mathbb{Z}}
\newcommand{\CC}{\mathbb{C}}
\newcommand{\NN}{\mathbb{N}}
\newcommand{\QQ}{\mathbb{Q}}
\newcommand{\PP}{\mathbb{P}}
\newcommand{\TT}{\mathbb{T}}
\newcommand{\HH}{\mathbb{H}}
\newcommand{\kk}{\mathsf{k}}
\newcommand{\FF}{\mathbb{F}}
\newcommand{\cS}{\mathcal{S}}
\newcommand{\cT}{\mathcal{T}}
\newcommand{\ssC}{\mathsf{C}}
\newcommand{\sU}{\mathscr{U}}
\newcommand{\ol}{\overline}
\newcommand{\dd}{\mathsf{d}}

\newcommand{\id}{\operatorname{id}}
\newcommand{\Int}{\operatorname{Int}}
\newcommand{\cs}{\mathbin{\#}}
\newcommand{\Ab}{\mathsf{Ab}}
\newcommand{\Top}{\mathsf{Top}}
\newcommand{\Grp}{\mathsf{Grp}}
\newcommand{\Aut}{\operatorname{Aut}}
\newcommand{\SU}{\mathrm{SU}}
\newcommand{\SO}{\mathrm{SO}}
\renewcommand{\O}{\mathrm{O}}
\newcommand{\U}{\mathrm{U}}
\newcommand{\GL}{\mathrm{GL}}
\newcommand{\gl}{\mathfrak{gl}}
\newcommand{\Lie}{\operatorname{Lie}}
\newcommand{\dR}{\mathrm{dR}}



\title{Math 546: Manifolds\\ Homework due Friday Week 8}
%\author{Your Name}

\begin{document}
\maketitle

\noindent Problems taken from \emph{Introduction to Smooth Manifolds} are marked ISM $x$--$y$. Please review the syllabus for expectations and policies regarding homework.

\begin{prob}[ISM 17--1]
Let $M$ be a smooth manifold with or without boundary, and let $\omega\in\Omega^p(M),\eta\in\Omega^q(M)$ be closed forms. Show that the de Rham cohomology class of $\omega\wedge\eta$ depends only on the cohomology classes of $\omega$ and $\eta$, and thus there is a well-defined bilinear map
\[
\begin{aligned}
  \smile\colon H^p_{\dR}(M)\times H^q_{\dR}(M)&\longrightarrow H^{p+q}_{\dR}(M)\\
  ([\omega],[\eta])&\longmapsto [\omega]\smile[\eta]:=[\omega\wedge\eta]
\end{aligned}
\]
called the \emph{cup product}.
\end{prob}

\begin{prob}[ISM 17--5]
For each $n\ge 1$, compute the de Rham cohomology groups of $\RR^n\smallsetminus \{e_1,-e_1\}$; and for each nonzero cohomology group, give specific differential forms whose cohomology classes form a basis.
\end{prob}

\begin{prob}[ISM 17--6]
Let $M$ be a connected smooth manifold of dimension $n\ge 3$. For any $x\in M$ and $0\le p\le n-2$, prove that the map $H^p_{\dR}\to H^p_{\dR}(M\smallsetminus \{x\})$ induced by the inclusion $M\smallsetminus\{x\}\hookrightarrow M$ is an isomorphism. Prove that the same is true for $p=n-1$ if $M$ is compact and orientable. [\emph{Hint}: Use the Mayer--Vietoris theorem. The cases $p=0$, $p=1$, and $p=n-1$ require special handling.]
\end{prob}

\begin{prob}[ISM 17--10]
Let $p\in \CC[z]$ be a nonzero polynomial in the variable $z$ with complex coefficients. Read Problems 2--8 and 2--9 and convince yourself that there is a unique well-defined smooth map $\tilde p\colon \CC\PP^1\to\CC\PP^1$ such that $\tilde p([z,1]) = [p(z),1]$. (You do not need to write up solutions to these progblems.) Prove that the degree of $\tilde p$ (as a smooth map between manifolds) is equal to the degree of the polynomial $p$ in the usual sense.
\end{prob}

\begin{prob}[ISM 17--12]
Suppose $M$ and $N$ are compact, connected, oriented smooth $n$-manifolds, and $F\colon M\to N$ is a smooth map. Prove that if $\int_M F^*\eta\ne 0$ for some $\eta\in \Omega^n(N)$, then $F$ is surjective. Give an example to show that $F$ can be surjective even if $\int_M F^*\eta = 0$ for every $\eta\in \Omega^n(N)$.
\end{prob}


\end{document}