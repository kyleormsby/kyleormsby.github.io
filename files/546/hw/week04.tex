\documentclass[11pt,twoside]{amsart}
\usepackage{amssymb, amsmath, enumerate, palatino, hyperref,tikz,tikz-cd}
\usepackage[normalem]{ulem}
\usepackage{fullpage}
\usepackage[T1]{fontenc}
\renewcommand{\labelitemi}{\guillemotright}
\usepackage{mathrsfs}
\usepackage{phaistos}


\theoremstyle{plain}
\newtheorem{prop}{Proposition}%[section]
\newtheorem{lemma}[prop]{Lemma}
\newtheorem{thm}[prop]{Theorem}
\newtheorem{obs}[prop]{Observation}
\newtheorem{app}[prop]{Application}
\newtheorem*{MainThm}{Main Theorem}
\newtheorem{cor}[prop]{Corollary}
\newtheorem{conj}[prop]{Conjecture}
\theoremstyle{remark}
\newtheorem{rmk}[prop]{Remark}
\newtheorem{prob}{Problem}
\newtheorem{bonus}[prop]{Bonus Problem}
\newtheorem{exc}{Exercise}
\theoremstyle{definition}
\newtheorem{ex}[prop]{Example}
\theoremstyle{definition}
\newtheorem{defn}[prop]{Definition}

\newcommand{\RR}{\mathbb{R}}
\newcommand{\ZZ}{\mathbb{Z}}
\newcommand{\CC}{\mathbb{C}}
\newcommand{\NN}{\mathbb{N}}
\newcommand{\QQ}{\mathbb{Q}}
\newcommand{\PP}{\mathbb{P}}
\newcommand{\TT}{\mathbb{T}}
\newcommand{\HH}{\mathbb{H}}
\newcommand{\kk}{\mathsf{k}}
\newcommand{\FF}{\mathbb{F}}
\newcommand{\cS}{\mathcal{S}}
\newcommand{\cT}{\mathcal{T}}
\newcommand{\ssC}{\mathsf{C}}
\newcommand{\sU}{\mathscr{U}}
\newcommand{\ol}{\overline}

\newcommand{\id}{\operatorname{id}}
\newcommand{\Int}{\operatorname{Int}}
\newcommand{\cs}{\mathbin{\#}}
\newcommand{\Ab}{\mathsf{Ab}}
\newcommand{\Top}{\mathsf{Top}}
\newcommand{\Grp}{\mathsf{Grp}}
\newcommand{\Aut}{\operatorname{Aut}}
\newcommand{\SU}{\mathrm{SU}}
\newcommand{\SO}{\mathrm{SO}}
\renewcommand{\O}{\mathrm{O}}
\newcommand{\U}{\mathrm{U}}
\newcommand{\GL}{\mathrm{GL}}
\newcommand{\gl}{\mathfrak{gl}}
\newcommand{\Lie}{\operatorname{Lie}}



\title{Math 546: Manifolds\\ Homework due Friday Week 4}
%\author{Your Name}

\begin{document}
\maketitle

\noindent Problems taken from \emph{Introduction to Smooth Manifolds} are marked ISM $x$--$y$. Please review the syllabus for expectations and policies regarding homework.

\begin{prob}[ISM 9--12]
Suppose $M_1$ and $M_2$ are connected smooth $n$-manifolds and $M_1\cs M_2$ is their smooth connected sum (see Example 9.31). Show that the smooth structure on $M_1\cs M_2$ can be chosen in such a way that there are open subsets $\tilde M_1,\tilde M_2\subseteq M_1\cs M_2$ that are diffeomorphic to $M_1\smallsetminus \{p_1\}$ and $M_2\smallsetminus \{p_2\}$, respectively, such that $\tilde M_1\cup \tilde M_2 = M_1\cs M_2$ and $\tilde M_1\cap \tilde M_2$ is diffeomorphic to $(-1,1)\times S^{n-1}$.
\end{prob}

\begin{prob}[ISM 9--16]
Give an example of smooth vector fields $V$, $\tilde V$, and $W$ on $\RR^2$ such that $V = \tilde V=\partial/\partial x$ along the $x$-axis but $\mathscr L_V W\ne \mathscr L_{\tilde V}W$ at the origin.
\end{prob}

\begin{prob}[ISM 10--1]
Let $E$ be the total space of the M\"obius bundle constructed in Example 10.3.
\begin{enumerate}[(a)]
\item Show that $E$ has a unique smooth structure such that the quotient map $q\colon \RR^2\to E$ is a smooth covering map.
\item Show that $\pi\colon E\to S^1$ is a smooth rank-1 vector bundle.
\item Show that it is not the trivial bundle.
\end{enumerate}
\end{prob}

\begin{prob}[ISM 10--11]
Prove Proposition 10.26: A bijective bundle homomorphism is a bundle isomorphism.
\end{prob}

\begin{prob}[ISM 10--18]
Suppose $S$ is a properly embedded codimension $k$ submanifold of $\RR^n$. Show that the following are equivalent:
\begin{enumerate}[(a)]
\item There exists a smooth defining function for $S$ on some neighborhood $U$ of $S$ in $\RR^n$; \emph{i.e.}, there is a smooth function $\Phi\colon U\to \RR^k$ such that $S$ is a regular level set of $\Phi$.
\item The normal bundle $NS$ is a trivial vector bundle.
\end{enumerate}
\end{prob}



\end{document}