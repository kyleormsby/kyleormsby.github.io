\documentclass[11pt,twoside]{amsart}
\usepackage{amssymb, amsmath, enumerate, palatino, hyperref,tikz,tikz-cd}
\usepackage[normalem]{ulem}
\usepackage{fullpage}
\usepackage[T1]{fontenc}
\renewcommand{\labelitemi}{\guillemotright}
\usepackage{mathrsfs}
\usepackage{phaistos}


\theoremstyle{plain}
\newtheorem{prop}{Proposition}%[section]
\newtheorem{lemma}[prop]{Lemma}
\newtheorem{thm}[prop]{Theorem}
\newtheorem{obs}[prop]{Observation}
\newtheorem{app}[prop]{Application}
\newtheorem*{MainThm}{Main Theorem}
\newtheorem{cor}[prop]{Corollary}
\newtheorem{conj}[prop]{Conjecture}
\theoremstyle{remark}
\newtheorem{rmk}[prop]{Remark}
\newtheorem{prob}{Problem}
\newtheorem{bonus}[prop]{Bonus Problem}
\newtheorem{exc}{Exercise}
\theoremstyle{definition}
\newtheorem{ex}[prop]{Example}
\theoremstyle{definition}
\newtheorem{defn}[prop]{Definition}

\newcommand{\RR}{\mathbb{R}}
\newcommand{\ZZ}{\mathbb{Z}}
\newcommand{\CC}{\mathbb{C}}
\newcommand{\NN}{\mathbb{N}}
\newcommand{\QQ}{\mathbb{Q}}
\newcommand{\PP}{\mathbb{P}}
\newcommand{\TT}{\mathbb{T}}
\newcommand{\HH}{\mathbb{H}}
\newcommand{\kk}{\mathsf{k}}
\newcommand{\FF}{\mathbb{F}}
\newcommand{\cS}{\mathcal{S}}
\newcommand{\cT}{\mathcal{T}}
\newcommand{\ssC}{\mathsf{C}}
\newcommand{\sU}{\mathscr{U}}
\newcommand{\ol}{\overline}

\newcommand{\id}{\operatorname{id}}
\newcommand{\Int}{\operatorname{Int}}
\newcommand{\cs}{\mathbin{\#}}
\newcommand{\Ab}{\mathsf{Ab}}
\newcommand{\Top}{\mathsf{Top}}
\newcommand{\Grp}{\mathsf{Grp}}
\newcommand{\Aut}{\operatorname{Aut}}
\newcommand{\SU}{\mathrm{SU}}
\newcommand{\SO}{\mathrm{SO}}
\renewcommand{\O}{\mathrm{O}}
\newcommand{\U}{\mathrm{U}}
\newcommand{\GL}{\mathrm{GL}}
\newcommand{\gl}{\mathfrak{gl}}
\newcommand{\Lie}{\operatorname{Lie}}



\title{Math 546: Manifolds\\ Homework due Friday Week 3}
%\author{Your Name}

\begin{document}
\maketitle

\noindent Problems taken from \emph{Introduction to Smooth Manifolds} are marked ISM $x$--$y$. Please review the syllabus for expectations and policies regarding homework.

\begin{prob}[ISM 7--15]
Let $G$ be a Lie group and let $G_0$ be its identity component. Then $G_0$ is a normal subgroup of $G$, and is the only connected open subgroup. Every connected component of $G$ is diffeomorphic to $G_0$.
\end{prob}

\begin{prob}[ISM 8--10]
Let $M$ be the open submanifold of $\RR^2$ where both $x$ and $y$ are positive, and let $F\colon M\to M$ be the map $F(x,y) = (xy,y/x)$. Show that $F$ is a diffeomorphism, and compute $F_*X$ and $F_*Y$, where
\[
  X = x\frac\partial{\partial x}+y\frac\partial{\partial y},\qquad Y = y\frac\partial{\partial x}.
\]
Also compute $[X,Y]$ and $[F_*X,F_*Y]$.
\end{prob}

\begin{prob}[ISM 8--25]
Prove that if $G$ is an Abelian Lie group, then $\mathfrak g = \Lie(G)$ is Abelian. [\emph{Hint}: Show that the inversion map $i\colon G\to G$ is a group isomorphism. Prove any parts of ISM Problem 7--2 that might be relevant.]
\end{prob}

\begin{prob}[ISM 8--29]
Theorem 8.46 implies that the Lie algebra of any Lie subgroup of $\GL_n\RR$ is canonically isomorphic to a subalgebra of $\gl_n\RR$, with a similar statement for Lie subgroups of $\GL_n\CC$. Under this isomorphism, show that
\[
  \Lie(\SO(n))\cong \mathfrak o(n) = \{A\in \gl_n\RR\mid A^T+A=0\}
\]
and
\[
  \Lie(\U(n))\cong \mathfrak u(n) = \{A\in \gl_n\CC\mid A^*+A=0\}.
\]
\end{prob}

\begin{prob}[ISM 9--3(b)]
Compute the flow the vector field
\[
  W = x\frac\partial{\partial x}+2y\frac\partial{\partial y}
\]
on $\RR^2$.
\end{prob}

\end{document}