\documentclass[11pt,twoside]{amsart}
\usepackage{amssymb, amsmath, enumerate, palatino, hyperref,tikz,tikz-cd}
\usepackage[normalem]{ulem}
\usepackage{fullpage}
\usepackage[T1]{fontenc}
\renewcommand{\labelitemi}{\guillemotright}
\usepackage{mathrsfs}
\usepackage{phaistos}


\theoremstyle{plain}
\newtheorem{prop}{Proposition}%[section]
\newtheorem{lemma}[prop]{Lemma}
\newtheorem{thm}[prop]{Theorem}
\newtheorem{obs}[prop]{Observation}
\newtheorem{app}[prop]{Application}
\newtheorem*{MainThm}{Main Theorem}
\newtheorem{cor}[prop]{Corollary}
\newtheorem{conj}[prop]{Conjecture}
\theoremstyle{remark}
\newtheorem{rmk}[prop]{Remark}
\newtheorem{prob}{Problem}
\newtheorem{bonus}[prop]{Bonus Problem}
\newtheorem{exc}{Exercise}
\theoremstyle{definition}
\newtheorem{ex}[prop]{Example}
\theoremstyle{definition}
\newtheorem{defn}[prop]{Definition}

\newcommand{\RR}{\mathbb{R}}
\newcommand{\ZZ}{\mathbb{Z}}
\newcommand{\CC}{\mathbb{C}}
\newcommand{\NN}{\mathbb{N}}
\newcommand{\QQ}{\mathbb{Q}}
\newcommand{\PP}{\mathbb{P}}
\newcommand{\TT}{\mathbb{T}}
\newcommand{\HH}{\mathbb{H}}
\newcommand{\kk}{\mathsf{k}}
\newcommand{\FF}{\mathbb{F}}
\newcommand{\cS}{\mathcal{S}}
\newcommand{\cT}{\mathcal{T}}
\newcommand{\ssC}{\mathsf{C}}
\newcommand{\sU}{\mathscr{U}}
\newcommand{\ol}{\overline}

\newcommand{\id}{\operatorname{id}}
\newcommand{\Int}{\operatorname{Int}}
\newcommand{\cs}{\mathbin{\#}}
\newcommand{\Ab}{\mathsf{Ab}}
\newcommand{\Top}{\mathsf{Top}}
\newcommand{\Grp}{\mathsf{Grp}}
\newcommand{\Aut}{\operatorname{Aut}}
\newcommand{\SU}{\mathrm{SU}}
\newcommand{\SO}{\mathrm{SO}}
\renewcommand{\O}{\mathrm{O}}



\title{Math 546: Manifolds\\ Homework due Friday Week 2}
%\author{Your Name}

\begin{document}
\maketitle

\noindent Problems taken from \emph{Introduction to Smooth Manifolds} are marked ISM $x$--$y$. Please review the syllabus for expectations and policies regarding homework.

\begin{prob}[ISM 7--11]
Considering $S^{2n+1}$ as the unit sphere in $\CC^{n+1}$, define an action of $S^1$ on $S^{2n+1}$, called the \emph{Hopf action}, by
\[
  z\cdot (w^1,\ldots,w^{n+1}) = (zw^1,\ldots,zw^{n+1}).
\]
Show that this action is smooth and its orbits are disjoint unit circles in $\CC^{n+1}$ whose union is $S^{2n+1}$.
\end{prob}

% \begin{prob}[ISM 7--16]
% Prove that $\SU(2)$ is diffeomorphic to $S^3$.
% \end{prob}

% \begin{prob}[parts of ISM 7--19, 7--20, and 7--21]
% \begin{enumerate}[(a)]
% \item Suppose $G$, $N$, and $H$ are Lie groups. Prove that $G$ is isomorphic to a semi-direct product $N\rtimes H$ if and only if there are Lie group homomorphisms $\varphi\colon G\to H$ and $\psi\colon H\to G$ such that $\varphi\circ \psi = \id_H$ and $\ker\varphi\cong N$.
% \item Prove that $\O(n)\cong \SO(n)\rtimes \O(1)$ as Lie groups for $n\ge 1$.
% \item Prove that $\O(n)$ is not isomorphic to $\SO(n)\times \O(1)$ for $n>1$.
% \end{enumerate}
% \end{prob}

\begin{prob}[ISM 7--22]
Let $\HH=\CC\times \CC$ and consider it as a real vector space. Define a bilinear product
\[
\begin{aligned}
  \HH\times\HH&\longrightarrow \HH\\
  ((a,b),(c,d))&\longmapsto (ac-d\ol b,\ol ad+cb)
\end{aligned}
\]
for $a,b,c,d\in \CC$. This makes $\HH$ a $4$-dimensional $\RR$-algebra called the \emph{quaternions}. For each $p=(a,b)\in \HH$, define $p^* = (\ol a,-b)$. It is useful to work with the basis $(1,i,j,k) = ((1,0),(i,0),(0,1),(0,-i))$ for $\HH$. One may verify that this basis satisfies
\[
\begin{aligned}
&i^2=j^2=k^2=-1,\qquad 1q=q1=q\text{ for all }q\in \HH,\\
&ij=-ji=k,\qquad jk=-kj=i,\qquad ki=-ik=j,\\
&1^*=1,\qquad i^*=-i,\qquad j^*=-j,\qquad k^*=-k.
\end{aligned}
\]
A quaternion $p$ is said to be \emph{real} when $p^*=p$, and \emph{imaginary} when $p^*=-p$. Real quaternions can be identified with real numbers via the correspondence $\lambda \leftrightarrow x1$.
\begin{enumerate}[(a)]
\item Show that quaternionic multiplication is associative but not commutative.
\item Show that $(pq)^* = q^*p^*$ for all $p,q\in \HH$.
\item Show that $\langle p,q\rangle = \frac{1}{2}(p^*q+q^*p)$ is an inner product on $\HH$ with associated norm satisfying $|pq|=|p||q|$.
\item Show that every nonzero quaternion has a two-sided multiplicative inverse given by $p^{-1} = |p|^{-2}p^*$.
\item Show that the set $\HH^\times$ of nonzero quaternions is a Lie group under quaternionic multiplication.
\end{enumerate}
\end{prob}

\begin{prob}[ISM 7--23]
Let $\HH^\times$ be the Lie group of nonzero quaternions, and let $S^3\subseteq \HH^\times$ be the set of norm $1$ quaternions. Show that $S^3$ is a properly embedded Lie subgroup of $\HH^\times$, isomorphic to $\SU(2)$.
\end{prob}

\begin{prob}[ISM 8--4]
Let $M$ be a smooth manifold with boundary. Show that there exists a global smooth vector field on $M$ whose restriction to $\partial M$ is everywhere inward-pointing, and one whose restriction to $\partial M$ is everywhere outward-pointing.
\end{prob}







\end{document}