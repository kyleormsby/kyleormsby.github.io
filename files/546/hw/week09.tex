\documentclass[11pt,twoside]{amsart}
\usepackage{amssymb, amsmath, enumerate, palatino, hyperref,tikz,tikz-cd}
\usepackage[normalem]{ulem}
\usepackage{fullpage}
\usepackage[T1]{fontenc}
\renewcommand{\labelitemi}{\guillemotright}
\usepackage{mathrsfs}
\usepackage{phaistos}


\theoremstyle{plain}
\newtheorem{prop}{Proposition}%[section]
\newtheorem{lemma}[prop]{Lemma}
\newtheorem{thm}[prop]{Theorem}
\newtheorem{obs}[prop]{Observation}
\newtheorem{app}[prop]{Application}
\newtheorem*{MainThm}{Main Theorem}
\newtheorem{cor}[prop]{Corollary}
\newtheorem{conj}[prop]{Conjecture}
\theoremstyle{remark}
\newtheorem{rmk}[prop]{Remark}
\newtheorem{prob}{Problem}
\newtheorem{bonus}[prop]{Bonus Problem}
\newtheorem{exc}{Exercise}
\theoremstyle{definition}
\newtheorem{ex}[prop]{Example}
\theoremstyle{definition}
\newtheorem{defn}[prop]{Definition}

\newcommand{\RR}{\mathbb{R}}
\newcommand{\ZZ}{\mathbb{Z}}
\newcommand{\CC}{\mathbb{C}}
\newcommand{\NN}{\mathbb{N}}
\newcommand{\QQ}{\mathbb{Q}}
\newcommand{\PP}{\mathbb{P}}
\newcommand{\TT}{\mathbb{T}}
\newcommand{\HH}{\mathbb{H}}
\newcommand{\kk}{\mathsf{k}}
\newcommand{\FF}{\mathbb{F}}
\newcommand{\cS}{\mathcal{S}}
\newcommand{\cT}{\mathcal{T}}
\newcommand{\ssC}{\mathsf{C}}
\newcommand{\sU}{\mathscr{U}}
\newcommand{\ol}{\overline}
\newcommand{\dd}{\mathsf{d}}

\newcommand{\id}{\operatorname{id}}
\newcommand{\Int}{\operatorname{Int}}
\newcommand{\cs}{\mathbin{\#}}
\newcommand{\Ab}{\mathsf{Ab}}
\newcommand{\Top}{\mathsf{Top}}
\newcommand{\Grp}{\mathsf{Grp}}
\newcommand{\Aut}{\operatorname{Aut}}
\newcommand{\SU}{\mathrm{SU}}
\newcommand{\SO}{\mathrm{SO}}
\renewcommand{\O}{\mathrm{O}}
\newcommand{\U}{\mathrm{U}}
\newcommand{\GL}{\mathrm{GL}}
\newcommand{\gl}{\mathfrak{gl}}
\newcommand{\Lie}{\operatorname{Lie}}
\newcommand{\dR}{\mathrm{dR}}
\newcommand{\PD}{\mathrm{PD}}



\title{Math 546: Manifolds\\ Homework due Friday Week 9}
%\author{Your Name}

\begin{document}
\maketitle

\noindent Problems taken from \emph{Introduction to Smooth Manifolds} are marked ISM $x$--$y$. Please review the syllabus for expectations and policies regarding homework.

\begin{prob}[ISM 18--7, \textsc{The Poincar\'e Duality Theorem}]
Let $M$ be an oriented smooth $n$-manifold. Define a map $\PD\colon \Omega^p(M)\to \Omega^{n-p}_c(M)^*$ by
\[
  \PD(\omega)(\eta) = \int_M \omega\wedge \eta.
\]
\begin{enumerate}[(a)]
\item Show that $\PD$ descends to a linear map (still denoted by the same symbols) $\PD\colon H_{\dR}^p(M)\to H^{n-p}_c(M)^*$.
\item Show that $\PD$ is an isomorphism for each $p$ [\emph{Hint}: Imitate the proof of the de Rham theorem. You will need Lemma 17.27 and you may assume that compactly supported de Rham cohomology has a Mayer--Vietoris sequence as in Problem 18--6 without proof.]
\end{enumerate}
\end{prob}

\begin{prob}[ISM 18--8]
Let $M$ be a compact smooth $n$-manfiold.
\begin{enumerate}[(a)]
\item Show that all de Rham groups of $M$ are finite-dimensional. [\emph{Hint}: for the orientable case, use Poincar\'e duality to show that $H^p_{\dR}(M)\cong H^p_{\dR}(M)^{**}$. (You may use the fact that an infinite-dimensional vector space is not isomorphic to its double dual.) For the nonorientable case, use Lemma 17.33.]
\item Show that if $M$ is orientable, then $\dim H^p_{\dR}(M) = \dim H^{n-p}_{\dR}(M)$ for all $p$.
\end{enumerate}
\end{prob}

\begin{prob}[ISM 18--9]
Let $M$ be a smooth $n$-manifold, all of whose de Rham groups are finite-dimensional. The \emph{Euler characteristic} of $M$ is the number
\[
  \chi(M) := \sum_{p=0}^n (-1)^p\dim H^p_{\dR}(M).
\]
Show that $\chi(M)$ is a homotopy invariant of $M$, and $\chi(M)=0$ when $M$ is compact, orientable, and odd-dimensional.
\end{prob}
\end{document}