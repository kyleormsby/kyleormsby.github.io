\documentclass[11pt,twoside]{amsart}
\usepackage{amssymb, amsmath, enumerate, palatino, hyperref}
\usepackage[normalem]{ulem}
\usepackage{fullpage}
\usepackage[T1]{fontenc}
\renewcommand{\labelitemi}{\guillemotright}
\usepackage{mathrsfs}


\theoremstyle{plain}
\newtheorem{prop}{Proposition}%[section]
\newtheorem{lemma}[prop]{Lemma}
\newtheorem{thm}[prop]{Theorem}
\newtheorem{obs}[prop]{Observation}
\newtheorem{app}[prop]{Application}
\newtheorem*{MainThm}{Main Theorem}
\newtheorem{cor}[prop]{Corollary}
\newtheorem{conj}[prop]{Conjecture}
\theoremstyle{remark}
\newtheorem{rmk}[prop]{Remark}
\newtheorem{prob}{Problem}
\newtheorem{bonus}[prop]{Bonus Problem}
\newtheorem{exc}{Exercise}
\theoremstyle{definition}
\newtheorem{ex}[prop]{Example}
\theoremstyle{definition}
\newtheorem{defn}[prop]{Definition}

\newcommand{\RR}{\mathbb{R}}
\newcommand{\ZZ}{\mathbb{Z}}
\newcommand{\CC}{\mathbb{C}}
\newcommand{\NN}{\mathbb{N}}
\newcommand{\QQ}{\mathbb{Q}}
\newcommand{\PP}{\mathbb{P}}

\newcommand{\cC}{\mathscr{C}}
\newcommand{\Ob}{\operatorname{Ob}}
\newcommand{\Mor}{\operatorname{Mor}}

\newcommand{\Set}{\operatorname{Set}}
\newcommand{\FinSet}{\operatorname{FinSet}}
\newcommand{\Vect}{\operatorname{Vect}}
\newcommand{\FinVect}{\operatorname{FinVect}}
\newcommand{\Mat}{\operatorname{Mat}}
\newcommand{\Gp}{\operatorname{Gp}}
\newcommand{\FinGp}{\operatorname{FinGp}}
\newcommand{\AbGp}{\operatorname{AbGp}}
\newcommand{\Ring}{\operatorname{Ring}}
\newcommand{\CommRing}{\operatorname{CommRing}}
\newcommand{\Field}{\operatorname{Field}}
\newcommand{\Top}{\operatorname{Top}}
\newcommand{\Aut}{\operatorname{Aut}}
\newcommand{\id}{\operatorname{id}}


\newcommand{\defeq}{:=}

\title{Math 113: Discrete Structures\\ Homework 13}
%\author{} %Remove the leading % and put your name here if using this .tex file as a template for your homework.

\begin{document}
\maketitle

\noindent
\textbf{Due:} Friday, February 27 at 10pm.
\bigskip

\begin{prob} In a class, 18 students like to play chess , 23 like to play soccer, 21 like baking and 17 like video games. The number of those who like to both play chess and soccer is 9. We also know that 7 students like chess and baking, 6 like chess and video games, 12 like soccer and baking, 9 like soccer and video games, and finally 12 students like baking and video games. There are 4 students who like chess, soccer and baking, 3 who like chess, soccer and video games, 5 who like chess, baking and video games, and 7 who like soccer, baking and video games. Finally, there are 3 students who like all four activities. In addition, we know what every student likes at least one of these activities. How many students are in the class (with explanation)?
\end{prob}

\begin{prob}
Use the principle of inclusion/exclusion to find how many numbers in
\[[200] = \{1,2,\ldots,200\}\] are multiples of $2$, $3$, $5$,
or~$7$? (Show your work.)
\end{prob}

\begin{prob}
Your Reed account requires a password that uses only the letters X, Y, and Z and must use each letter at least once. How many possible passwords of length 10 are there?
\end{prob}
\end{document}
