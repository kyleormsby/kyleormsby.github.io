\documentclass[11pt,twoside]{amsart}
\usepackage{amssymb, amsmath, enumerate, palatino, hyperref}
\usepackage[normalem]{ulem}
\usepackage{fullpage}
\usepackage[T1]{fontenc}
\renewcommand{\labelitemi}{\guillemotright}
\usepackage{mathrsfs}


\theoremstyle{plain}
\newtheorem{prop}{Proposition}%[section]
\newtheorem{lemma}[prop]{Lemma}
\newtheorem{thm}[prop]{Theorem}
\newtheorem{obs}[prop]{Observation}
\newtheorem{app}[prop]{Application}
\newtheorem*{MainThm}{Main Theorem}
\newtheorem{cor}[prop]{Corollary}
\newtheorem{conj}[prop]{Conjecture}
\theoremstyle{remark}
\newtheorem{rmk}[prop]{Remark}
\newtheorem{prob}{Problem}
\newtheorem{bonus}[prop]{Bonus Problem}
\newtheorem{exc}{Exercise}
\theoremstyle{definition}
\newtheorem{ex}[prop]{Example}
\theoremstyle{definition}
\newtheorem{defn}[prop]{Definition}

\newcommand{\RR}{\mathbb{R}}
\newcommand{\ZZ}{\mathbb{Z}}
\newcommand{\CC}{\mathbb{C}}
\newcommand{\NN}{\mathbb{N}}
\newcommand{\QQ}{\mathbb{Q}}
\newcommand{\PP}{\mathbb{P}}

\newcommand{\cC}{\mathscr{C}}
\newcommand{\Ob}{\operatorname{Ob}}
\newcommand{\Mor}{\operatorname{Mor}}

\newcommand{\Set}{\operatorname{Set}}
\newcommand{\FinSet}{\operatorname{FinSet}}
\newcommand{\Vect}{\operatorname{Vect}}
\newcommand{\FinVect}{\operatorname{FinVect}}
\newcommand{\Mat}{\operatorname{Mat}}
\newcommand{\Gp}{\operatorname{Gp}}
\newcommand{\FinGp}{\operatorname{FinGp}}
\newcommand{\AbGp}{\operatorname{AbGp}}
\newcommand{\Ring}{\operatorname{Ring}}
\newcommand{\CommRing}{\operatorname{CommRing}}
\newcommand{\Field}{\operatorname{Field}}
\newcommand{\Top}{\operatorname{Top}}
\newcommand{\Aut}{\operatorname{Aut}}
\newcommand{\id}{\operatorname{id}}


\newcommand{\defeq}{:=}

\title{Math 113: Discrete Structures\\ Homework 07}
%\author{} %Remove the leading % and put your name here if using this .tex file as a template for your homework.

\begin{document}
\maketitle

\noindent
\textbf{Due:} Friday, February 13 at 10pm.
\bigskip

Suppose we have an identity~$E=F$ where~$E$ and~$F$ are two algebraic
expressions that evaluate to the same integer (see the examples below).  A {\em
combinatorial} explanation for the identity~$E=F$ requires identifying both~$E$
and~$F$ as solutions to counting problems and explaining why these
counting problems should have the same solution.  As an example, we give a proof
of the identity
\[
  \binom{n}{k}=\binom{n-1}{k-1}+\binom{n-1}{k}
\]
in the case where~$n>k>0$.  (It is true for general~$n$ and~$k$, but we will
  skip these trivial cases.)

\begin{proof}
Let~$S=\left\{ 1,\dots,n
\right\}$.  The left-hand side counts the~$k$-subsets of~$S$.  Each~$k$-subset
of~$S$ is of exactly one of two types: (1) those that contain~$n$, and (2) those
that do not.  To find the number of~$k$-subsets of~$S$, we can just count
the numbers of each type and add.  A subset of size~$k$ containing~$n$, i.e, of
type (1), is the same thing as
a subset of~$\left\{ 1,\dots,n-1 \right\}$ of size~$k-1$ to which we then 
append~$n$.  Thus, there are~$\binom{n-1}{k-1}$ subsets of type (1).
A~$k$-subset of~$S$ that does not contain~$n$, i.e., of type (2), is the same as a subset
of~$\left\{ 1,\dots,n-1 \right\}$, and there are~$\binom{n-1}{k}$ of these.
\end{proof}
\medskip

\begin{prob}
Consider the identity
\[
  \binom{n}{k}-\binom{n-3}{k}=\binom{n-1}{k-1}+\binom{n-2}{k-1}+\binom{n-3}{k-1}
\]
for~$n\geq 3$ and~$n\geq k$.
\begin{enumerate}[(a)]
  \item\label{lhs} Suppose there is a set $S$ of $n$ people, and in that set, there are three special people $a$, $b$, and $c$.  What is the left-hand side of the identity counting in the context of $S$ and its three distinguished members?
  \item Provide a combinatorial proof of the identity by showing the thing you
    counted in part~(\ref{lhs}) can be counted a different way.
\end{enumerate}
\end{prob}

\begin{prob}
Give a combinatorial explanation of the following identity:
\[
  \binom{17}{5}=\binom{10}{0}\binom{7}{5}+\binom{10}{1}\binom{7}{4}+
  \binom{10}{2}\binom{7}{3}+\binom{10}{3}\binom{7}{2}+\binom{10}{4}\binom{7}{1}+
  \binom{10}{5}\binom{7}{0}.
\]
Hint: you might think about coloring the elements of a set.
\end{prob}

\end{document}
