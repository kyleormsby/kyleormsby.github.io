\documentclass[11pt,twoside]{amsart}
\usepackage{amssymb, amsmath, enumerate, palatino, hyperref}
\usepackage[normalem]{ulem}
\usepackage{fullpage}
\usepackage[T1]{fontenc}
\renewcommand{\labelitemi}{\guillemotright}
\usepackage{mathrsfs}


\theoremstyle{plain}
\newtheorem{prop}{Proposition}%[section]
\newtheorem{lemma}[prop]{Lemma}
\newtheorem{thm}[prop]{Theorem}
\newtheorem{obs}[prop]{Observation}
\newtheorem{app}[prop]{Application}
\newtheorem*{MainThm}{Main Theorem}
\newtheorem{cor}[prop]{Corollary}
\newtheorem{conj}[prop]{Conjecture}
\theoremstyle{remark}
\newtheorem{rmk}[prop]{Remark}
\newtheorem{prob}{Problem}
\newtheorem{bonus}[prop]{Bonus Problem}
\newtheorem{exc}{Exercise}
\theoremstyle{definition}
\newtheorem{ex}[prop]{Example}
\theoremstyle{definition}
\newtheorem{defn}[prop]{Definition}

\newcommand{\RR}{\mathbb{R}}
\newcommand{\ZZ}{\mathbb{Z}}
\newcommand{\CC}{\mathbb{C}}
\newcommand{\NN}{\mathbb{N}}
\newcommand{\QQ}{\mathbb{Q}}
\newcommand{\PP}{\mathbb{P}}

\newcommand{\cC}{\mathscr{C}}
\newcommand{\Ob}{\operatorname{Ob}}
\newcommand{\Mor}{\operatorname{Mor}}

\newcommand{\Set}{\operatorname{Set}}
\newcommand{\FinSet}{\operatorname{FinSet}}
\newcommand{\Vect}{\operatorname{Vect}}
\newcommand{\FinVect}{\operatorname{FinVect}}
\newcommand{\Mat}{\operatorname{Mat}}
\newcommand{\Gp}{\operatorname{Gp}}
\newcommand{\FinGp}{\operatorname{FinGp}}
\newcommand{\AbGp}{\operatorname{AbGp}}
\newcommand{\Ring}{\operatorname{Ring}}
\newcommand{\CommRing}{\operatorname{CommRing}}
\newcommand{\Field}{\operatorname{Field}}
\newcommand{\Top}{\operatorname{Top}}
\newcommand{\Aut}{\operatorname{Aut}}
\newcommand{\id}{\operatorname{id}}


\newcommand{\defeq}{:=}

\title{Math 113: Discrete Structures\\ Homework 08}
%\author{} %Remove the leading % and put your name here if using this .tex file as a template for your homework.

\begin{document}
\maketitle

\noindent
\textbf{Due:} Monday, February 16 at 10pm.
\bigskip

\begin{prob} 
Suppose that $f\colon A\to B$ is a surjective function.  Define a relation $\asymp_f$
on $A$ so that $a\asymp_f b$ if and only if $f(a)=f(b)$.
\begin{enumerate}[(a)]
  \item Prove that $\asymp_f$ is an equivalence relation.
  \item Determine the number of equivalence classes under $\asymp_f$.
\end{enumerate}
\end{prob}

\begin{prob}
Suppose that we are playing a game in which we roll three six-sided dice (with
sides labeled $1,2,\ldots,6$).  Declare two rolls equivalent if their sums
match.  (Formally, a roll can be thought of as an ordered 3-tuple~$(a,b,c)$
  where~$a,b,c\in\left\{ 1,\dots,6 \right\}$, and our relation
is~$(a,b,c)\sim(a',b',c')$ if and only if~$a+b+c=a'+b'+c')$.)
\begin{enumerate}[(a)]
  \item Prove that this is indeed an equivalence relation.
  \item Determine the number of equivalence classes. 
  \item Are all of the equivalence classes of the same size?
\end{enumerate}
\end{prob}

\noindent{\bf Template for proving a relation is an equivalence relation.}
\medskip

\noindent{\bf Theorem.} Define a relation~$\sim$ on a set~$A$ by blah, blah,
blah.  Then~$\sim$ is an equivalence relation.
\smallskip

\noindent{\bf Proof.} {\em Reflexivity.} For each~$a\in A$, we have~$a\sim
a$ since blah, blah, blah.  Therefore,~$\sim$ is reflexive.

\noindent{\em Symmetry.}  Suppose that~$a\sim b$.  Then, blah, blah, blah. It
follows that~$b\sim a$.  Therefore~$\sim$ is symmetric.

\noindent{\em Transitivity.}  Suppose that~$a\sim b$ and~$b\sim c$.
Since blah, blah, blah, it follows that~$a\sim c$.  Therefore,~$\sim$ is
transitive.

\noindent  Since~$\sim$ is reflexive, symmetric, and transitive, it follows
that~$\sim$ is an equivalence relation.
\hfill$\Box$

\end{document}
