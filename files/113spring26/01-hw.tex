\documentclass[11pt,twoside]{amsart}
\usepackage{amssymb, amsmath, enumerate, palatino, hyperref}
\usepackage[normalem]{ulem}
\usepackage{fullpage}
\usepackage[T1]{fontenc}
\renewcommand{\labelitemi}{\guillemotright}
\usepackage{mathrsfs}


\theoremstyle{plain}
\newtheorem{prop}{Proposition}%[section]
\newtheorem{lemma}[prop]{Lemma}
\newtheorem{thm}[prop]{Theorem}
\newtheorem{obs}[prop]{Observation}
\newtheorem{app}[prop]{Application}
\newtheorem*{MainThm}{Main Theorem}
\newtheorem{cor}[prop]{Corollary}
\newtheorem{conj}[prop]{Conjecture}
\theoremstyle{remark}
\newtheorem{rmk}[prop]{Remark}
\newtheorem{prob}{Problem}
\newtheorem{bonus}[prop]{Bonus Problem}
\newtheorem{exc}{Exercise}
\theoremstyle{definition}
\newtheorem{ex}[prop]{Example}
\theoremstyle{definition}
\newtheorem{defn}[prop]{Definition}

\newcommand{\RR}{\mathbb{R}}
\newcommand{\ZZ}{\mathbb{Z}}
\newcommand{\CC}{\mathbb{C}}
\newcommand{\NN}{\mathbb{N}}
\newcommand{\QQ}{\mathbb{Q}}
\newcommand{\PP}{\mathbb{P}}

\newcommand{\cC}{\mathscr{C}}
\newcommand{\Ob}{\operatorname{Ob}}
\newcommand{\Mor}{\operatorname{Mor}}

\newcommand{\Set}{\operatorname{Set}}
\newcommand{\FinSet}{\operatorname{FinSet}}
\newcommand{\Vect}{\operatorname{Vect}}
\newcommand{\FinVect}{\operatorname{FinVect}}
\newcommand{\Mat}{\operatorname{Mat}}
\newcommand{\Gp}{\operatorname{Gp}}
\newcommand{\FinGp}{\operatorname{FinGp}}
\newcommand{\AbGp}{\operatorname{AbGp}}
\newcommand{\Ring}{\operatorname{Ring}}
\newcommand{\CommRing}{\operatorname{CommRing}}
\newcommand{\Field}{\operatorname{Field}}
\newcommand{\Top}{\operatorname{Top}}

\newcommand{\Aut}{\operatorname{Aut}}

\newcommand{\defeq}{:=}

% tikz
\usepackage{tikz}
\usepackage{tikz-cd}
\usepackage{pgfplots}
\pgfplotsset{compat=1.13}
\usetikzlibrary{positioning}
\usetikzlibrary{trees}
\usetikzlibrary{decorations.pathmorphing}    
\usetikzlibrary{decorations.pathreplacing}
\usetikzlibrary{arrows.meta,calc}
\usetikzlibrary{bending}
\usetikzlibrary{decorations.markings,shapes.geometric}
\tikzset{->-/.style={decoration={markings, mark=at position #1 with
  {\arrow{>}}},postaction={decorate}}}
\tikzset{-|-/.style={decoration={markings, mark=at position #1 with
  {\arrow{stealth}}},postaction={decorate}}}
\tikzset{movearrow/.style 2 args ={
        decoration={markings,
	  mark= at position {#1} with {\arrow{#2}} ,
        },
        postaction={decorate}
    }
}
\tikzset{<--/.style={decoration={markings, mark=at position #1 with
  {\arrow{<}}},postaction={decorate}}}
\tikzstyle{ball} = [circle,shading=ball, ball color=black,
    minimum size=1mm,inner sep=1.3pt]
\tikzstyle{bball} = [circle,shading=ball, ball color=blue,
    minimum size=1mm,inner sep=1.3pt]
\tikzstyle{miniball} = [circle,shading=ball, ball color=black,
    minimum size=1mm,inner sep=0.5pt]
\tikzstyle{redminiball} = [circle,shading=ball, ball color=red,
    minimum size=1mm,inner sep=0.5pt]
\tikzstyle{mminiball} = [circle,shading=ball, ball color=black,
    minimum size=0.6mm,inner sep=0.1pt]


\title{Math 113: Discrete Structures\\ Homework 1}
%\author{} %Remove the leading % and put your name here if using this .tex file as a template for your homework.

\begin{document}
\maketitle

\textbf{Due:} Friday January 30 by 10pm.
\bigskip

THIS HOMEWORK WILL NOT BE GRADED FOR CONTENT (only for completion). We will practice submitting assignments to Gradescope. The rest of the assignments will be graded for content.
\bigskip

\begin{rmk}
This is homework on the content covered Monday of Week 1.  In general,
homework on content delivered during course meeting $n$ will be due by 10pm on the day
of course meeting $n+2$.  You are encouraged to start working on the homework
shortly after course meeting $n$ in order to have time to ask questions during
office hours and tutoring.
\end{rmk}

\begin{rmk}
Make sure to review the \emph{Homework} portion of the syllabus before writing
up your solutions!  For instance: \textbf{you will only receive full credit if
you provide full explanations}.  Also, your \textbf{solutions should consist
\emph{solely} of complete sentences}.  Simply providing the correct numerical
solution does not suffice.  See the \emph{Mathematical Writing} appendix of the
textbook for more tips.
\end{rmk}
\bigskip

\begin{prob}
  \begin{enumerate}[(a)]
    \item   There are 12 students in this class. If you all decided to shake hands with each other, how many handshakes would that be?
\item Again, there are 12 students. How many ways can I pair you up? This means breaking up the whole class into groups of two. How many ways can I form groups of three? 
  \end{enumerate}
\end{prob}

\end{document}
