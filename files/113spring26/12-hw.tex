\documentclass[11pt,twoside]{amsart}
\usepackage{amssymb, amsmath, enumerate, palatino, hyperref}
\usepackage[normalem]{ulem}
\usepackage{fullpage}
\usepackage[T1]{fontenc}
\renewcommand{\labelitemi}{\guillemotright}
\usepackage{mathrsfs}


\theoremstyle{plain}
\newtheorem{prop}{Proposition}%[section]
\newtheorem{lemma}[prop]{Lemma}
\newtheorem{thm}[prop]{Theorem}
\newtheorem{obs}[prop]{Observation}
\newtheorem{app}[prop]{Application}
\newtheorem*{MainThm}{Main Theorem}
\newtheorem{cor}[prop]{Corollary}
\newtheorem{conj}[prop]{Conjecture}
\theoremstyle{remark}
\newtheorem{rmk}[prop]{Remark}
\newtheorem{prob}{Problem}
\newtheorem{bonus}[prop]{Bonus Problem}
\newtheorem{exc}{Exercise}
\theoremstyle{definition}
\newtheorem{ex}[prop]{Example}
\theoremstyle{definition}
\newtheorem{defn}[prop]{Definition}

\newcommand{\RR}{\mathbb{R}}
\newcommand{\ZZ}{\mathbb{Z}}
\newcommand{\CC}{\mathbb{C}}
\newcommand{\NN}{\mathbb{N}}
\newcommand{\QQ}{\mathbb{Q}}
\newcommand{\PP}{\mathbb{P}}

\newcommand{\cC}{\mathscr{C}}
\newcommand{\Ob}{\operatorname{Ob}}
\newcommand{\Mor}{\operatorname{Mor}}

\newcommand{\Set}{\operatorname{Set}}
\newcommand{\FinSet}{\operatorname{FinSet}}
\newcommand{\Vect}{\operatorname{Vect}}
\newcommand{\FinVect}{\operatorname{FinVect}}
\newcommand{\Mat}{\operatorname{Mat}}
\newcommand{\Gp}{\operatorname{Gp}}
\newcommand{\FinGp}{\operatorname{FinGp}}
\newcommand{\AbGp}{\operatorname{AbGp}}
\newcommand{\Ring}{\operatorname{Ring}}
\newcommand{\CommRing}{\operatorname{CommRing}}
\newcommand{\Field}{\operatorname{Field}}
\newcommand{\Top}{\operatorname{Top}}
\newcommand{\Aut}{\operatorname{Aut}}
\newcommand{\id}{\operatorname{id}}


\newcommand{\defeq}{:=}

\title{Math 113: Discrete Structures\\ Homework 12}
%\author{} %Remove the leading % and put your name here if using this .tex file as a template for your homework.

\begin{document}
\maketitle

\noindent
\textbf{Due:} Wednesday, February 25 at 10pm.
\bigskip


\noindent{\bf Note.} See the next page for a model proof by induction.  Try to emulate it
in your own work.
\bigskip

\begin{prob}
Let $n\geq 1$. Use induction to prove that
\[
  \sum_{k=1}^n k^3=1+8+27+\dots + n^3=\frac{n^2(n+1)^2}{4}.
\]
\end{prob}

\begin{prob}
Suppose that $n$ lines in the plane are drawn in such a fashion that no two are
parallel and no three intersect in a common point.  Using induction, prove that the plane is
divided into precisely $\frac{n(n+1)}{2}+1$ regions by the lines. (\emph{Hint:} by playing with small examples of $n$, find out how many new regions are formed when you add a new line.)
\end{prob}

\newpage


\centerline{\bf A typical induction proof}
\vskip0.3in

\noindent{\bf Proposition.} For $n\ge 1$,
\[
1+2+\dots+n = \frac{n(n+1)}{2}.
\]
\smallskip

\begin{proof}
  We will prove this by induction.  First note that the statement holds when
  $n=1$:
  \[
  1 = \frac{1(1+1)}{2}.
  \]
  Next, suppose the statement holds for some $n\ge 1$:
  \[
  1+2+\dots+n=\frac{n(n+1)}{2}.
  \]
  It follows that
  \begin{eqnarray*}  
    1+2+\dots+n+(n+1)&=& \frac{n(n+1)}{2}+(n+1)\\
    &=& \frac{n(n+1)+2(n+1)}{2}\\[5pt]
    &=& \frac{(n+1)(n+2)}{2},
  \end{eqnarray*}
  and the result then holds for $n+1$, too.  Hence, the statement holds for all
  $n\geq1$ by induction.
\end{proof}

\noindent{\bf Note:}
\begin{itemize}
  \item The first sentence of the proof is obligatory.  The reader needs to
    know you are about to give a proof by induction.
  \item It is good to explicitly state your induction hypothesis.  In the above
    proof, it is the sentence starting ``Next, suppose \dots''  You are not
    claiming this statement is true!  Your argument will be that {\em if} this
    statement is true, then something good happens (namely, the statement also
      holds for the case~$n+1$).
    \item Notice the easy-to-follow linear arrangement of equations following
      ``It follows that''.  When you have a string of calculations, please try
      to use a similar form.
\end{itemize}

\end{document}
