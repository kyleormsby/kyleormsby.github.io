\documentclass[11pt,twoside]{amsart}
\usepackage{amssymb, amsmath, enumerate, palatino, hyperref}
\usepackage[normalem]{ulem}
\usepackage{fullpage}
\usepackage[T1]{fontenc}
\renewcommand{\labelitemi}{\guillemotright}
\usepackage{mathrsfs}


\theoremstyle{plain}
\newtheorem{prop}{Proposition}%[section]
\newtheorem{lemma}[prop]{Lemma}
\newtheorem{thm}[prop]{Theorem}
\newtheorem{obs}[prop]{Observation}
\newtheorem{app}[prop]{Application}
\newtheorem*{MainThm}{Main Theorem}
\newtheorem{cor}[prop]{Corollary}
\newtheorem{conj}[prop]{Conjecture}
\theoremstyle{remark}
\newtheorem{rmk}[prop]{Remark}
\newtheorem{prob}{Problem}
\newtheorem{bonus}[prop]{Bonus Problem}
\newtheorem{exc}{Exercise}
\theoremstyle{definition}
\newtheorem{ex}[prop]{Example}
\theoremstyle{definition}
\newtheorem{defn}[prop]{Definition}

\newcommand{\RR}{\mathbb{R}}
\newcommand{\ZZ}{\mathbb{Z}}
\newcommand{\CC}{\mathbb{C}}
\newcommand{\NN}{\mathbb{N}}
\newcommand{\QQ}{\mathbb{Q}}
\newcommand{\PP}{\mathbb{P}}

\newcommand{\cC}{\mathscr{C}}
\newcommand{\Ob}{\operatorname{Ob}}
\newcommand{\Mor}{\operatorname{Mor}}

\newcommand{\Set}{\operatorname{Set}}
\newcommand{\FinSet}{\operatorname{FinSet}}
\newcommand{\Vect}{\operatorname{Vect}}
\newcommand{\FinVect}{\operatorname{FinVect}}
\newcommand{\Mat}{\operatorname{Mat}}
\newcommand{\Gp}{\operatorname{Gp}}
\newcommand{\FinGp}{\operatorname{FinGp}}
\newcommand{\AbGp}{\operatorname{AbGp}}
\newcommand{\Ring}{\operatorname{Ring}}
\newcommand{\CommRing}{\operatorname{CommRing}}
\newcommand{\Field}{\operatorname{Field}}
\newcommand{\Top}{\operatorname{Top}}

\newcommand{\Aut}{\operatorname{Aut}}

\newcommand{\defeq}{:=}

\title{Math 113: Discrete Structures\\ Homework 05}
%\author{} %Remove the leading % and put your name here if using this .tex file as a template for your homework.

\begin{document}
\maketitle

\noindent
\textbf{Due:} Monday, February 9 at 10pm.
\bigskip

\noindent\emph{Instructions.} \begin{enumerate}
\item Copy and reread your solutions to the Homework 02, which was due on Monday, February 2 (problems below). 
\item Rewrite your solutions with your new knowledge on how to write solutions. 
\item Write a short explanation about the changes you made.
\end{enumerate}
\bigskip

\begin{prob}
At a Go tournament, there are eight players and four boards.  In how many ways can the players sit down to play if
\begin{enumerate}[(a)]
\item We count the order in which pairs of players are seated at the boards, but do not care which side each player sits on?  (Here A vs B, C vs D, E vs F, G vs H is the same as B vs A, C vs D, E vs F, G vs H but is different from G vs H, E vs F, C vs D, A vs B.)
\item we count who sits on which side of each board, but do not care about the ordering of the boards?  (In this version, A vs B, C vs D, E vs F, G vs H is different from B vs A, C vs D, E vs F, G vs H but is the same as G vs H, E vs F, C vs D, A vs B.)
\end{enumerate}
\end{prob}

\begin{prob}
In how many ways can King Arthur and his twelve knights (13 people, in all) sit down at the
legendary Round Table in Camelot?  (Since the table is round, we will not
consider rotations of a given seating arrangement as different.)
\end{prob}

\end{document}
