\documentclass[11pt,twoside]{amsart}
\usepackage{amssymb, amsmath, enumerate, palatino, hyperref}
\usepackage[normalem]{ulem}
\usepackage{fullpage}
\usepackage[T1]{fontenc}
\renewcommand{\labelitemi}{\guillemotright}
\usepackage{mathrsfs}

\usepackage{tikz}
\tikzstyle{ball} = [circle,shading=ball, ball color=black,
    minimum size=1mm,inner sep=1.3pt]

\theoremstyle{plain}
\newtheorem{prop}{Proposition}%[section]
\newtheorem{lemma}[prop]{Lemma}
\newtheorem{thm}[prop]{Theorem}
\newtheorem{obs}[prop]{Observation}
\newtheorem{app}[prop]{Application}
\newtheorem*{MainThm}{Main Theorem}
\newtheorem{cor}[prop]{Corollary}
\newtheorem{conj}[prop]{Conjecture}
\theoremstyle{remark}
\newtheorem{rmk}[prop]{Remark}
\newtheorem{prob}{Problem}
\newtheorem{bonus}[prop]{Bonus Problem}
\newtheorem{exc}{Exercise}
\theoremstyle{definition}
\newtheorem{ex}[prop]{Example}
\theoremstyle{definition}
\newtheorem{defn}[prop]{Definition}

\newcommand{\RR}{\mathbb{R}}
\newcommand{\ZZ}{\mathbb{Z}}
\newcommand{\CC}{\mathbb{C}}
\newcommand{\NN}{\mathbb{N}}
\newcommand{\QQ}{\mathbb{Q}}
\newcommand{\PP}{\mathbb{P}}

\newcommand{\cC}{\mathscr{C}}
\newcommand{\Ob}{\operatorname{Ob}}
\newcommand{\Mor}{\operatorname{Mor}}

\newcommand{\Set}{\operatorname{Set}}
\newcommand{\FinSet}{\operatorname{FinSet}}
\newcommand{\Vect}{\operatorname{Vect}}
\newcommand{\FinVect}{\operatorname{FinVect}}
\newcommand{\Mat}{\operatorname{Mat}}
\newcommand{\Gp}{\operatorname{Gp}}
\newcommand{\FinGp}{\operatorname{FinGp}}
\newcommand{\AbGp}{\operatorname{AbGp}}
\newcommand{\Ring}{\operatorname{Ring}}
\newcommand{\CommRing}{\operatorname{CommRing}}
\newcommand{\Field}{\operatorname{Field}}
\newcommand{\Top}{\operatorname{Top}}
\newcommand{\Aut}{\operatorname{Aut}}
\newcommand{\id}{\operatorname{id}}


\newcommand{\defeq}{:=}

\title{Math 113: Discrete Structures\\ Homework 15}
%\author{} %Remove the leading % and put your name here if using this .tex file as a template for your homework.

\begin{document}
\maketitle

\noindent
\textbf{Due:} Wednesday, March 4 at 10pm.
\bigskip

\noindent
This is homework on the content covered Monday of Week 6.  
\bigskip

\begin{prob}
  How many graphs are there with vertex set~$\left\{ 1,\dots,50 \right\}$?
  Graphs are considered to be equal if they have the same edge sets.  For
  instance, consider the case of graphs on the vertex set~$\left\{ 1,\dots,4
  \right\}$.  The following two graphs are different (e.g., the first has
  edge~$\left\{ 1,4 \right\}$ and the second does not):
  \begin{center}
    \begin{tikzpicture}[scale=0.5]
      \node[ball,label={below left:$1$}] at (0,0) (1) {};
      \node[ball,label={below right:$2$}] at (2,0) (2) {};
      \node[ball,label={above right:$3$}] at (2,2) (3) {};
      \node[ball,label={above left:$4$}] at (0,2) (4) {};
      \draw (1)--(2)--(3)--(4)--(1);
      \begin{scope}[xshift=8cm]
	\node[ball,label={below left:$1$}] at (0,0) (1) {};
	\node[ball,label={below right:$2$}] at (2,0) (2) {};
	\node[ball,label={above right:$4$}] at (2,2) (3) {};
	\node[ball,label={above left:$3$}] at (0,2) (4) {};
	\draw (1)--(2)--(3)--(4)--(1);
      \end{scope}
    \end{tikzpicture}
  \end{center}
  and the following are the same:
  \begin{center}
    \begin{tikzpicture}[scale=0.5]
      \node[ball,label={below left:$1$}] at (0,0) (1) {};
      \node[ball,label={below right:$2$}] at (2,0) (2) {};
      \node[ball,label={above right:$3$}] at (2,2) (3) {};
      \node[ball,label={above left:$4$}] at (0,2) (4) {};
      \draw (1)--(2)--(3)--(4)--(1);
      \begin{scope}[xshift=8cm]
	\node[ball,label={below left:$4$}] at (0,0) (1) {};
	\node[ball,label={below right:$1$}] at (2,0) (4) {};
	\node[ball,label={above right:$2$}] at (2,2) (3) {};
	\node[ball,label={above left:$3$}] at (0,2) (2) {};
	\draw (1)--(2)--(3)--(4)--(1);
      \end{scope}
      \node at (12,0) {.};
    \end{tikzpicture}
  \end{center}
  For this problem, we assume our graphs have no loops or multiple edges (i.e., each edge contains
  exactly two vertices).  Also, note that the graph with no edges (consisting solely of isolated
  vertices) counts as a graph.
\end{prob}

\begin{prob} At every party, one can find two people who know the same number of
  other people at the party.  (The property of ``knowing'' someone is assumed to
  be a symmetric relation but not reflexive.)  Restate this assertion as a
  question about graphs,
  and prove it. [Hint: if there are~$n$ vertices in a graph, what is the list of
  possible vertex degrees?]
\end{prob}
\end{document}
