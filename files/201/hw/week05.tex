\documentclass[11pt,twoside]{amsart}
\usepackage{amssymb, amsmath, enumerate, libertine, microtype, hyperref,tikz-cd}
\usepackage[normalem]{ulem}
\usepackage{fullpage}
\usepackage[T1]{fontenc}
\renewcommand{\labelitemi}{$\cdot$}
\usepackage{mathrsfs}
\usepackage{phaistos}


\theoremstyle{plain}
\newtheorem{prop}{Proposition}%[section]
\newtheorem{lemma}[prop]{Lemma}
\newtheorem{thm}[prop]{Theorem}
\newtheorem{obs}[prop]{Observation}
\newtheorem{app}[prop]{Application}
\newtheorem*{MainThm}{Main Theorem}
\newtheorem{cor}[prop]{Corollary}
\newtheorem{conj}[prop]{Conjecture}
\theoremstyle{remark}
\newtheorem{rmk}[prop]{Remark}
\newtheorem{prob}{Problem}
\newtheorem{bonus}[prop]{Bonus Problem}
\newtheorem{exc}{Exercise}
\theoremstyle{definition}
\newtheorem{ex}[prop]{Example}
\theoremstyle{definition}
\newtheorem{defn}[prop]{Definition}

\newcommand{\RR}{\mathbb{R}}
\newcommand{\R}{\mathbb{R}}
\newcommand{\ZZ}{\mathbb{Z}}
\newcommand{\Z}{\mathbb{Z}}
\newcommand{\CC}{\mathbb{C}}
\newcommand{\NN}{\mathbb{N}}
\newcommand{\QQ}{\mathbb{Q}}
\newcommand{\Q}{\mathbb{Q}}
\newcommand{\PP}{\mathbb{P}}
\newcommand{\kk}{\mathsf{k}}
\newcommand{\FF}{\mathbb{F}}
\newcommand{\cS}{\mathcal{S}}
\newcommand{\cT}{\mathcal{T}}
\newcommand{\ssC}{\mathsf{C}}

\newcommand{\id}{\operatorname{id}}
\newcommand{\Mat}{\mathsf{Mat}}
\newcommand{\spn}{\operatorname{span}}
\newcommand{\Hom}{\operatorname{Hom}}
\newcommand{\im}{\operatorname{im}}


\title{Math 201: Linear Algebra\\ Homework due Friday Week 5}
% uncomment the following line and add your name if you are using this as a template for solutions
% \author{Your Name}

\begin{document}
\maketitle


\begin{prob}
Find the coordinates of each given vector~$v$ with respect to the
    ordered list of linearly independent vectors~$B=\langle
      \beta_1,\dots,\beta_n \rangle$.  Show
      your work.
    \begin{enumerate}[(a)]
      \item $v=(11,-6)$, \ $B=\langle (1,2),(-2,3)\rangle$.
      \item $v=(11,-6)$, \  $B=\langle (1,0),(0,1)\rangle$.
      \item $v=x^2+7x-5$, \ $B=\langle 1, (x-1), (x-1)^2\rangle$.
      \item $v=x^2+7x-5$, \ $B=\langle 1, x, x^2, x^3 \rangle$. (Note:
    $x^3\in B$).
      \item
  \[
    v =\left(\begin{array}{cc}
        3&7\\
        8&11
    \end{array} \right),\qquad
    B = \left\langle
    \left(\begin{array}{cc}
        1&1\\
        2&1
    \end{array} \right),
    \left(\begin{array}{rr}
        -1&1\\
        -1&3
    \end{array} \right)
    \right\rangle.
  \]
    \end{enumerate}
\end{prob}
% uncomment the following and use the below proof environment to write your solution
% \begin{proof}[Solution]

% \end{proof}

\begin{prob}
Let~$A$ be an~$m\times n$ matrix with~$i,j$-th entry~$A_{ij}$.
    The {\em transpose} of~$A$, denoted~$A^{T}$, is the~$n\times m$ matrix
    with~$i,j$-th entry~$A_{ji}$: the~$i$-th row of~$A^{T}$ is the~$i$-th column
    of~$A$.  Thus, for example,
    \[
      \left(\begin{array}{cc}
    a&b\\
    c&d\\
    e&f
      \end{array} \right)^{T}
=
\left(\begin{array}{ccc}
    a&c&e\\
    b&d&f
\end{array} \right).
    \]
    A matrix~$A$ is {\em symmetric} if~$A^{T}=A$.  A matrix~$A$ is {\em skew-symmetric} if~$A^{T}=-A$.  A~$3\times 3$ skew-symmetric matrix has the form:
    \[
      \left(\begin{array}{rrr}
    0&a&b\\
    -a&0&c\\
    -b&-c&0
      \end{array} \right).
    \]
    Let~$W$ be the set of~$3\times 3$ skew-symmetric matrices over a
    field~$F$.
    \begin{enumerate}[(a)]
      \item Prove that~$W$ is a subspace of the vector space of all~$3\times 3$ matrices
  over~$F$.
      \item Give a basis for~$W$.
      \item What is~$\dim(W)$?
    \end{enumerate}
\end{prob}

\begin{prob}
Define the following matrix over the real numbers:
    \[
      M = 
      \left(\begin{array}{rrrr}
    -14 & 56 & 40 & 92 \\
    8 & -32 & -23 & -53 \\
    6 & -24 & -17 & -39 \\
    -1 & 4 & 3 & 7
      \end{array}\right).
    \]
    \begin{enumerate}[(a)]
      \item What is the reduced echelon form for~$M$? (You do not need to show
  your work for this.  Thinking a bit about your choices will save work.)
  \item Compute (i) a basis for the row space of~$M$ and (ii) a basis for
    the column space of~$M$ using the algorithm presented in class on
    Monday of Week~$4$. (Make sure you follow the algorithm precisely. The
    solution is then unique.)
    \end{enumerate}
\end{prob}

\begin{prob}
\begin{enumerate}[(a)]
     \item Prove that there exists a linear transformation $f\colon \R^2 \to \R^3$ such that $f(1,1)=(1,0,2)$ and $f(2,3)=(1,-1,4)$. What is $f(8,11)$?
     \item Is there a linear transformation $f\colon \R^3 \to \R^2$ such that $f(1,2,1)=(2,3)$, $f(3,1,4)=(6,2)$ and $f(7,-1,10)=(10,1)$? Explain your reasoning.
    \end{enumerate}
\end{prob}

\begin{prob}
Recall that $\Hom(V,W)$ denotes the vector space of linear transformations $V\to W$ under pointwise addition and scalar multiplication.\footnote{\emph{I.e.}, $(f+g)(v) = f(v)+g(v)$ and $(\lambda f)(v) = \lambda f(v)$ for all $f,g\in \Hom(V,W)$ and $\lambda\in F$.} When $W = F = F^1$, the vector space $V^* := \Hom(V,F)$ is the called the \emph{dual} of $V$. Suppose that $V$ has ordered basis $\langle v_1,\ldots,v_n\rangle$. Prove that $V^*$ has basis $\langle v_1^*,\ldots,v_n^*\rangle$ where $v_i^*$ satisfies
\[
  v_i^*(v_j) =
  \begin{cases}
    1&\text{if }i=j,\\
    0&\text{if }i\ne j.
  \end{cases}
\]
(Your solution should include a justification of why $v_i^*$ is a well-defined element of $V^*$.)
\end{prob}

\begin{prob}
For the following functions $f$:
   \begin{enumerate}[(i)]
    \item prove that $f$ is a linear transformation,
    \item find bases for $\ker(f)$ and $\im(f)$, and
    \item compute the nullity and the rank.
   \end{enumerate}
   \begin{enumerate}[(a)]
    \item $f\colon \R^3 \to \R^2$ defined by $f(x,y,z)=(x-y,2z)$. 
    \item $f \colon \R^2 \to \R^3$ defined by $f(x,y)=(x+y, 0, 2x-y)$. 
    \item $f\colon \RR[x]_{\le 2} \to \RR[x]_{\le 3}$ defined by $f(p(x))=x\cdot p(x)+p'(x)$.
  Here, $p'(x)$ denotes the usual derivative from one-variable calculus.
   \end{enumerate}
    (Recall that $F[x]_{\le n}$ denotes the vector space of polynomials with
    coefficients in $F$ of degree less than or equal to $n$.  One basis for it is
  $\left\{ 1,x,x^2,\dots,x^n \right\}$, and hence, it has dimension~$n+1$.)
\end{prob}

\end{document}