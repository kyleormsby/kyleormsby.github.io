\documentclass[11pt,twoside]{amsart}
\usepackage{amssymb, amsmath, enumerate, libertine, microtype, hyperref,tikz-cd}
\usepackage[normalem]{ulem}
\usepackage{fullpage}
\usepackage[T1]{fontenc}
\renewcommand{\labelitemi}{$\cdot$}
\usepackage{mathrsfs}
\usepackage{phaistos}


\theoremstyle{plain}
\newtheorem{prop}{Proposition}%[section]
\newtheorem{lemma}[prop]{Lemma}
\newtheorem{thm}[prop]{Theorem}
\newtheorem{obs}[prop]{Observation}
\newtheorem{app}[prop]{Application}
\newtheorem*{MainThm}{Main Theorem}
\newtheorem{cor}[prop]{Corollary}
\newtheorem{conj}[prop]{Conjecture}
\theoremstyle{remark}
\newtheorem{rmk}[prop]{Remark}
\newtheorem{prob}{Problem}
\newtheorem{bonus}[prop]{Bonus Problem}
\newtheorem{exc}{Exercise}
\theoremstyle{definition}
\newtheorem{ex}[prop]{Example}
\theoremstyle{definition}
\newtheorem{defn}[prop]{Definition}

\newcommand{\RR}{\mathbb{R}}
\newcommand{\R}{\mathbb{R}}
\newcommand{\ZZ}{\mathbb{Z}}
\newcommand{\Z}{\mathbb{Z}}
\newcommand{\CC}{\mathbb{C}}
\newcommand{\NN}{\mathbb{N}}
\newcommand{\QQ}{\mathbb{Q}}
\newcommand{\Q}{\mathbb{Q}}
\newcommand{\PP}{\mathbb{P}}
\newcommand{\kk}{\mathsf{k}}
\newcommand{\FF}{\mathbb{F}}
\newcommand{\cS}{\mathcal{S}}
\newcommand{\cT}{\mathcal{T}}
\newcommand{\ssC}{\mathsf{C}}

\newcommand{\id}{\operatorname{id}}
\newcommand{\Mat}{\mathsf{Mat}}
\newcommand{\spn}{\operatorname{span}}
\newcommand{\Hom}{\operatorname{Hom}}
\newcommand{\im}{\operatorname{im}}
\newcommand{\ev}{\operatorname{ev}}
\newcommand{\tr}{\operatorname{tr}}


\title{Math 201: Linear Algebra\\ Homework due Friday Week 7}
% uncomment the following line and add your name if you are using this as a template for solutions
% \author{Your Name}

\begin{document}
\maketitle

\begin{prob}
Suppose that $A$ is an invertible square matrix and $P$ and $Q$ are square matrices such that
\[
  PAQ = I.
\]
Prove that
\[
  A^{-1} = QP.
\]
\end{prob}

\begin{prob}
Square matrices are called \emph{similar} when they represent the same linear transformation, each with respect to the same starting and ending bases. In other words, $A\in F^{n\times n}$ is similar to $B\in F^{n\times n}$ if and only if there are ordered bases $\alpha$ and $\beta$ of $F^n$ and a linear transformation $f\colon F^n\to F^n$ such that
\[
  A = A_\alpha^\alpha(f)\qquad\text{and}\qquad B = A_{\beta}^\beta(f).
\]
\begin{enumerate}[(a)]
\item Prove that $A, B\in F^{n\times n}$ are similar if and only if there is an invertible matrix $P\in F^{n\times n}$ such that $A = P^{-1}BP$.
\item Prove that similarity is an equivalence relation on $F^{n\times n}$.
\item Show that if $A$ is similar to $B$, then $A^k$ is similar to $B^k$ for all natural numbers $k$.
\end{enumerate}
\end{prob}

\begin{prob}
Suppose that, with respect to $\alpha = \mathcal E_2$ and $\beta = ((1,1),(1,-2))$, the linear transformation $t\colon \RR^2\to \RR^2$ is represented by the matrix
\[
  A_\alpha^\beta(t) =
  \begin{pmatrix}
    1&2 \\ 3&4
  \end{pmatrix}.
\]
Use change-of-basis matrices to represent $t$ with respect to the following pairs of bases:
\begin{enumerate}[(a)]
\item $\delta = ((0,1),(1,1))$, $\gamma = ((-1,0),(2,1))$,
\item $\varepsilon = ((1,2),(1,0))$, $\zeta = ((1,2),(2,1))$.
\end{enumerate}
\end{prob}

\begin{prob}
Suppose $f\colon V\to W$ is a linear transformation.  Define $f^*\colon W^*\to V^*$ by the rule
\[
  \phi\longmapsto f^*(\phi) = \phi\circ f.
\]
(Recall that $\phi$ is a linear map $W\to F$, so $\phi\circ f$ makes sense as a linear map $V\to F$, \emph{i.e.}, an element of $V^*$.)
\begin{enumerate}[(a)]
\item Prove that $f^*$ is a linear transformation.
\item Suppose $g\colon U\to V$ is a linear transformation.  Prove that $(f\circ g)^* = g^*\circ f^*$.  Also verify that $\id_V^* = \id_{V^*}$.\footnote{This makes $(~)^*$ a \emph{constravariant functor}, but you needn't know what that means.}
\item Suppose that $V$ and $W$ have ordered bases $\langle v_1,\ldots,v_n\rangle$ and $\langle w_1,\ldots,w_m\rangle$, respectively, and suppose that $f$ has matrix $A\in\Mat_{m\times n}(F)$ with respect to these ordered bases.  Prove that the matrix of $f^*$ relative to $\langle w_1^*,\ldots,w_m^*\rangle$ and $\langle v_1^*,\ldots,v_n^*\rangle$ is the \emph{transpose} of $A$, \emph{i.e.}, $A^\top$ with
\[
  (A^\top)_{ij} = A_{ji}.
\]
\item Use your work from (b) and (c) to write a \textbf{short}, non-computational proof that $(AB)^\top = B^\top A^\top$.
\end{enumerate}
\end{prob}

\begin{prob}
The {\em trace} of an~$n\times n$ matrix~$A$ is the sum of its diagonal
    elements:
    \[
      \tr(A)=\sum_{i=1}^nA_{ii}.
    \]
    \begin{enumerate}[(a)]
      \item If~$A$ and~$B$ are~$n\times n$ matrices, prove
  that~$\tr(AB)=\tr(BA)$. (You could use the definition of matrix multiplication and
    summation notation in your proof, or there is a slick proof that uses duals and your work from the previous problem --- how does $\tr(A)$ compare to $\tr(A^\top)$?)
      \item If~$P$ is an invertible~$n\times n$ matrix, prove
  that~$\tr(PAP^{-1})=\tr(A)$.
      \item Consider the following ordered basis
  for~$\mathcal{M}_{2\times 2}$:
  \[
\alpha=
\left(
\left(\begin{array}{cc}
    1&0\\0&0
\end{array} \right),\quad
\left(\begin{array}{cc}
    0&1\\0&0
\end{array} \right),\quad
\left(\begin{array}{cc}
    0&0\\1&0
\end{array} \right),\quad
\left(\begin{array}{cc}
    0&0\\0&1
\end{array} \right)
\right).
  \]
    \end{enumerate}
A rote check reveals that $\tr$ is a linear transformation. Assuming this fact, compute the matrix representing the trace
function~$\tr\colon F^{2\times 2}\to F$ with respect to~$\alpha$ for the
domain and with respect to the basis~$\left\{ 1 \right\}$ for the codomain.
\end{prob}

\end{document}