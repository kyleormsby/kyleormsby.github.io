\documentclass[11pt,twoside]{amsart}
\usepackage{amssymb, amsmath, enumerate, libertine, microtype, hyperref,tikz-cd}
\usepackage[normalem]{ulem}
\usepackage{fullpage}
\usepackage[T1]{fontenc}
\renewcommand{\labelitemi}{$\cdot$}
\usepackage{mathrsfs}
\usepackage{phaistos}


\theoremstyle{plain}
\newtheorem{prop}{Proposition}%[section]
\newtheorem{lemma}[prop]{Lemma}
\newtheorem{thm}[prop]{Theorem}
\newtheorem{obs}[prop]{Observation}
\newtheorem{app}[prop]{Application}
\newtheorem*{MainThm}{Main Theorem}
\newtheorem{cor}[prop]{Corollary}
\newtheorem{conj}[prop]{Conjecture}
\theoremstyle{remark}
\newtheorem{rmk}[prop]{Remark}
\newtheorem{prob}{Problem}
\newtheorem{bonus}[prop]{Bonus Problem}
\newtheorem{exc}{Exercise}
\theoremstyle{definition}
\newtheorem{ex}[prop]{Example}
\theoremstyle{definition}
\newtheorem{defn}[prop]{Definition}

\newcommand{\RR}{\mathbb{R}}
\newcommand{\R}{\mathbb{R}}
\newcommand{\ZZ}{\mathbb{Z}}
\newcommand{\Z}{\mathbb{Z}}
\newcommand{\CC}{\mathbb{C}}
\newcommand{\NN}{\mathbb{N}}
\newcommand{\QQ}{\mathbb{Q}}
\newcommand{\Q}{\mathbb{Q}}
\newcommand{\PP}{\mathbb{P}}
\newcommand{\kk}{\mathsf{k}}
\newcommand{\FF}{\mathbb{F}}
\newcommand{\cS}{\mathcal{S}}
\newcommand{\cT}{\mathcal{T}}
\newcommand{\ssC}{\mathsf{C}}

\newcommand{\id}{\operatorname{id}}
\newcommand{\Mat}{\mathsf{Mat}}
\newcommand{\spn}{\operatorname{span}}

\title{Math 201: Linear Algebra\\ Homework due Friday Week 3}
% uncomment the following line and add your name if you are using this as a template for solutions
% \author{Your Name}

\begin{document}
\maketitle


\begin{prob}
Prove that the the following sets and operations do not form vectors spaces.  As usual, to disprove something, you need to provide a concrete counterexample, ideally as simple as possible.   
\begin{enumerate}[(a)]
\item $V=\RR^2$, with 
\[\begin{pmatrix}
x_1\\
y_1
\end{pmatrix}
+
\begin{pmatrix}
x_2\\
y_2
\end{pmatrix}
=
\begin{pmatrix}
x_1+x_2\\
y_1y_2
\end{pmatrix}\quad \mbox{and} \quad
r\cdot\begin{pmatrix}
x_1\\
y_1
\end{pmatrix}
=
\begin{pmatrix}
rx_1\\
y_1
\end{pmatrix}.
\]
\item $V=\RR^2$, with 
\[\begin{pmatrix}
x_1\\
y_1
\end{pmatrix}
+
\begin{pmatrix}
x_2\\
y_2
\end{pmatrix}
=
\begin{pmatrix}
x_1+x_2\\
y_1+y_2
\end{pmatrix}\quad \mbox{and} \quad
r\cdot\begin{pmatrix}
x_1\\
y_1
\end{pmatrix}
=
\begin{pmatrix}
rx_1\\
0
\end{pmatrix}.
\]
\item $V = \{(x,y)\in\RR^2:x+2y=3\}$ with the usual addition and scalar multiplication for vectors in~$\RR^2$.
\end{enumerate}
\end{prob}

\begin{prob}
Here are two templates for showing a subset~$W$ of a vector space~$V$ over a field~$F$ is a subspace:
\medskip

{\bf Proof 1.} First note that~$0\in W$ since \underline{\hspace{2cm}}. Hence, $W\neq\emptyset$. Next, suppose that~$u,v\in W$. Then \underline{\hspace{2cm}}. Hence,~$u+v\in W$.  Now suppose~$\lambda\in F$ and~$w\in W$.  Then \underline{\hspace{2cm}}.  Therefore,~$\lambda w\in W$.   \hfill$\Box$
\medskip

{\bf Proof 2.} First note that~$0\in W$ since \underline{\hspace{2cm}}. Hence, $W\neq\emptyset$. Next, suppose that~$\lambda\in F$ and~$u,v\in W$. Then \underline{\hspace{2cm}}. Hence,~$\lambda u+v\in W$. \hfill$\Box$

Use one of these two templates for each of the following exercises.
\begin{enumerate}[(a)]
\item Show that~$W=\{(x,y,z)\in\RR^3:3x+2y-z=0\}$ is a subspace of~$\RR^3$.
\item Show that the set $W=\left\{ f\colon\RR\to\RR: f(t)=f(-t) \text{ for all~$t\in \RR$} \right\}$ is a subspace of the vector space of real-valued functions of one variable. (Hint: you will need to carefully use the definitions given in Example~1.12, p.~90, of the text.)
\end{enumerate}
\end{prob}

\begin{prob}
In each of the following:
    \begin{itemize}
      \item Determine whether the given vector~$v$ is in the span of the
  set~$S$ by creating a relevant system of linear equations in the usual form
  \begin{align*}
    a_{11}x_1+&\dots+a_{1n}x_n = b_1\\
    &\quad\vdots\\
    a_{m1}x_1+&\dots+a_{mn}x_n = b_m\\
  \end{align*}
  and then row reducing the corresponding augmented matrix for the system.
      \item If~$v$ is in the span of~$S$, then explicitly write~$v$ as a linear
  combination of the vectors in~$S$.
    \end{itemize} 
    Assume we are working over the field $\Q$ of rational numbers.
    \begin{enumerate}[(a)]
      \item $v=(0,-1,-6)$, $S=\left\{ (1,0,-1), (2,1,3), (4,2,5) \right\}$.
      \item $v=(1,2,4)$, $S=\left\{ (1,4,7),(2,5,8),(3,6,9) \right\}$.
      \item $v=x^3-13x^2+7x+27$, $S=\left\{ x^3+3x^2-2, x^3+x^2+4x+1,2x^2+x+4 \right\}$.
      \item $v=\left(\begin{array}{rrr}9&12\\10& 9 \end{array}\right)$, 
  $S=\left\{ 
    \left(\begin{array}{rrr}1&1\\1&-2 \end{array}\right),
    \left(\begin{array}{rrr}-1&2\\1& 2 \end{array}\right),
    \left(\begin{array}{rrr}2&4\\3& 5 \end{array}\right)
  \right\}$.
    \end{enumerate}
\end{prob}
% uncomment the following and use the below proof environment to write your solution
% \begin{proof}[Solution]

% \end{proof}

\begin{prob}
Determine whether the following sets are linearly dependent or linearly independent.
  \begin{enumerate}[(a)]
   \item $\{x^3+2x^2,-x^2+3x+1,x^3-x^2+2x-1\}$ in $\RR[x]$.
   \item $\{(1,-1,2),(1,-2,1),(1,1,4)\}$ in $\R^3$.
   \item $\{(1,1,0),(1,0,1),(0,1,1)\}$ in $\R^3$.
   \item $\{(1,1,0),(1,0,1),(0,1,1)\}$ in $(\Z/2\Z)^3$ (where $\Z/2\Z$ is the field with two elements).
  \end{enumerate}
\end{prob}

\begin{prob}
Let $F$ be a finite field with $|F|=q$.\footnote{Here $|F|$ denotes the \emph{cardinality} of $F$, \emph{i.e.}, the number of elements in the set $F$. In abstract algebra, you will discover that $q$ must be of the form $p^n$ for some prime $p$.} Let $S = \{u_1,\ldots,u_n\}$ be a set of linearly independent vectors in an $F$-vector space.  Determine the cardinality of $\spn S$.
\end{prob}

\begin{prob}
Fix a field $F$. Using only the material we have developed in class thus far, show that a set
\[
  S = \{(a,b),(c,d)\}\subseteq F^2
\]
is linearly independent if and only if
\[
  ad-bc \ne 0.
\]
\end{prob}

\end{document}