\documentclass[11pt,twoside]{amsart}
\usepackage{amssymb, amsmath, enumerate, libertine, microtype, hyperref,tikz-cd}
\usepackage[normalem]{ulem}
\usepackage{fullpage}
\usepackage[T1]{fontenc}
\renewcommand{\labelitemi}{$\cdot$}
\usepackage{mathrsfs}
\usepackage{phaistos}


\theoremstyle{plain}
\newtheorem{prop}{Proposition}%[section]
\newtheorem{lemma}[prop]{Lemma}
\newtheorem{thm}[prop]{Theorem}
\newtheorem{obs}[prop]{Observation}
\newtheorem{app}[prop]{Application}
\newtheorem*{MainThm}{Main Theorem}
\newtheorem{cor}[prop]{Corollary}
\newtheorem{conj}[prop]{Conjecture}
\theoremstyle{remark}
\newtheorem{rmk}[prop]{Remark}
\newtheorem{prob}{Problem}
\newtheorem{bonus}[prop]{Bonus Problem}
\newtheorem{exc}{Exercise}
\theoremstyle{definition}
\newtheorem{ex}[prop]{Example}
\theoremstyle{definition}
\newtheorem{defn}[prop]{Definition}

\newcommand{\RR}{\mathbb{R}}
\newcommand{\ZZ}{\mathbb{Z}}
\newcommand{\CC}{\mathbb{C}}
\newcommand{\NN}{\mathbb{N}}
\newcommand{\QQ}{\mathbb{Q}}
\newcommand{\PP}{\mathbb{P}}
\newcommand{\kk}{\mathsf{k}}
\newcommand{\FF}{\mathbb{F}}
\newcommand{\cS}{\mathcal{S}}
\newcommand{\cT}{\mathcal{T}}
\newcommand{\ssC}{\mathsf{C}}

\newcommand{\id}{\operatorname{id}}
\newcommand{\Mat}{\mathsf{Mat}}
\newcommand{\eval}{\operatorname{eval}}
\newcommand{\colim}{\operatorname{colim}}

\title{Math 411: Topics in Advanced Analysis\\ Homework due Wednesday Week 11}
% uncomment the following line and add your name if you are using this as a template for solutions
% \author{Your Name}

\begin{document}
\maketitle

\begin{prob}
Suppose $A$ is an LCA group and $f\in L^1_{bc}(A)$. Recall that for $s\in A$, $L_sf\in L^1_{bc}(A)$ is the function given by
\[
  L_sf(x) = f(s^{-1}x).
\]
Find a nice formula for $\widehat{L_sf}$ and prove that it is true.
\end{prob}

\begin{prob}
Let $S = [a_1,b_1]\times\cdots\times[a_d,b_d]\subseteq \RR^d$ and let $1_S$ denote its characteristic function. Compute $\widehat{1_S}$.
\end{prob}

\begin{prob}
Suppose $M\colon \RR^d\to \RR^d$ is a linear transformation, and $f\in L^1(\RR^d)$. Prove that $\widehat{f\circ M}$ is given by
\[
  \widehat{f\circ M}(\xi) = \frac{1}{|\det M|}\hat f(M^{-\top}\xi)
\]
where $M^{-\top}$ denotes the inverse transpose of $M$.
\end{prob}

\begin{prob}
Suppose $\Delta\subseteq \RR^2$ is a (solid) triangle with vertices in $\ZZ^2$. Show that the following properties are equivalent:
\begin{enumerate}[(a)]
\item $\Delta$ contains no other points of $\ZZ^2$ (in its interior or boundary),
\item the area of $\Delta$ is $1/2$,
\item $\Delta$ is \textbf{unimodular}, \emph{i.e.}, $v_3-v_1$ and $v_2-v_1$ form a basis for $\ZZ^2$.
\end{enumerate}
\emph{Hint}: You might want to ``double'' the triangle to form a parallelogram.
\end{prob}





\end{document}