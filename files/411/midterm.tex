\documentclass[11pt,twoside]{amsart}
\usepackage{amssymb, amsmath, enumerate, libertine, hyperref, fullpage,tikz,microtype}
\usetikzlibrary{decorations.markings}
\tikzset{-|-/.style={decoration={markings, mark=at position #1 with
  {\arrow{stealth}}},postaction={decorate}}}
\usepackage[normalem]{ulem}
\usepackage{fullpage}
\usepackage[T1]{fontenc}
\renewcommand{\labelitemi}{$\cdot$}
\usepackage{mathrsfs,linearb}
\usepackage{systeme}


\theoremstyle{plain}
\newtheorem{prop}{Proposition}%[section]
\newtheorem{lemma}[prop]{Lemma}
\newtheorem{thm}[prop]{Theorem}
\newtheorem{obs}[prop]{Observation}
\newtheorem{app}[prop]{Application}
\newtheorem*{MainThm}{Main Theorem}
\newtheorem{cor}[prop]{Corollary}
\newtheorem{conj}[prop]{Conjecture}
\theoremstyle{remark}
\newtheorem{rmk}[prop]{Remark}
\newtheorem{prob}{Problem}
\newtheorem{refl}{Reflection}
\newtheorem{question}[prop]{Question}
\newtheorem{bonus}[prop]{Bonus Problem}
\newtheorem{exc}{Exercise}
\theoremstyle{definition}
\newtheorem{ex}[prop]{Example}
\theoremstyle{definition}
\newtheorem{defn}[prop]{Definition}

\newcommand{\RR}{\mathbb{R}}
\newcommand{\R}{\RR}
\newcommand{\ZZ}{\mathbb{Z}}
\newcommand{\CC}{\mathbb{C}}
\newcommand{\NN}{\mathbb{N}}
\newcommand{\QQ}{\mathbb{Q}}
\newcommand{\PP}{\mathbb{P}}

\newcommand{\cC}{\mathscr{C}}
\newcommand{\Ob}{\operatorname{Ob}}
\newcommand{\Mor}{\operatorname{Mor}}
\newcommand{\Bb}{\mathscr{B}}

\newcommand{\Set}{\operatorname{Set}}
\newcommand{\FinSet}{\operatorname{FinSet}}
\newcommand{\Vect}{\operatorname{Vect}}
\newcommand{\FinVect}{\operatorname{FinVect}}
\newcommand{\Mat}{\operatorname{Mat}}
\newcommand{\Gp}{\operatorname{Gp}}
\newcommand{\FinGp}{\operatorname{FinGp}}
\newcommand{\AbGp}{\operatorname{AbGp}}
\newcommand{\Ring}{\operatorname{Ring}}
\newcommand{\CommRing}{\operatorname{CommRing}}
\newcommand{\Field}{\operatorname{Field}}
\newcommand{\Top}{\operatorname{Top}}
\newcommand{\ul}[1]{\underline{#1}}

\title{Math 411: Midterm Exam}

\begin{document}
\maketitle

\subsection*{Instructions}
This is an open book, open notes, open Internet, closed collaboration, closed ``AI'' take-home exam. It is distributed Monday 10 March at 7\textsc{a.m.}~and is due Wednesday 12 March at 10\textsc{p.m.}~via Gradescope. The recommended exam period is 180 minutes, but this is only a suggestion. You are welcome --- even encouraged --- to read the problems early and then sit down to `properly' write solutions later. The work that you turn in may be handwritten or typed, and all of it must be your own. You may use a computer and/or the Internet only for the following purposes:
\begin{itemize}
\item Accessing course materials from the course website or Zulip;
\item Looking at reference texts or your own notes;
\item Typesetting or looking up how to typeset something in \LaTeX;
\item Checking arithmetic and other computations with a calculator or computer algebra system.
\end{itemize}
\textbf{The Honor Principle prevails.}

\bigskip

\noindent Keep the following in mind as you work on the exam:
\begin{itemize}
    \setlength{\itemsep}{0.3\baselineskip}
  \item \textbf{Unless explicitly instructed otherwise, you must show your work and
justify your claims in order to receive full credit.} 
  \item The solutions you submit should not contain scratch work. Do your
    scratch work on separate sheets of paper which you do not submit.
\end{itemize}
\bigskip

\noindent Your completed exam must be uploaded to Gradescope as a PDF by
10\textsc{p.m.}~Pacific time on Wednesday 12 March.  \textbf{Given the flexibility of this exam format, I reserve the right to not accept late submissions.}  If something might prevent you from turning the exam in on time, please contact me ASAP.

\bigskip
\noindent Problems will be scored in a manner approximately proportional to the amount of work each requires.  Exams will be returned via Gradescope, and I will supply exam solutions after all exams are graded upon request.

\bigskip

\begin{prob}
Suppose $f\colon \RR\to \CC$ is $1/2$-periodic, so that $f(x+1/2) = f(x)$ for all $x\in \RR$. Show that $\hat f(n) = 0$ for all odd $n$.
\end{prob}

\begin{prob}
Let $f\colon [a,b]\to \CC$ be a continuous function. Show that for any $\varepsilon>0$ there exists a polynomial $P\in \CC[x]$ such that
\[
  \sup_{x\in [a,b]}|f(x)-P(x)|<\varepsilon
\]
This is called the \emph{Weierstrass approximation theorem}. [\emph{Hint}: A corollary from the lecture of 5 February implies that continuous functions on $S^1$ may be uniformly approximated by trigonometric polynomials. You may want to also show that that $e^{2\pi ix}$ can be uniformly approximated by polynomials on any interval.]
\end{prob}

\begin{prob}
Recall that for $s>1$,
\[
  \zeta(s) := \sum_{n\ge 1} \frac{1}{n^s}.
\]
\begin{enumerate}[(a)]
\item Fix a real number $t>0$ and let
\[
  f_t(x) = \frac{t}{\pi(x^2+t^2)}.
\]
Prove that
\[
  \widehat{f_t}(\gamma) = e^{-2\pi t|\gamma|}.
\]
\item Prove that
\[
  \frac 1\pi \sum_{n\in \ZZ} \frac{t}{t^2+n^2} = \sum_{n\in \ZZ} e^{-2\pi t|n|}.
\]
\item Prove that the following identity holds for $0<t<1$:
\[
  \frac 1\pi \sum_{n\in \ZZ}\frac{t}{t^2+n^2} = \frac{1}{\pi t} + \frac{2}{\pi}\sum_{m\ge 1} (-1)^{m+1}\zeta(2m)t^{2m-1}.
\]
\item Prove that
\[
  \sum_{n\in \ZZ}e^{-2\pi t|n|} = \frac{2}{1-e^{-2\pi t}}-1.
\]
\item The \emph{Bernoulli numbers} $B_k$ are defined by the Taylor series for $z/(e^z-1)$ in the sense that
\[
  \frac{z}{e^z-1} = 1-\frac z2 + \sum_{m\ge 1}\frac{B_{2m}}{(2m)!}z^{2m}.
\]
Use the above work to deduce that
\[
  2\zeta(2m) = (-1)^{m+1}\frac{(2\pi)^{2m}}{(2m)!}B_{2m}.
\]
(You may \emph{not} cite or use the results about $\zeta$ we proved in class or in homework.)
\end{enumerate}
\end{prob}

\end{document}
