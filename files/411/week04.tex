\documentclass[11pt,twoside]{amsart}
\usepackage{amssymb, amsmath, enumerate, libertine, microtype, hyperref,tikz-cd}
\usepackage[normalem]{ulem}
\usepackage{fullpage}
\usepackage[T1]{fontenc}
\renewcommand{\labelitemi}{$\cdot$}
\usepackage{mathrsfs}
\usepackage{phaistos}


\theoremstyle{plain}
\newtheorem{prop}{Proposition}%[section]
\newtheorem{lemma}[prop]{Lemma}
\newtheorem{thm}[prop]{Theorem}
\newtheorem{obs}[prop]{Observation}
\newtheorem{app}[prop]{Application}
\newtheorem*{MainThm}{Main Theorem}
\newtheorem{cor}[prop]{Corollary}
\newtheorem{conj}[prop]{Conjecture}
\theoremstyle{remark}
\newtheorem{rmk}[prop]{Remark}
\newtheorem{prob}{Problem}
\newtheorem{bonus}[prop]{Bonus Problem}
\newtheorem{exc}{Exercise}
\theoremstyle{definition}
\newtheorem{ex}[prop]{Example}
\theoremstyle{definition}
\newtheorem{defn}[prop]{Definition}

\newcommand{\RR}{\mathbb{R}}
\newcommand{\ZZ}{\mathbb{Z}}
\newcommand{\CC}{\mathbb{C}}
\newcommand{\NN}{\mathbb{N}}
\newcommand{\QQ}{\mathbb{Q}}
\newcommand{\PP}{\mathbb{P}}
\newcommand{\kk}{\mathsf{k}}
\newcommand{\FF}{\mathbb{F}}
\newcommand{\cS}{\mathcal{S}}
\newcommand{\cT}{\mathcal{T}}
\newcommand{\ssC}{\mathsf{C}}

\newcommand{\id}{\operatorname{id}}
\newcommand{\Mat}{\mathsf{Mat}}

\title{Math 411: Topics in Advanced Analysis\\ Homework due Wednesday Week 4}
% uncomment the following line and add your name if you are using this as a template for solutions
% \author{Your Name}

\begin{document}
\maketitle

\begin{prob}
Suppose that a sequence $(a_n)$ of real numbers is equidistributed mod 1, and that $c$ is a real number. Show that $(a_n+c)$ is equidistributed mod 1.
\end{prob}
% uncomment the following lines and add your solution
% \begin{proof}[Solution]

% \end{proof}

\begin{prob}
Let $\phi = (1+\sqrt 5)/2$ and $\bar\phi = (1-\sqrt 5)/2$. Set $a_n = \phi^n+\bar\phi^n$.
\begin{enumerate}[(a)]
\item Prove that for $a_0 = 2$, $a_1 = 1$, and for $n\ge 1$,
\[
  a_{n+1} = a_n+a_{n-1}.
\]
Deduce that $a_n\in \ZZ$ for all $n\in \NN$.
\item Prove that
\[
  \langle \bar\phi^{2r+1}\rangle\to 0\text{ as }\NN\ni r\to \infty
\]
and
\[
  \langle \bar\phi^{2r}\rangle\to 0\text{ as }\NN\ni r\to \infty.
\]
\item Deduce that
\[
  \lim_{n\to \infty}\frac{1}{n}\left|\{r\in \{1,\ldots,n\}\mid \langle \phi^r\rangle \in [1/4,3/4]\}\right|\to 0\text{ as }n\to \infty
\]
and thus $(\phi^r)$ is not equidistributed mod 1.
\end{enumerate}
\end{prob}
% uncomment the following lines and add your solution
% \begin{proof}[Solution]

% \end{proof}

\begin{prob}
Let $B_\ell(x)$ denote the $\ell$-th Bernoulli polynomial we defined in class.\footnote{Beware that many authors choose a different normalization of Bernoulli polynomials, for instance with $B_k'(x) = kB_{k-1}(x)$.}
\begin{enumerate}[(a)]
\item Prove that $B_\ell(x)$ satisfies
\[
  B_\ell(1-x) = (-1)^\ell B_\ell(x).
\]
In other words, $B_\ell$ is symmetric about $x=1/2$ for $\ell$ even, and skew-symmetric about $x=1/2$ for $\ell$ odd.
\item Suppose that $B_{\ell-1}(x) = \sum_n c_n (x-1/2)^n$. Prove that
\[
  B_\ell(x) =
  \begin{cases}
    \sum_n \frac{c_n}{n+1}(x-1/2)^{n+1} - \sum_n \frac{c_n}{(n+1)(n+2)}\cdot \frac{1}{2^{n+1}}&\text{if $\ell$ is even},\\
    \sum_n \frac{c_n}{n+1}(x-1/2)^{n+1}&\text{if $\ell$ is odd}.
  \end{cases}
\]
\end{enumerate}
\end{prob}
% uncomment the following lines and add your solution
% \begin{proof}[Solution]

% \end{proof}

\begin{prob}[continuation of Problem 3]
We now connect $B_{2\ell}(1/2)$ to $\zeta$-values.
\begin{enumerate}[(a)]
\item Making the same assumptions as we made in class, prove that
\[
  B_{2\ell}(1/2) = \frac{(-1)^\ell}{2^{\ell-1}\pi^{2\ell}}\sum_{n\ge 1} \frac{(-1)^{n+1}}{n^{2\ell}}.
\]
\item Prove that for $s\ge 2$ an integer,
\[
  \sum_{n\ge 1}\frac{(-1)^{n+1}}{n^s} = \zeta(s)-\frac{1}{2^{s-1}}\zeta(s)
\]
and thus
\[
  B_{2\ell}(1/2) = \frac{(-1)^\ell}{2^{2\ell-1}\pi^{2\ell}}\left(1-\frac{1}{2^{2\ell-1}}\right)\zeta(2\ell).
\]
\item Compute $B_3(x)$ as a polynomial in $(x-1/2)$ and use Problem 3(b) to deduce the value of $B_4(1/2)$. Use this to compute $\zeta(4)$.
\end{enumerate}
\end{prob}
% uncomment the following lines and add your solution
% \begin{proof}[Solution]

% \end{proof}

\begin{prob}
Let $\mathscr H$ and $\mathscr H'$ be Hilbert spaces, let $V\le \mathscr H$ be a subspace of $\mathscr H$, and suppose that $T\colon V\to \mathscr H'$ is linear. Consider the following properties:
\begin{itemize}
\item[(UC)] The operator $T$ is uniformly continuous on $V$.
\item[(C0)] The operator $T$ is continuous at $0\in V$.
\item[(B)] The operator $T$ is \emph{bounded}, \emph{i.e.}, there exists $M>0$ such that for all $f\in V$,
\[
  \| T(f) \| \le M\| f \|.
\]
\end{itemize}
Note that condition (UC) implies (C0) \emph{a fortiori}. Prove that all three conditions are equivalent by showing (C0) $\implies$ (B) $\implies$ (UC).
\end{prob}
% uncomment the following lines and add your solution
% \begin{proof}[Solution]

% \end{proof}

\end{document}