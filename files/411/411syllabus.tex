\documentclass[11pt,twoside]{amsart}
\usepackage{amssymb, amsmath, enumerate, libertine, hyperref,xcolor,booktabs,longtable,microtype}
\usepackage[normalem]{ulem}
\usepackage{fullpage}
\usepackage[T1]{fontenc}
\renewcommand{\labelitemi}{$\cdot$}
\usepackage[normalem]{ulem}
\definecolor{dark-red}{rgb}{0.4,0.15,0.15}
%   \definecolor{dark-blue}{rgb}{0.15,0.15,0.4}
%   \definecolor{medium-blue}{rgb}{0,0,0.5}
\setcounter{secnumdepth}{2}
\setcounter{tocdepth}{1}
\hypersetup{
    colorlinks, linkcolor=dark-red,
    citecolor=dark-red, urlcolor=dark-red
}

\title{Math 411: Topics in Advanced Analysis}
\author[Math 411: Topics in Advanced Analysis]{Spring 2025}

\begin{document}
\maketitle

%\vspace{-5mm}

\begin{center}
\fbox{
\begin{minipage}{4.25in}
\begin{tabular}{rl}
Place:  &Library 389\\
Time: &MWF, 9:00--9:50\textsc{a.m.}\\
Instructor: &Kyle Ormsby (\href{mailto:ormsbyk@reed.edu}{ormsbyk@reed.edu})\\
Office Hours: &TBD\\
Textbook: &None required; see recommendations below\\%\href{https://www.crcpress.com/Bilinear-Algebra-An-Introduction-to-the-Algebraic-Theory-of-Quadratic-Forms/Szymiczek/p/book/9789056990763}{\emph{Bilinear algebra}} and \href{https://bookstore.ams.org/gsm-67}{\emph{Introduction to quadratic forms over fields}} recommended\\
Website: &\href{http://people.reed.edu/~ormsbyk/412/}{kyleormsby.github.io/411/}
\end{tabular}
\end{minipage}
}
\end{center}

\smallskip

\subsection*{Course description}
This course will start with a geometric exploration of Fourier series and the Fourier transform, and then --- depending on student preference --- either head towards PDEs and applications, or towards harmonic analysis on locally compact Abelian groups and matrix groups.

\subsection*{Learning outcomes}
By the end of this course, you should be able to:
\begin{itemize}
\item reason with and solve problems involving function spaces, Hilbert spaces, Fourier series, convolutions, Dirac kernels, Fourier transforms, and the Plancherel theorem;
\item reason with and solve problems involving additional advanced analysis topics selected by the class (likely PDEs and test functions or harmonic analysis on LCAs and the Peter--Weyl theorem);
\item independently research and study an advanced topic in analysis;
\item communicate mathematics through writing and presentations.
\end{itemize}

\subsection*{Distribution requirements}
This course can be used towards your Group III, ``Natural, Mathematical, and Psychological Science,'' requirement.  It accomplishes the following goals for the group:
\begin{itemize}
\item Use and evaluate quantitative data or modeling, or use logical/mathematical reasoning to evaluate, test, or prove statements.
\item Given a problem or question, formulate a hypothesis or conjecture, and design an experiment, collect data or use mathematical reasoning to test or validate it.
\end{itemize}
This course \textbf{does not} satisfy the ``primary data collection and analysis'' requirement.

\subsection*{Participation}
I will deliver interactive in-person lectures. This course does not have a formal attendance policy, but I will use your engagement to assess the participation portion of your grade. If you miss a class, it is your responsibility to catch up with the material via the course notes and discussions with peers.

\subsection*{Texts}
The course has no assigned textbook.  The instructor will write course notes as the semester proceeds and post these to the course website.  Supplementary texts which may prove useful include \emph{Fourier Series, Fourier Transforms, and Function Spaces: A Second Course in Analysis} by Hsu, \emph{Fourier Analysis and its Applications} by Folland, \emph{A First Course in Harmonic Analysis} by Deitmar, and \emph{Principles of Harmonic Analysis} by Deitmar and Echterhoff.

\subsection*{Homework}
Homework is due most Wednesdays by 10\textsc{p.m.}~via Gradescope. Excellent solutions take many forms, but they all have the following characteristics:

\begin{itemize}
\item they are written as explanations for other students in the course; in particular, they fully explain all of their reasoning and do not assume that the reader will fill in details;
\item when graphical reasoning is called for, they include large, carefully drawn and labelled diagrams;
\item they are neatly written or typeset;\footnote{Interested students are 
encouraged to prepare solutions in the \LaTeX~document preparation 
system.  A guide to \LaTeX~resources is available on the course 
website.  Nearly all of the \texttt{.pdf} files on the course website are produced by \LaTeX; you can find their associated source files by changing the \texttt{.pdf} suffix to \texttt{.tex} in the URL.} and
\item they use complete sentences, even when formulas or symbols are involved.
\end{itemize}

Because of time constraints, I will mark some homework problems for completion and others (semi-randomly selected) fully.  Fully graded homework problems can earn up to five points for mathematical content and up to two points for writing.

\textbf{Given the exigencies of contemporary existence, I will be flexible with deadlines as long as you communicate with me about extensions.} If health, family, or emergent (inter)national crises might impede the timely completion of your homework, please contact me as early as possible.

\subsection*{Collaboration}
You are permitted and encouraged to work with your peers on homework problems.  You must cite those with whom you worked, and you must write up solutions independently.  \textbf{Duplicated solutions will not be accepted and constitute a violation of the Honor Principle.}

\subsection*{Revisions}
You may revise any submitted fully graded homework problem after receiving comments, and you will sometimes be encouraged to revise problems.  (You may not ``revise'' homework problems that were not turned in the first time.)  This will allow you the opportunity to perfect the skills required to solve the problems.  You may revise multiple times, and will receive the average of all of your scores.  Revisions must be turned in at most one week after you receive comments on the previous version of a solution.

\subsection*{Tests}
We will have one midterm exam and a final exam. All exams will be open book, open notes, and take-home.  They will have suggested time constraints and firm due dates, but you will be permitted to spend an arbitrary amount of time on the problems between when the exam is distributed and collected. 

\begin{itemize}
\item Midterm Exam (tentative): distributed Monday 3 March, due Wednesday 5 March.
\item Final Exam: distributed Monday 12 May, due Wednesday 14 May.
\end{itemize}

\subsection*{Presentations and final paper}
Each student will give a 20-minute presentation on an advanced topic in the final two weeks of the course.  You will also submit a final paper on the same topic.  More details will follow, but expect that topics will be assigned early, that you will give a practice presentation, and that comments will be given on draft papers (prepared in \LaTeX) before the final version is submitted.

\subsection*{Joint expectations}
As members of a communal learning environment, we should all expect consideration, fairness, patience, and curiosity from each other.  Our aim is to all learn through cooperation and genuine listening and sharing, not to compete or show off.  I expect diligence and academic and intellectual honesty from each of you.  You should expect that I will do my best to focus the course on interesting, pertinent topics, and that I will provide feedback and guidance which will help you excel as a student.

\subsection*{Help}
Everyone is welcome and encouraged to attend my drop-in hours, times TBD, in Library 306.  If you are unable to make these times, I am happy to schedule alternate times at which to meet with you --- just ask!

\subsection*{Zulip}
Our section of Math 411 has a Zulip workspace.  Use the Zulip workspace to ask questions (of me or the class), collaborate on problems, and share resources. The Zulip workspace is an extension of our classroom and the above joint expectations extend to this setting.

Follow \href{https://math411-2025.zulipchat.com/join/6tyxsz2asnf5zum6bfblf2qu/}{this invitation link} to join the Zulip workspace. Please use channels and threads to keep conversations organized.

\subsection*{Technology, LLMs, ``AI'', \& the Internet}
The use of electronic devices (cell phones, computers, tablets, 
calculators, \emph{etc}.) is prohibited in the classroom without prior authorization from the instructor.  That said, legitimate uses 
of technology (\emph{e.g.}, note-taking) will be accommodated --- 
just talk to me first.

You are welcome to use Internet resources to supplement content we cover in this course, with the exception of solutions to homework or exam problems.  \textbf{Copying solutions from the Internet or an AI chatbot is an Honor Principle violation and will result in an academic misconduct report.}

Why do I have this policy? Your job as a student is to think, and my job as a teacher is to inquire into your thinking so that I can help you think more like an expert. If you submit computer-generated solutions, then you aren't doing your job, and I can't do mine.

\subsection*{Academic accommodations}
If you have a documented disability requiring academic accommodation, please have  Disability \& Accessibility Resources (DAR)  provide a letter during the first week of classes.  I will then contact you to schedule a meeting during which we can discuss your accommodations.  If you believe you have an undocumented disability and that accommodations would ensure equal access to your Reed education, I would be happy to help you contact DAR.

\subsection*{Grades}
Your grade will reflect a composite assessment of the work you produce for the class, weighted in the following fashion:  $r$\% homework, $s$\% midterm exam, $t$\% final exam, $u$\% final paper and presentation, and $v$\% class participation, where $r+s+t+u+v = 100$ and $r,s,t,u,v\ge 0$. We will discuss and decide upon the values of $r,s,t,u,v$ in class.

\bigskip \bigskip

\begin{center}
Remember: \emph{Math is hard, but we'll get through this together!}
\end{center}


\end{document}