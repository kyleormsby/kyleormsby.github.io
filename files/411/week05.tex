\documentclass[11pt,twoside]{amsart}
\usepackage{amssymb, amsmath, enumerate, libertine, microtype, hyperref,tikz-cd}
\usepackage[normalem]{ulem}
\usepackage{fullpage}
\usepackage[T1]{fontenc}
\renewcommand{\labelitemi}{$\cdot$}
\usepackage{mathrsfs}
\usepackage{phaistos}


\theoremstyle{plain}
\newtheorem{prop}{Proposition}%[section]
\newtheorem{lemma}[prop]{Lemma}
\newtheorem{thm}[prop]{Theorem}
\newtheorem{obs}[prop]{Observation}
\newtheorem{app}[prop]{Application}
\newtheorem*{MainThm}{Main Theorem}
\newtheorem{cor}[prop]{Corollary}
\newtheorem{conj}[prop]{Conjecture}
\theoremstyle{remark}
\newtheorem{rmk}[prop]{Remark}
\newtheorem{prob}{Problem}
\newtheorem{bonus}[prop]{Bonus Problem}
\newtheorem{exc}{Exercise}
\theoremstyle{definition}
\newtheorem{ex}[prop]{Example}
\theoremstyle{definition}
\newtheorem{defn}[prop]{Definition}

\newcommand{\RR}{\mathbb{R}}
\newcommand{\ZZ}{\mathbb{Z}}
\newcommand{\CC}{\mathbb{C}}
\newcommand{\NN}{\mathbb{N}}
\newcommand{\QQ}{\mathbb{Q}}
\newcommand{\PP}{\mathbb{P}}
\newcommand{\kk}{\mathsf{k}}
\newcommand{\FF}{\mathbb{F}}
\newcommand{\cS}{\mathcal{S}}
\newcommand{\cT}{\mathcal{T}}
\newcommand{\ssC}{\mathsf{C}}

\newcommand{\id}{\operatorname{id}}
\newcommand{\Mat}{\mathsf{Mat}}

\title{Math 411: Topics in Advanced Analysis\\ Homework due Wednesday Week 5}
% uncomment the following line and add your name if you are using this as a template for solutions
% \author{Your Name}

\begin{document}
\maketitle

\begin{prob}
Suppose $u\colon S^1\times \RR_{>0}\to \RR$ is a solution to the heat equation on the circle with $u(x,t)\xrightarrow{t\to 0^+}f(x)\in L^2(S^1)$.
\begin{enumerate}[(a)]
\item Prove that for each $t>0$ there is a \emph{bounded} linear operator
\[
  T(t)\colon L^2(S^1)\longrightarrow L^2(S^1)
\]
satisfying
\[
  u(-,t) = T(t)f.
\]
\item Use the relationship between convolution and Fourier coefficients that you established on a previous homework to deduce that for each $t>0$ there is a function $g^t\in C^\infty(S^1)$ such that
\[
  T(t)f = g^t*f.
\]
You should include a formula for the Fourier series of $g^t$.
\item Prove that for $s,t>0$,
\[
  T(s)T(t) = T(s+t).
\]
(A family of operators with this property is called a \emph{one-parameter semi-group}.)
\item {[optional]} Explain a sense in which it might be reasonable to write
\[
  T(t) = e^{t\frac{\partial^2}{\partial x^2}}.
\]
\end{enumerate}
\end{prob}
% uncomment the following lines and add your solution
% \begin{proof}[Solution]

% \end{proof}

\begin{prob}
Suppose $u$ is a solution to the heat equation on the circle with initial condition $f$. Define
\[
\begin{aligned}
  v\colon S^1\times \RR_{>0}&\longrightarrow \RR\\
  (x,t)&\longmapsto u(x,-t).
\end{aligned}
\]
\begin{enumerate}[(a)]
\item Show that $v$ does not satisfy the heat equation in general, but does satisfy a closely related equation.
\item {[related but different]} Can the heat equation be solved ``backward in time''? Make sense of what this might mean formally and say something insightful about the class of initial conditions that admit such time-reversed solutions.
\end{enumerate}
\end{prob}
% uncomment the following lines and add your solution
% \begin{proof}[Solution]

% \end{proof}

\begin{prob}
Read Section 11.4 of Hsu's book, \emph{Fourier series, Fourier transforms, and function spaces}. Prove Theorem 11.4.4:
\begin{quote}
The set $\mathscr B_{\text{odd}} = \{\sin(2\pi n x)\mid n\ge 1\}$ is an eigenbasis for $\Delta$ with domain
\[
  \mathscr D(\Delta)_{\text{Dir}} = \left\{ f\in C^2([0,1/2])\mid  f(0) = f(1/2) = 0 \right\},
\]
and the set $\mathscr B_{\text{even}} = \{\cos(2\pi nx)\mid n\ge 0\}$ is an eigenbasis for $\Delta$ with domain
\[
  \mathscr D(\Delta)_{\text{Neu}} = \left\{ f\in C^2([0,1/2]) \mid f'(0)=f'(1/2)=0 \right\}.
\]
\end{quote}
\end{prob}
% uncomment the following lines and add your solution
% \begin{proof}[Solution]

% \end{proof}

\begin{prob}[continuation of Problem 3]
We will now compute some explicit solutions to the heat equation with Dirichlet and Neumann boundary conditions.
\begin{enumerate}[(a)]
\item Compute the sine series of the odd extension of $f(x)=1$ for $x\in [0,1/2]$ and use this to express the solution to the heat equation with Dirichlet boundary conditions and initial condition $f$.
\item Compute the cosine series of the even extension of $f(x)=x$ for $x\in [0,1/2]$ and use this to express the solution to the heat equation with Neumann boundary conditions and initial condition $f$.
\end{enumerate}
\end{prob}
% uncomment the following lines and add your solution
% \begin{proof}[Solution]

% \end{proof}


\end{document}